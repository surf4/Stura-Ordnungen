%\addchap[Richtlinie über die finanzielle Förderung studentischer Projekte der Studentenschaft der TU Dresden]{Richtlinie über die finanzielle Förderung studentischer Projekte\\der Studentenschaft der TU Dresden}
\markright{Richtlinie über die finanzielle Förderung studentischer Projekte}
\setcounter{section}{0} % ist nötig um den Paragrafenzähler zurücksetzen
\setcounter{sentence}{0} % ist nötig um den Satzzähler zurückzusetzen
\begin{multicols}{2}

\section{Förderausschuss}

\Abs \Satz Der Förderausschuss ist ein Ausschuss gemäß §~24 der Grundordnung\. Er besteht aus vier vom
StuRa gewählten StuRa-Mitgliedern und der Geschäftsführerin Finanzen.
\\ 

\Abs \Satz Der Förderausschuss entscheidet über die finanzielle Förderung studentischer Projekte laut §~33 der Finanzordnung und die Anerkennung von Hochschulgruppen gemäß Richtlinie zur Anerkennung von Hochschulgruppen.

\section{Haushaltsvorbehalt und Rechtsanspruch}
\Abs \Satz Eine Förderung erfolgt unter dem Vorbehalt verfügbarer Mittel im zugeordneten
Haushaltstitel.\\

\Abs \Satz Die Höhe der Förderung muss in Relation zur Gesamthöhe des Budgets liegen.\\

\Abs \Satz Auf eine Förderung besteht kein Rechtsanspruch.

\section{Grundsätzliches}
\Abs \Satz Projekte die gegen grundsätzliche Positionen des StuRa laufen werden nicht gefördert.\\

\Abs \Satz Der StuRa muss in Publikationen zum geförderten Projekt als Förderer genannt werden.\\

\Abs \Satz Kosten für Verpflegung werden nicht übernommen.\\

\Abs \Satz Materialien für den dauerhaften Gebrauch bleiben Eigentum der Studentenschaft
und werden nur als Dauerleihgaben vergeben.\\

\Abs \Satz Über dauerhafte Förderung über ein Wirtschaftsjahr hinaus entscheidet der StuRa gemäß
§~35 der Finanzordnung\. Der Förderausschuss gibt hierfür eine Empfehlung ab.\\

\Abs \Satz Genehmigte und nichtabgerufene Förderanträge verfallen 4 Monate nach Bewilligung.\\

\Abs \Satz Für die Abrechnung eines Förderantrages müssen alle tatsächlich angefallenen Einnahmen
und Ausgaben belegt werden.\\


\section{Öffentlichkeit}
\Abs \Satz Veranstaltungen und Exkursionen werden nur gefördert, wenn diese ausreichend beworben
werden und die Teilnahme grundsätzlich allen Studentinnen möglich ist.\\

\Abs \Satz Für Veranstaltungen und Exkursionen kann eine Eigenbeteiligung der Teilnehmerinnen
vorgesehen werden\. Die Höhe der Eigenbeteiligung darf nicht sozial Selektiv wirken.\\

\Abs \Satz Vom StuRa geförderte Veranstaltungen müssen barrierefrei sein\. Ist die Barrierefreiheit
nicht möglich, muss dies kurz und schriftlich erklärt werden.

\section{Sport}
\Abs \Satz Der StuRa fördert den freiwilligen Studierendensport finanziell\. Dazu gehören insbesondere
die Übernahme der Kosten von Sachpreisen und Mieten bei Turnieren, von Fahrtkosten zu Wettbewerben und von Werbungskosten für Veranstaltungen.\\

\Abs \Satz Der Wirtschaftsplan sieht einen eigenen Titel für Sportförderung vor.

\section{Lehrveranstaltungen und Exkursionen}
\Abs \Satz Kosten für Seminare, Ringvorlesungen und Exkursionen für die es Leistungsnachweise gibt
oder die zum Studienablauf gehören, werden nur übernommen wenn sie hauptsächlich der
Erfüllung der Aufgaben der Studentenschaft laut SächsHSG dienen.

\section{Partys}
\Abs \Satz Der StuRa fördert keine Partys großer Dimension.\\

\Abs \Satz Partys werden nur in Form von Ausfallbürgschaften gefördert\. Der vom StuRa gedeckte
Anteil beträgt höchstens die Hälfte des gesamten Fehlbetrags, maximal jedoch 500 Euro.\\

\Abs \Satz Stehen der Veranstalterin mehrere Bürgen zur Finanzierung des Fehlbetrages zur
Verfügung, übernimmt der StuRa nur einen der Anzahl der Bürgen entsprechenden Anteil am
Fehlbetrag.

\section{Förderung der Fachschaften}
\Abs \Satz Projekte einer Fachschaft werden nur gefördert wenn deren Rücklage (über 1500 Euro) das
Dreifache der Semestereinnahmen nicht übersteigt.\\

\Abs \Satz Der StuRa zahlt nicht mehr als der jeweilige FSR, sofern der FSR über weniger
als 100 Euro Guthaben verfügt.\\

\Abs \Satz Büroausstattung und Rechentechnik muss durch den FSR eigenständig finanziert werden.\\

\Abs \Satz Der Wirtschaftsplan sieht einen eigenen Titel für die Förderung der Fachschaften vor.\\

\Abs \Satz Bei Veranstaltungen von mehr als einem FSR gilt Abs. (1) nicht.
\end{multicols}

\nopagebreak
\vspace{1cm}
Inkraftgetreten am 16.~Februar~2009.
\\

\normalsize
~\\*[4cm]
\begin{center}
\hspace*{\fill}
\parbox{7cm}{Armin Grundig\\GF Soziales}
\hfill\parbox{7cm}{Enrico Lovasz\\GF Finanzen}
\hspace*{\fill}
\end{center}
