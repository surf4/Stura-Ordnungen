%\addchap[Rechnernutzungsrichtlinie des Studentenrates der TU~Dresden]{Rechnernutzungsrichtlinie\\des Studentenrates der TU Dresden}
\markright{Rechnernutzungsrichtlinie}
\setcounter{section}{0}
\begin{multicols}{2}


\section{Geltungsbereich}

\Abs \Satz Die Rechnernutzungsrichtlinie (IT-Richtlinie) gilt für für alle rechen- und kommunikationstechnischen Installationen des Studentenrates (StuRa)\. Weiterhin ist diese Richtlinie im Umgang mit Daten und Diensten des Studentenrates einzuhalten. 

\Abs \Satz Neben dieser Richtlinie sind die Nutzungsbedingungen übergeordneter Netzdienste und Dienstanbieter, insbesondere des Zentrums für Informationsdienste und Hochleistungsrechnen der TU Dresden (ZIH) und des Deutschen Forschungsnetzes (DFN), sowie geltende gesetzliche Bestimmungen einzuhalten.


\section{Begriffsbestimmung}

\Abs \Satz Unter rechen- und kommunikationstechnischen Installationen werden grundsätzlich alle Arbeitsrechner, Server und Peripheriegeräte aufgefasst, die im Besitz oder Eigentum des StuRas sind oder durch das Referat Technik betreut werden\. Weiterhin fallen unter dieser Definition alle Installationen, Geräte und Einrichtungsgegenstände, die eine Nutzung der IT-Infrastruktur ermöglichen oder unterstützen. 

\Abs \Satz Unter rechen- und kommunikationstechnischen Installationen werden grundsätzlich alle Arbeitsrechner, Server und Peripheriegeräte aufgefasst, die im Besitz oder Eigentum des StuRas sind oder durch das Referat Technik betreut werden\. Weiterhin fallen unter dieser Definition alle Installationen, Geräte und Einrichtungsgegenstände, die eine Nutzung der IT-Infrastruktur ermöglichen oder unterstützen. 

\Abs \Satz Daten sind alle Informationen, die im Rahmen der Aufgabenerfüllung des Studentenrates oder aufgrund der Nutzeraktivität mittels der IT-Infrastruktur verarbeitet oder gespeichert werden. 

\Abs \Satz Unter Dienste des Studentenrates werden alle vom StuRa zur Verfügung gestellten Server- und Rechnerfunktionen verstanden\. Der Nutzerkreis von Diensten des StuRas kann auf definierte Personengruppen eingeschränkt werden. 

\Abs \Satz Software sind Programme oder Programmteile, die für den StuRa lizenziert und mittels der IT-Infrastruktur des StuRas für die Benutzung zur Verfügung gestellt werden.

\Abs \Satz Als Zugang wird die persönliche Zugangskennung eines Nutzers definiert. 
 

\section{Nutzung, Zugang und Dauer}

\Abs \Satz Die Nutzung der IT-Infrastruktur erfolgt grundsätzlich durch personengebundene Zugänge\. Die Einrichtung eines Zuganges erfolgt durch die mit der Account-Verwaltung beauftragten Personen des Studentenrates oder durch das Referat Technik.

\Abs \Satz Vor Erteilung eines Zuganges muss die IT-Richtlinie anerkannt werden\. Die Anerkennung ist schriftlich aktenkundig zu machen. 

\Abs \Satz Folgenden Personenkreisen kann ein Zugang erteilt werden: 
\begin{enumerate}
\item Mitglieder der verfassten Studentenschaft der TU Dresden
\item vom StuRa beauftragte Personen, sofern zur Auftragserfüllung die Nutzung der IT-Infrastruktur erforderlich ist
\end{enumerate}

\Abs \Satz Die Erteilung eines Zuganges ist grundsätzlich auf die Dauer eines Semesters befristet\. Auf Beschluss des StuRa-Plenums oder der Geschäftsführung, insbesondere durch Wahrnehmung eines Wahlamtes oder einer Entsendung, kann eine weiterreichende Befristung erteilt werden. 


\section{Entzug des Zuganges}

\Abs \Satz Der Zugang zur IT-Infrastruktur ist zu entziehen, wenn 
\begin{enumerate}
\item der Nutzer es selbst wünscht
\item der Nutzer nicht mehr einer Zuteilung eines Zuganges nach §~3~(3) berechtigt ist
\item der Nutzer die Anerkennung der IT-Richtlinie zurückzieht.
\end{enumerate}
 
\Abs \Satz Bei Verstößen gegen die IT-Richtlinie oder den übergeordneten Bestimmungen durch einen Nutzer kann auf Beschluss der Geschäftsführung oder des StuRa-Plenums ihm der Zugang entzogen werden\. Ist durch den Verstoß die Integrität der IT-Infrastruktur gefährdet, erfolgt eine sofortige Sperre durch das Referat Technik\. Der Vorfall ist der Geschäftsführung zu melden. 


\section{Verarbeitung personenbezogener Daten}

\Abs \Satz Gemäß §~14~(4) des Sächsischen Hochschulfreiheitsgesetzes (SächsHSFG) in Verbindung mit §~4~(1)~Punkt~2 des Sächsischen Datenschutzgesetzes (SächsDSG) werden für die Nutzung der IT-Infrastruktur personenbezogene Daten vom Nutzer erhoben. 

\Abs \Satz Für die Erteilung eines Zuganges werden vom Nutzer folgende personenbezogene Daten erhoben, verarbeitet und gespeichert: 
\begin{enumerate}
\item Vorname und Name
\item E-Mail-Adresse
\item Angaben über den Tätigkeitsbereich im Studentenrat
\item Mitgliedsstatus in der verfassten Studentenschaft
\item Beginn und Ende Studentenstatus an der TU Dresden 
\item Durch die Benutzung entstehende dienst-spezifische Metadaten, insbesondere Zeitpunkt des letzten Logins und Anzahl der versuchten Passworteingaben.
\end{enumerate}

\section{Rechte und Pflichten des Nutzers}

\Abs \Satz Der persönliche Zugang darf nur vom Nutzer selbst benutzt werden, eine Weitergabe an Dritte ist ein Verstoß gegen die IT-Richtlinie\. Beim Verlassen des Rechnerarbeitsplatzes ist der dieser so zu hinterlassen, dass eine Nutzung des Zuganges durch Dritte nicht möglich ist.

\Abs \Satz Private Tätigkeiten sind gegenüber inhaltlichen Arbeiten zurückzustellen. 

\Abs \Satz Neben den Geschäftsführern sind auch nachrangig die Referenten und die Referatsmitglieder berechtigt, die Nutzung der Rechner jederzeit zu verlangen, insofern sie den StuRa betreffende Aufgaben zu erledigen haben\. Daraufhin sind die betreffenden Rechner freizugeben.

\Abs \Satz Die Rechner sind bei Systemwartungsarbeiten des Referates Technik sofort freizugeben. 

\Abs \Satz Dem Nutzer ist nicht gestattet, auf den installierten Speichermedien nicht für den StuRa lizenzierte Programme (auch keine Spiele und Schriften) abzulegen. 

\Abs \Satz Dem Nutzer ist nicht gestattet, Änderungen an der installierten Software, insbesondere Betriebssystem, Anwendungen, Schriftarten, und den Systemeinstellungen selbst vorzunehmen. Änderungswünsche sind dem Referat Technik mitzuteilen und von diesem nach Prüfung gegebenenfalls umzusetzen. 

\Abs \Satz Zum Speichern von Daten sind ausschließlich die vom Referat Technik dafür vorgesehene Verzeichnisse zu nutzen\. Auf den allgemein zugänglichen Netzlaufwerken dürfen ausnahmslos nur Daten gespeichert werden, die dem StuRa direkt zuzuordnen sind. 

\Abs \Satz Die Speicherung von Daten bei externen Dienstleistern, die nicht die Voraussetzung gemäß § 7 SächsDSG erfüllen, ist zu unterlassen\. Hierzu zählen insbesondere Webseiten und Cloud-Dienste Dritter. 

\Abs \Satz Hinweise auf Fehler in der installierten Software, unsachgemäße Nutzung von Laufwerken, sonstige Störungen und der Verdacht auf Viren müssen umgehend den Mitgliedern des Referates Technik oder den Angestellten mitgeteilt werden. 


\section{Haftung}

\Abs \Satz Die Nutzung der IT-Infrastruktur erfolgt eigenverantwortlich. 

\Abs \Satz Ansprüche Dritter, die sich auf Handlungen des Nutzers begründen, sind von im selbst zu regulieren\. Hierzu zählen insbesondere Verstöße des Nutzer gegen das Urheber- und Markenrecht. 

\Abs \Satz Der Nutzer haftet gegenüber dem StuRa in Höhe des entstandenen Sachschadens bei vorsätzlicher oder grob fahrlässiger Beschädigung der IT-Infrastruktur. 



\end{multicols}

\nopagebreak
\vspace{1cm}
Inkraftgetreten am 24.~April~2014.
\\



\normalsize
~\\*[4cm]
\begin{center}
\hspace*{\fill}
\parbox{7cm}{Matthias Funke\\GF Finanzen}
\hfill\parbox{7cm}{Andreas Spranger\\GF Hochschulpolitik}
\hspace*{\fill}
\end{center}