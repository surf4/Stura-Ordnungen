%Header Geschäftsordnung multicol
\markright{Geschäftsordnung}
\setcounter{section}{0} % ist nötig um den Paragrafenzähler zurücksetzen

\begin{multicols}{2}

\section{Konstituierung}

\Abs \Satz Die konstituierende Sitzung findet in der zweiten Woche nach Bekanntgabe der Ergebnisse der Wahlen der FSR statt.



\section{Zusammentreten}

\Abs \Satz Der Stura tagt donnerstags von 19.30~Uhr~bis~23.00 Uhr\. Einer gesonderten Einladung bedarf es nicht.

\Abs \Satz In der Woche nach der Wahl der FSR findet keine Sitzung statt.

\Abs \Satz Als Einladung für Sondersitzungen nach §~22 der Grundordnung gilt die fristgemäße Versendung einer E-Mail an das StuRa-Mitglied\. Auf Wunsch eines StuRa-Mitgliedes kann ihm die Einladung auch per Telefon, Fax oder auf dem Postweg (als fristwahrend gilt hier der Poststempel) zugestellt werden.



\section{Öffentlichkeit}

\Abs \Satz Die Sitzungen des StuRa sind grundsätzlich öffentlich\. Alle Anwesenden haben das Rederecht.

\Abs \Satz Angelegenheiten, die die Persönlichkeitsphäre oder die Angestellten des StuRa betreffen, sind in nicht-öffentlicher Sitzung zu behandeln.

\Abs \Satz Für den nicht-öffentlichen Teil sind die Anwesenden zur Verschwiegenheit verpflichtet.



\section{Beschlussfähigkeit}

\Abs \Satz Nach Eröffnung der Sitzung sind die Anwesenheit der Mitglieder und die Beschlussfähigkeit festzustellen.



\section{Sitzungsvorlagen und Fristen}

\Abs \Satz Die Sitzungsvorlagen an die StuRa-Mitglieder bestehen aus:
\begin{itemize}
\item zu behandelnden ordentlichen Anträgen nach §~10,
\item Kandidaturen,
\item dem Vorschlag zur Tagesordnung,
\item den Rechenschaftsberichten nach §~19,
\item den Beschlüssen der Geschäftsführung und der Ausschüsse,
\item dem Protokoll der Sitzungen der Geschäftsführung und der Ausschüsse,
\item aus unbestätigten Protokollen,
\item aus weiteren Vorlagen zu einzelnen Tagesordnungspunkten.
\end{itemize}

\Abs \Satz Die Sitzungsvorlagen müssen den Mitgliedern drei Tage vor Beginn der Sitzung zugänglich gemacht werden.

\Abs \Satz Initiativanträge müssen vor Sitzungsbeginn eingereicht werden.



\section{Tagesordnung}

\Abs \Satz Zu Beginn der Sitzung ist der Tagesordnungsvorschlag des Sitzungsvorstands vorzustellen und über Änderungsanträge zu beschließen\. Danach ist die Tagesordnung zu verabschieden.

\Abs \Satz Die Tagesordnung muss ein Verzeichnis aller vorliegenden Anträge, sowie deren Zuordnung zu Tagesordnungspunkten enthalten\. Sie muss folgende Punkte vorsehen:
\begin{enumerate}
\item die Genehmigung der vorliegenden Protokolle,
\item Bericht der Geschäftsführung und Debatte des Berichts,
\item Sonstiges.
\end{enumerate}

\Satz Die Punkte~1~und~2 dürfen nur auf ordentlichen Sitzungen behandelt werden.

\Abs \Satz In der Regel sind für Anträge eigene Tagesordnungspunkte einzurichten\. Tages\-ordnungspunkte, die unter Ausschluss der Öffentlichkeit behandelt werden, sind nach Möglichkeit an das Ende der Sitzung zu legen.

\Abs \Satz Abweichend von Abs.~1 ist auf außerordentlichen Sitzungen der TO-Vorschlag der Antragstellerinnen, so wie er im Beschluss der Sondersitzung enthalten ist, vorzustellen\. Änderungsanträge dürfen nur die Gliederung der außerordentlichen Sitzung betreffen.



\section{Versammlungsleiterin}

\Abs \Satz Die Versammlungsleiterin hat die Kompetenzen aus §~23 der Grundordnung.

\Abs \Satz Die Versammlungsleiterin strukturiert die Sitzung gemäß der Tagesordnung\. Sie kann Pausen nach eigenem Ermessen vorsehen.

\Abs \Satz Die Versammlungsleiterin stellt fest, wann die Behandlung eines Tagesordnungspunktes oder die Durchführung einer Wahl oder Beschlussfassung beginnt und endet.

\Abs \Satz Sie hat das Recht, einen Antrag nach ihrem Ermessen aufzugliedern und entsprechend diskutieren zu lassen.

\Abs \Satz Die Versammlungsleiterin erteilt das Wort\. Sie kann die Redezeit begrenzen, eine Rednerin zur Sache oder zur Form rufen\. Kommt eine Rednerin einer solchen Aufforderung nicht nach, kann die Versammlungsleiterin ihr das Wort entziehen.

\Abs \Satz Bei Diskussionen oder Beschlüssen, die die Versammlungsleiterin selbst betreffen, hat sie die Versammlungsleitung abzugeben.

\Abs \Satz Die Auslegung der Geschäftsordnung obliegt, mit Wirkung für die aktuelle Sitzung, der Versammlungsleiterin, gegebenenfalls nach Beratung des Sitzungsvorstands.



\section{Redeliste}

\Abs \Satz Vor Beginn einer Diskussion bittet die Versammlungsleiterin um Wortmeldungen und bildet eine Redeliste\. Nach dieser erteilt sie das Wort und ergänzt sie während der Debatte.

\Abs \Satz Vor der Debatte eines Antrags erteilt die Versammlungsleiterin der Antragstellerin das Wort\. Nach der Vorstellung des Antrags kann die Geschäftsführung zum Antrag Stellung nehmen.

\Abs \Satz Die Redeliste kann nach Ermessen der Versammlungsleiterin unterbrochen werden:
\begin{enumerate}
\item durch Wortmeldung der Antragstellerin bzw. Berichterstatterin zu diesem Tagesordnungspunkt und
\item durch Wortmeldungen der Geschäftsführung sofern Fragen an sie gerichtet sind.
\end{enumerate}

\Abs \Satz Es gilt das Erstrednerinnenrecht.

\Abs \Satz Eine Sitzungsteilnehmerin darf nur sprechen, wenn ihr die Versammlungsleiterin das Wort erteilt\. Möchte die Versammlungsleiterin selbst zur Sache sprechen, so setzt sie sich an das derzeitige Ende der Redeliste.



\section{Anträge zur Geschäftsordnung}

\Abs \Satz Anträge zur Geschäftsordnung gehen allen anderen Wortmeldungen vor\. Sie können nur von StuRa-Mitgliedern gestellt werden und sind durch das Erheben beider Hände zu kennzeichnen.

\Abs \Satz Ein Redebeitrag, eine Wahl oder Abstimmung darf durch einen Geschäftsordnungsantrag nicht unterbrochen werden.

\Abs \Satz Über Geschäftsordnungsanträge ist sofort zu beschließen.

\Abs \Satz Als Geschäftsordnungsanträge sind folgende Anträge anzusehen:
\begin{enumerate}
\item Änderung der beschlossenen Tagesordnung;
\item Schluss der Debatte, gegebenenfalls sofortige Beschlussfassung;
\item Ausschluss der Öffentlichkeit;
\item Abweichung von einzelnen Punkten der Geschäftsordnung;
\item Verlängerung der Sitzung um eine Stunde;
\item Auszählung, gegebenenfalls erneute Auszählung, der Stimmen;
\item erneute Feststellung der Beschlussfähigkeit;
\item fünfminütige Beratungspause;
\item Geheime Abstimmung;
\item einmaliger sofortige Richtigstellung,
\item Personaldebatte;
\item Schluss der Redeliste;
\item Zulassung Einzelner zur geschlossenen Sitzung;
\item Nichtbefassung eines Antrages;
\item Beschränkung der Redezeit;
\item schriftliche Abstimmung;
\item Vertagung eines Punktes der Tagesordnung.
\end{enumerate}

\Abs \Satz Anträge nach Abs.~4~Nr.~1~-~5 bedürfen einer \nicefrac{2}{3}~Mehrheit der anwesenden Mitglieder.

\Abs \Satz Bei einem Geschäftsordnungsantrag nach Abs.~4~Nr.~6~-~10 ist kein Widerspruch zulässig.

\Abs \Satz Der Geschäftsordnungsantrag nach Abs.~4~Nr.~6 muss unmittelbar nach erfolgter Abstimmung gestellt werden.

\Abs \Satz Geschäftsordnungsanträge Nr.~6~und~7 können auch kombiniert gestellt werden.

\Abs \Satz Beratungspausen können einmal pro Tagesordnungspunkt beantragt werden.

\Abs \Satz Personaldebatten finden unter Ausschluss der Öffentlichkeit und der Betroffenen statt.

\Abs \Satz Vor Schluss der Redeliste ist jedem Mitglied des StuRa Gelegenheit zu geben, sich noch auf diese setzen zu lassen.

\Abs \Satz Vertagungen nach § 9 (4) Satz 1 Nummer 17 können mit Terminen und Bedingungen versehen werden. Geschieht dies nicht, werden sie auf die nächste Sitzung vertagt.



\section{Anträge}

\Abs \Satz Neben den Anträgen nach §~9 sind folgende Anträge an den Studentenrat zulässig:
\begin{enumerate}
\item ordentliche Anträge,
\item Initiativanträge,
\item Änderungsanträge,
\item Antrag auf Neubefassung.
\end{enumerate}


\Abs \Satz Alle Anträge nach Abs.~1 sind schriftlich zu stellen\. Sie enthalten den Namen der Antragstellerin, den Antragstext und gegebenenfalls eine Begründung\. Anträge mit dem Ziel eine Finanzwirksamkeit für den StuRa zu entfalten, müssen zusätzlich eine Finanzaufstellung enthalten\. Anträge auf Einrichtung oder Änderung eines StuRa-Projektes müssen insbesondere die Namen der Projektsprecherin und der Mitarbeiterinnen enthalten.

%Absatznummerierung in 2a sollte besser gemacht werden
(2a) Die Rücknahme von Anträgen durch die Antragstellerin ist jederzeit zulässig.

\Abs \Satz Ordentliche Anträge, die vom StuRa behandelt werden, werden beim Sitzungsvorstand eingereicht\. Für Ordentliche Anträge nach Abs.~1~Nr.~1 gelten die Fristen aus §~5.

\Abs \Satz Der Initiativantrag ist der Form und dem Inhalt nach ein ordentlicher Antrag, der die Fristen für ordentliche Anträge gemäß §~5~Abs.~1~und~2 nicht erfüllt\. Für sie gilt §~5~Abs.~3. Er bedarf der Unterschrift sieben stimmberechtigter Mitglieder.

\Abs \Satz Änderungsanträge sind Anträge zu ordentlichen Anträgen, die diese in ihrer Sache oder Ausgestaltung ändern\.  Änderungsanträge werden beim Sitzungsvorstand eingereicht\. Über sie ist vor dem Hauptantrag zu beschließen\. Soweit der StuRa den Änderungsanträgen zustimmt oder sie von der Hauptantragsstellerin übernommen werden, wird der Hauptantrag in der geänderten Fassung zur Beschlussfassung gestellt\.  Die Antragstellerin des Hauptantrages hat bis zur endgültigen Beschlussfassung das Recht, auch eine geänderte Fassung ihres Antrages zurückzuziehen.

\Abs \Satz Anträge auf Neubefassung dürfen nur in Fällen nach § 20, Abs. 5 GrO und nur im Tagesordnungspunkt "`Bericht der Geschäftsführung und Debatte des Berichts"' gestellt werden\. Für sie gelten nicht die Fristen nach § 5.



\section{Lesungen}

\Abs \Satz Für Änderungen der Grundordnung und deren Ergänzungsordnungen sind drei Lesungen erforderlich\. Für die Aufstellung des Haushaltsplanes sind nur zweite und dritte Lesung erforderlich.

\Abs \Satz In der ersten Lesung wird der Antrag nur dem Grundsatz nach besprochen\.  Änderungsanträge dürfen entgegen §~10 nicht gestellt werden\. Am Ende der ersten Lesung beschließt der StuRa über die Überweisung in die zweite Lesung\. Diese findet im Anschluss statt.

\Abs \Satz In der zweiten Lesung wird der Antrag inhaltlich zur Diskussion gestellt\. Am Ende der zweiten Lesung beschließt der StuRa über die Überweisung in die dritte Lesung\. Diese erfolgt in der nächsten ordentlichen Sitzung.

\Abs \Satz In der dritten Lesung wird der Antrag erneut inhaltlich zur Diskussion gestellt\. Abschließend wird der Antrag verlesen und darüber beschlossen.



\section{Beschlussfassung}

\Abs \Satz Die Versammlungsleiterin eröffnet nach Abschluss der Beratung und Wiederholung der Anträge die Beschlussfassung.

\Abs \Satz Änderungsanträge sowie Redebeiträge sind von diesem Zeitpunkt an nicht mehr zulässig\. Das Recht auf Anträge zur Geschäftsordnung nach §~9~Abs.~4~Nr.~9~und~16 bleibt unberührt.

\Abs \Satz Soweit für einen Beschluss nicht eine einfache Mehrheit erforderlich ist, hat die Versammlungsleiterin vor der Beschlussfassung darauf hinzuweisen und die abgegeben Stimmen auszuzählen.

\Abs \Satz Ein Antrag gilt als beschlossen, wenn ihm nicht auf Nachfrage der Versammlungsleiterin widersprochen wird\. Der Widerspruch muss nicht begründet werden (formale Gegenrede).

\Abs \Satz Bei Widerspruch führt die Versammmlungsleiterin unverzüglich durch Abfrage von Zustimmung, Ablehnung und Stimmenthaltung die Abstimmung durch\. Die Abstimmung erfolgt in der Regel durch Handzeichen.

\Abs \Satz Die Abstimmung wird ohne erneute Aussprache einmal wiederholt, wenn die einfache Mehrheit der abgegebenen Stimmen Enthaltungen sind, außer wenn keine einzige Ja-Stimme abgegeben wurde.

\Abs \Satz Das Stimmrecht darf nur von anwesenden Mitgliedern des StuRa ausgeübt werden.

\Abs \Satz Liegen konkurrierende Anträge vor, so hat die Versammlungsleiterin die Beschlussfassung wie folgt durchzuführen:
\begin{enumerate}
\item Geht ein Antrag weiter als ein anderer, so ist über den weitergehenden zuerst zu beschließen. Wird dieser angenommen, so sind weniger weitgehende Anträge erledigt.
\item Lässt sich ein Weitergehen im Sinne von Nr.~1 nicht feststellen, so bestimmt sich die Reihenfolge, in der konkurrierende Anträge gestellt werden, nach der Reihenfolge der Antragstellung.
\end{enumerate}



\section{Schriftliche Abstimmungen}

\Abs \Satz Schriftliche Abstimmungen erfolgen mittels zugängiger Abstimmungsliste.

\Abs \Satz Die Abstimmungsliste enthält die zu Beginn der Abstimmung stimmberechtigten Mitglieder.

\Abs \Satz Schriftliche Abstimmungen können nur zu Gegenständen erfolgen, die mehr als eine einfache Mehrheit erfordern.

\Abs \Satz Die schriftliche Abstimmung ist mindestens bis zum Ablauf des auf die nächste Sitzung folgenden Tages zu ermöglichen, höchstens jedoch drei Wochen, außer in der vorlesungsfreien Zeit\. Die Abstimmungsdauer beschließt der StuRa unmittelbar nach dem Beschluss der schriftlichen Abstimmung.

\Abs \Satz Auf eine schriftliche Abstimmung und den Abstimmungsort ist auf der nächsten Sitzung sowie im Protokoll gesondert hinzuweisen.



\section{Geheime Abstimmungen}

\Abs \Satz Zur Durchführung von geheimen Abstimmungen bildet der StuRa eine Zählkommission\. Diese wird in der Regel für die Dauer einer Sitzung bestätigt.

\Abs \Satz Die Zählkommission hat aus mindestens drei Mitgliedern zu bestehen, die selbst nicht an der Abstimmung teilnehmen.

\Abs \Satz Die Zählkommission verteilt die Stimmzettel und sammelt sie ein\. Sie öffnet und schließt die erforderlichen Wahlgänge\. Sie zählt die Stimmen aus und verkündet dem StuRa das Abstimmungsergebnis\. Sie entscheidet bei Zweifeln über die Gültigkeit eines Stimmzettels.



\section{Schriftliche, geheime Abstimmungen}

\Abs \Satz Bei schriftlichen, geheimen Abstimmungen finden die Bestimmungen der §§~13~und~14 Anwendung\. Zusätzlich gilt:
\begin{enumerate}
\item Die Zugängigkeit zur Abstimmung gilt als gesichert, wenn der Abstimmungsort während der Arbeitszeiten der Kassenwärtin zugängig ist. In diesem Fall ist sicherzustellen, dass zu den Abstimmungszeiten mindestens ein Mitglied der Zählkommission im Abstimmungsraum anwesend ist.
\item Die Teilnahme an der Abstimmung wird durch Unterschrift bestätigt. Auf Verlangen eines Mitglieds der Zählkommission ist vor der Stimmabgabe ein Ausweisdokument vorzulegen.
\end{enumerate}



\section{Ausschreibungen}

\Abs \Satz Der StuRa schreibt zu Beginn einer neuen Legislatur alle Posten aus.

\Abs \Satz Die Posten gemäß §~16,~Abs.~2,~Nr.~4 der Grundordnung müssen ausgeschrieben werden.

\Abs \Satz Die Ausschreibungen erfolgen mit einer Dauer von mindestens zwei Wochen\. Nicht besetzte Posten bleiben bis auf weiteres ausgeschrieben.

\Abs \Satz Nach Rücktritt oder Abwahl ist sofort erneut auszuschreiben.



\section{Wahlen}

\Abs \Satz Kandidaturen auf ausgeschriebene Posten werden beim Sitzungsvorstand eingereicht.

\Abs \Satz Liegt für einen ausgeschriebenen Posten eine Kandidatur vor, findet auf der nächsten ordentlichen Sitzung eine Wahl statt\. Es gelten die Fristen nach §§~5~und~16.

\Abs \Satz Kandidatinnen können nur in Anwesenheit, einzeln und funktionsgebunden gewählt werden\. Als Geschäftsführerin kann nur gewählt werden, wer für die Wahlsitzung durch einen Fachschaftsrat in den Studentenrat entsendet ist\. Kandidaturen können jederzeit zurückgezogen werden.

\Abs \Satz Jedes Mitglied der Studentenschaft kann Fragen an die Kandidatinnen stellen\. Dies ist auch zwischen zwei Wahlgängen möglich.

\Abs \Satz Im ersten und zweiten Wahlgang ist die Mehrheit der Mitglieder erforderlich\. §~19~Abs.~2 der Grundordnung findet dabei keine Anwendung\. Soweit die erforderliche Mehrheit im ersten bzw. zweiten Wahlgang nicht erreicht wurde, erfolgt ein weiterer Wahlgang.

\Abs \Satz Wahlen finden durch geheime Abstimmung statt\. Eine Kandidatin ist gewählt, wenn sie die erforderliche Mehrheit erlangt und die Wahl angenommen hat.



\section{Protokollführung}

\Abs \Satz Die Protokolle der StuRa-Sitzungen werden durch den Sitzungsvorstand angefertigt und veröffentlicht.

\Abs \Satz Das Protokoll orientiert sich am Sitzungsverlauf.

\Abs \Satz Das Protokoll hat insbesondere zu enthalten:
\begin{enumerate}
\item Datum, Beginn und Ende der Sitzung,
\item die Anwesenheitsliste mit den entsprechenden Vermerken "`unentschuldigt"', "`entschuldigt"' bzw. "`ruht"' bei den fehlenden Mitgliedern,
\item den Wortlaut der Anträge und Beschlüsse gegebenenfalls nebst zugehöriger Abstimmungsergebnisse,
\item die wesentlichen Meinungen für und wieder den Antrag sowie
\item Wortmeldungen, die zuvor ausdrücklich zu Protokoll gegeben wurden.
\end{enumerate}

\Abs \Satz Personaldebatten werden nicht protokolliert.

\Abs \Satz Das Protokoll ist nach der Genehmigung durch den StuRa von der Protokollführerin und von der Versammlungsleiterin zu unterzeichnen und unverzüglich der Öffentlichkeit zugänglich zu machen.

\Abs \Satz Waren Teile der Sitzung nicht öffentlich, so sind die Protokollteile darüber nur den Mitgliedern des StuRa zugänglich.

\Abs \Satz Das Protokoll muss spätestens eine Woche nach der Sitzung den Mitgliedern zugestellt werden.



\section{Rechenschaftsberichte}

\Abs \Satz Die Rechenschaftsberichte im Sinne dieses Paragraphen sind vierteljährlich zu erstellen, dem StuRa vorzulegen und auf den nach § 21 (4) der Grundordnung festgelegten Sitzungen mündlich zu erläutern. Diese sind:
\begin{enumerate}
\item Übersicht über die Einnahmen und Ausgaben eines Monats sowie die Auslastung der Haushaltstitel,
\item kurzer Rechenschaftsbericht über die Arbeit jedes Referats,
\item kurzer politischer Bericht, der insbesondere Bezug nimmt auf die Umsetzung der Beschlüsse und des Arbeitsprogramms des StuRa.
\end{enumerate}

\Abs \Satz Die Berichte nach Abs.~1,~Nr.~1 sind von der Geschäftsführerin Finanzen, nach Abs.~1,~Nr.~2 von der jeweiligen Geschäftsführerin, nach Abs.~1,~Nr.~3 von der Geschäftsführung zu erstellen\. Die Berichterstattung nach Abs.~1,~Nr.~1 hat schriftlich zu erfolgen, sie müssen auch der Geschäftsführung vorgelegt werden\. Der Bericht nach Abs.~1,~Nr.~1 enthält insbesondere eine Übersicht aller gezahlten Aufwandsentschädigungen.



\section{Geschäftsführung}

\Abs \Satz Die Geschäftsführung tritt wöchentlich zusammen.

\Abs \Satz Die Geschäftsführung ist beschlussfähig, wenn die Mehrheit der Geschäftsführer anwesend ist\. Sie fasst Beschlüsse mit einfacher Mehrheit.

\Abs \Satz Die Sitzung der Geschäftsführung ist öffentlich\. Auf Beschluss der Geschäftsführung kann die Sitzung geschlossen werden\. Einzelne Gäste können zugelassen werden.

\Abs \Satz Es wird ein Protokoll geführt,dabei ist die GO § 18 (3) einzuhalten\.  Das Protokoll ist den StuRa-Mitgliedern zugänglich zu machen\. Es gelten die Fristen nach §~5\. Die Protokolle sind zu veröffentlichen.

\end{multicols}

\nopagebreak
\vspace{1cm}
Inkraftgetreten am 04.~Mai~2001.
\\


\footnotesize
Geändert am 04.~Juli~2003\\
§~16 Abs.~4: einfügen von: "` ;§~12 Abs.~2 Grundordnung findet dabei keine Anwendung."'

Geändert am 10.~August~2006\\
§~2 : in Abs. 1 einfügen von: "`bis 23.00 Uhr"'; NEU Abs. 3\\
§~5 : NEU\\
§~6,alt §~5 : in Abs. 2 streichen von "`Anträge"', in Abs. 3 NEU Satz 1; NEU Abs. 4\\
§~7,alt §§~17,19 und 20 : Zusammenfassen sämtlicher Kompetenzen des Versammlungsleiters, die nicht in der Satzung geregelt sind \\
§~8,alt §~18 : NEU Abs. 2; einfügen in Abs. 3 von "`nach Ermessen des Versammlungsleiters"'; NEU Abs. 4; alt Abs. 3 wird Abs. 5\\
§~9, alt §~7: Abs. 4: NEU Punkte 4,9 und 16; Änderung von 6, alt 5: "`erneute Auszählung"' statt "`erneute Beschlussfassung wegen objektiver Unklarheit"'; Änderung von 8, alt 7: "`Beratungs-"' statt "`Sitzungspause"'; NEU Abs. 7; Abs. 8, alt 7 Streichung von "`für jede im StuRa vertretene Fachschaft oder die Geschäftsführung von einem jeweiligen Vertreter"'\\
§~10, alt 8 und 9: NEU Abs. 1: Auflistung sämtlicher Anträge; NEU Abs. 2: Definition von Frist und Form der Anträge; NEU Abs. 3: spezielle Regelungen für ordentliche Anträge; NEU Abs. 4: spezielle Regelungen für Initiativanträge; NEU Abs. 5: spezielle Regelungen für AE-Anträge; NEU Abs. 6: spezielle Regelungen zu Änderungsanträgen; NEU Abs. 7: Regelungen zu Rücknahme von Anträgen\\
§~12, alt 10 und 11: Abs. 3: Pflicht zur Auszählung bei erhöhten Mehrheiten; Abs. 6: einfügen von "`,außer wenn keine einzige Ja-Stimme abgegeben wurde."'; Abs. 8 war vorher §~11\\
§~13: NEU Abs. 2\\
§~14: NEU\\
§~15: NEU\\
§~16: Abs. 3-5 (alt) gestrichen; alter Abs. 6 wird 3; alter Abs. 7 gestrichen\\
§~17: NEU Abs. 2; Aufteilung von alt Abs. 1 in neue Abs. 1 und 3\\
§~18, alt §~16: NEU Abs. 1; streichen von alt Abs. 5 weitere Kandidaturen zwischen Wahlgängen;\\
§~19, alt 21: Abs. 1 Änderung von "`durch den bestellten Protokollführer"' in "`durch die Sitzungsleitung"'; NEU Abs. 2, in Abs. 3 (alt 2) Streichung von "`die Schwerpunkte der Debatten"' und Ersetzung durch Punkt 4;\\
§~20, alt §~22: streichen in Satz 1 von "`nicht"'; NEU Abs. 4 und 5\\
Geändert am 23.\,11.\,06\\
§~5~(3): Einfügung von Satz 2.\\
§~5~(4): Streichen von "`bis zur 2. Sitzung"' und ersetzen durch "`bis zum 7. des Folgemonats"'

Geändert am 17.~Juli~2008\\
In der gesamten Ordnung "`Sitzungsleitung"' durch "`sitzungsvorstand"' ersetzt;\\
§ 5 Abs. 1 "`dem Bericht der Geschäftsführung"' in "`den Berichten nach § 19"' und "` dem Protokoll der Sitzung der Geschäftsführung"' in "`den Beschlüssen der Geschäftsführung und der Ausschüsse"' geändert;\\
§ 5 Abs. 2 "`72 Stunden"' in "`drei Tage"' geändert;\\
§ 5 Abs. 3 "`oder Ausschuss-Beschlusses"' eingefügt;\\
alt § 5 Abs. 4 gestrichen;\\
§ 6 Abs. 1 "`der Geschäftsführung"' in "`des Sitzungsvorstandes"' geändert;\\
§ 6 Abs. 2 Nr. 2 geändert in "`Bericht der Geschäftsführung und Debatee des Berichts"';\\
§ 7 Abs. 2 "`, dies erfolgt in der Regel nach circa eineinhalb Stunden"' gestrichen;\\
§ 8 Abs. 3 Nr. 1 NEU;\\
§ 9 Abs. 4 umnummeriert, Nr. 10 NEU, Nummerierung in nachfolgenden Absätzen entsprechend geändert;\\
§ 9 Abs. 8 NEU;\\
§ 10 Abs. 1 alt Nr. 3 gestrichen;\\
§ 10 Abs. 3 "`bei der Geschäftsführung"' durch "`beim Sitzungsvorstand"' ersetzt, "`die vom StuRa behandelt werden"' eingefügt;\\
§ 10 Abs. 4 alt S. 3 gestrichen;\\
§ 12 Abs. 2 S. 2 "`5"' durch "`9"' ersetzt;\\
§ 13 Abs. 14 S. 1 "`, außer in der vorlesungsfreien Zeit"' eingefügt;\\
alt § 16 gestrichen;\\
§ 14 Abs. 2 S. 2 NEU;\\
§ 16, alt § 17 Abs. 1 "`und Referate auf Grundlage der Struktur"' gestrichen;\\
§ 16, alt § 17 Abs. 2 Verweis korrigiert;\\
§ 16, alt § 17 Abs. 3 S. 2 NEU;\\
§ 17, alt § 18 Abs. 1 "`bei der Geschäftsführung"' durch "`beim Sitzungsvorstand"' ersetzt;\\
§ 17, alt § 18 Abs. 5 Verweis korrigiert;\\
§ 18, alt § 19 Abs. 7 vollständig neugefasst;\\
§ 19 NEU;\\
§ 20 Abs. 2 alt S. 3 gestrichen;\\
§ 20 Abs. 4 S. 4 NEU;\\
§ 20 alt Abs. 5 gestrichen;\\

Geändert am 16.~Juli~2010\\
§~18~Abs.~1 "`und veröffentlicht"' hinzugefasst;\\
§~18~Abs.~2~Satz~1 "`Das Protokoll wird ergebnisorientiert geführt\."' gestrichen;\\
§~18~Abs.~3 Punkt 4 eingefügt;\\
§~20~Abs.~4~Satz~1 "`dabei ist die GO § 18 (3) einzuhalten\."'eingefügt;\\

Geändert am 13.~August~2010\\
§~9~Abs.~12 hinzugefügt;\\
§~17~Abs.3 Satz 2 eingefügt;\\
§~19~Berichte in Rechenschaftsberichte geändert und von monatlich auf vierteljährlich geändert, "`und auf den nach § 21 (4) GrO festgelegten Sitzungen mündlich zu erläutern"' hinzugefügt;\\
§~5~Abs.1~Punkt 4 dementsprechen in Rechenschaftsberichte geändert;\\
§~21~entfernt, dafür in der Satzung §~4~a hinzugefügt;\\
§~10~Abs.~1~Punkt 4 hinzugefügt;\\
§~10~Abs.~2a hinzugefügt;\\
§~10~Abs.~5 Satz 4 hinzugefügt;\\
§~10~Abs.~6 neu;\\
§~5~Abs.~3~Satz 2 "`Initiativanträge zur Aufhebung eines Gf- oder Ausschuss-Beschlusses sind auf der Sitzung, auf der dieser Beschluss bekannt gegeben wird, davon ausgenommen."' gestrichen;\\
§~5~Abs.~1 "`und der Ausschüsse"' hinzugefügt;\\

Geändert am 24. Mai 2012\\
§ 10 Abs. 2 Satz 4 hinzugefügt\\

\normalsize
%~\\*[4cm]
%\begin{center}
%\hspace*{\fill}
%\parbox{7cm}{Christoph Lüdecke\\GF Soziales}
%\hfill\parbox{7cm}{Alexander Kasten\\GF Öffentliches}
%\hspace*{\fill}
%\end{center}