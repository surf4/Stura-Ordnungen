\markright{Durchführungsbestimmungen Geschäftsordnung}
\setcounter{section}{0} % ist nötig um den Paragrafenzähler zurücksetzen
\begin{multicols}{2}


\section{Durchführungsbestimmung zum Tagesordnungspunkt "`Debatte des Berichts der Geschäftsführung"' gemäß §~6~(2) der Geschäftsordnung}

Für den Bericht der Geschäftsführung und die Debatte des Berichts auf den StuRa-Sitzungen gelten folgende Bestimmungen:

\setcounter{absatz}{0}

\Abs \Satz Der Bericht der Geschäftsführung (Gf) soll ein gemeinsamer Bericht der Gf über alle Geschäftsbereiche sein.

\Abs \Satz "Debatte des Berichts” ist großzügig auszulegen: \Satz Nicht nur Themen, die im Bericht erwähnt werden, sondern auch Nachfragen und spezifische Kritik an einzelnen Geschäftsführerinnen (Referentinnen, Arbeitsgemeinschaften, Referatsmitgliedern etc.) bzw. dem Verhalten der Geschäftsführung während des Berichtszeitraums können in diesem TOP diskutiert werden.

\Abs \Satz Anfragen, die während dieses TOPs an die Gf gestellt werden, sind zu protokollieren und von der Gf möglichst sofort, spätestens jedoch innerhalb der Frist aus §~21 der Geschäftsordnung zu beantworten.

\Abs \Satz Für grundsätzlichen Diskussionsbedarf über Abläufe, Regelungen o.~ä. im StuRa sind jedoch eigene TOPs einzurichten, die nach Möglichkeit mit einer Beschlussvorlage versehen sind.

\Abs \Satz Für eine Kritik an Geschäftsführerinnen, Referentinnen, Referatsmitgliedern, Arbeitsgemeinschaften oder Angestellten des StuRa, die sehr umfangreich oder sehr grundsätzlich ist oder deren öffentliche Diskussion die Persönlichkeitsrechte der Betroffenen verletzen könnte, ist eine Personaldebatte vorzusehen.

\end{multicols}

\nopagebreak
\vspace{1cm}
Inkraftgetreten am 12.~Oktober~2006.
\\


\footnotesize
Geändert am 17.~Juli~2008\\
alt~§~1~Abs.~5~S.~2 gestrichen.


\normalsize
~\\*[4cm]
\begin{center}
\hspace*{\fill}
\parbox{7cm}{Christoph Lüdecke\\GF Soziales}
\hfill\parbox{7cm}{Alexander Kasten\\GF Öffentliches}
\hspace*{\fill}
\end{center}