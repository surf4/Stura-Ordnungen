% Normaldruck
\documentclass[10pt, a4paper, smallheadings, halfparskip, cleardoubleplain, pagesize, notitlepage, oneside, tocleft]{scrartcl}
%
% Entwurfsdruck (ohne Bilder usw.)
%\documentclass[10pt, a4paper, smallheadings, halfparskip, cleardoubleplain, pagesize, notitlepage, twoside, tocleft, draft]{scrartcl}

\usepackage[ngerman]{babel}
\usepackage[utf8]{inputenc}
\usepackage[T1]{fontenc}
\usepackage{graphicx}
\usepackage[bindingoffset=0.5cm,inner=1.5cm,outer=1.5cm,top=1cm,bottom=1cm,includeheadfoot]{geometry}
\usepackage{units}     %Brüche
\usepackage{multicol}  %Zweispaltigen Satz

\addtokomafont{section}{\centering}  %Format für Paragrafen
\addtokomafont{subsection}{\centering}  %Format für Absätze
\renewcommand*{\raggedsection}{\centering} %Mittige Kapitelüberschriften

\newcounter{absatz}[section] %Zähler für Absätze
\newcounter{sentence}[absatz] %Zähler für Sätze

\renewcommand*{\thesection}{\S\ensuremath{\,}\arabic{section}} %Ausgabeformat für Paragraphen
\renewcommand*{\theabsatz}{(\arabic{absatz})}   %Ausgabeformat für Absatzzähler

%\setcounter{secnumdepth}{5} %Bis zur Ebene Paragraph wird ein Zähler ausgegeben
%\setcounter{tocdepth}{1} %Nur Kapitelüberschriften im Inhaltsverzeichnis

%Absatzzähler
\newcommand*{\Abs}{\stepcounter{absatz}\theabsatz}

%Satzzähler
\renewcommand{\.}{. \Satz} %Definiert \. um auf \Satz, damit dieses Zeichen ebenfalls für eine Satzzählung sorgt.
\newcommand{\Satz}[1][\thesentence]{\stepcounter{sentence}\setcounter{sentence}{#1}\textsuperscript{\thesentence}}
\newcommand{\s}{\Satz}  %Michas altes Kommando fr Satzzhler; ich war zu faul, alles zu ersetzen; betrifft: S, GO, AEO

\setlength{\parindent}{0pt}
\setlength{\columnsep}{1cm}
\setlength{\columnseprule}{0.5pt}
%\pagestyle{myheadings}
%\markboth{Studentenrat der TU Dresden}{}
\hyphenation{Ta-ges-ord-nungs-punkte}

\begin{document}
%Kopfzeile Titelseite
 \titlehead{
    \begin{tabular}[b]{c}
    \includegraphics[height=2.1cm,keepaspectratio=true]{bilder/Logo_Graustufen.jpg}\\
    \end{tabular}
    \hfill
    \includegraphics[height=1.7cm,keepaspectratio=true]{bilder/TU_Logo_SW.pdf}
    \vskip5pt \hrule
    }

%Überschriftenzeile auf Titelseite --> in \title muss der jeweilige Titel eingefügt werden
 \vspace{3cm}
 \title{Wahlordnung\\der Studentenschaft der TU Dresden}
 \author{\normalsize{Erstellt am \today.}}
 \date{\ }
%Titel ausgeben
 \maketitle
 
%TOC (Liste aller Paragraphen zweispaltig auf Titelseite ausgeben)
%\begin{multicols}{2}
\tableofcontents
%\end{multicols}

%jeweiliges Hauptdokument einbinden
\normalsize
%%\addchap[AE-Ordnung]{AE-Ordnung\\der Studentenschaft der TU Dresden zu Finanzordnung § 40 Abs. 2}
\markright{AE-Ordnung}
\setcounter{section}{0}
\begin{multicols}{2}


\section{Allgemeines}

\Abs \Satz Gemäß §40 der Finanzordnung werden im Folgenden die Grundzüge der Art und Weise der Zahlungen von Aufwandsentschädigungen (AE) geregelt.

\Abs \Satz Als Anspruchszeitraum gilt genau ein Kalendermonat. Für die Sportobleute gilt als Anspruchszeitraum ein Semester.


\section{AE-Berechtigte}

\Abs \Satz AEs können beantragt werden durch
\begin{enumerate}
\item Referatsmitarbeiterinnen,
\item Referentinnen,
\item Geschäftsführerinnen,
\item Sportobleute,
\item Mitarbeiterinnen von Projekten des StuRa,
\item Ausschussmitarbeiterinnen, falls dies bei der Einrichtung des Ausschusses so geregelt wurde,
\item Mitglieder des Sitzungsvorstandes.
\end{enumerate}


\section{AE-Beantragung}

\Abs \Satz Anträge auf Aufwandsentschädigung müssen spätestens am 10. Tag nach dem Ende des Anspruchszeitraums gestellt werden.

\Abs \Satz Anträge auf Aufwandsentschädigung müssen begründet werden.

\Abs \Satz Die beantragten Aufwandsentschädigungen sind so aufzuschlüsseln, dass sie den jeweiligen Sachkonten des Wirtschaftsplanes zugeordnet werden können.

\section{Festlegung der AE Höhe}

\Abs \Satz Für die nach §2 (2) definierten Ämter können von Referatsmitarbeiterinnen 70 Euro, von Referentinnen 125 Euro und von Geschäftsführerinnen 210 Euro als AE beantragt werden.

\Abs \Satz Bei unvorhergesehenen und außerordentlichen Aufgaben oder Mitarbeit an Projekten kann über die in (1) genannte Summe bis zu 350 Euro beantragt werden.

\Abs \Satz  Die studentischen Sportobleute des Universitätssportzentrums der TU Dresden können eine AE in Höhe von maximal 200 Euro pro Person und Semester erhalten\. Mitglieder des Sitzungsvorstandes werden wie Referentinnen behandelt.

\Abs \Satz Die Höhe der Aufwandsentschädigung, die vom StuRa gezahlt wird, ist auf 350 Euro pro Person und Monat begrenzt. 

\section{Beschlussfassung über AE Anträge}

\Abs \Satz Die Beschlussfassung über Aufwandsentschädigungen wird in nichtöffentlicher Sitzung befunden.

\Abs \Satz Die Anträge auf Aufwandsentschädigung sowie deren Begründungen müssen allen StuRa- Mitgliedern zugänglich gemacht werden. Näheres wird in der Durchführungsbestimmung geregelt.

\Abs \Satz Die Aufwandsentschädigungen der Geschäftsführerinnen werden vom StuRa-Plenum beschlossen.

\Abs \Satz Sonstige Aufwandsentschädigungen werden von der Geschäftsführung beschlossen.


\section{Sonstige und Schlussbestimmungen}

\Abs \Satz Diese Ordnung gilt ab dem nächsten Anspruchszeitraum (§1, Absatz 2) nach Erlass.

%Ende für multicols
\end{multicols}

\nopagebreak
\vspace{1cm}

\footnotesize
Inkraftgetreten am 30. August 2012.\\


%\normalsize
%~\\*[4cm]
%\begin{center}
%\hspace*{\fill}
%\parbox{7cm}{Matthias Zagermann\\GF Inneres}
%\hfill\parbox{7cm}{Ulrich Gebler \\GF Lehre und Studium}
%\hspace*{\fill}
%\end{center}
%%Header für spaltenlose Version
%\addchap[Beitragsordnung der Studentenschaft der TU Dresden]{Beitragsordnung\\der %Studentenschaft der TU Dresden}
%\markright{Beitragsordnung}
%\setcounter{section}{0}


%Header für multicols
\markright{Beitragsordnung}
\setcounter{section}{0}
\begin{multicols}{2}
 

\section{Beitragszweck}

\Abs \Satz  Die Studentenschaft der TU Dresden erhebt zur Durchführung ihrer Aufgaben von ihren Mitgliedern Beiträge [§~2~Abs.~2 Grundordnung der Studentenschaft der TU Dresden]. 

\section{Beitragshöhe}

\Abs \Satz  Der Beitrag ist in folgender Höhe für folgende Zwecke bestimmt:
\begin{enumerate}
\item Für den StuRa 3,70~Euro
\item Für die Fachschaften 0,90~Euro
\item Für das Studentenjahresticket 332,40~Euro pro Studienjahr (Wintersemester und nachfolgendes Sommersemester)
\end{enumerate}

\Abs \Satz  Studentinnen, die erstmals im Sommersemester immatrikuliert werden, zahlen für den verbleibenden Gültigkeitszeitraum nur den halben Beitrag des Studentenjahrestickets. 


\section{Beitragspflicht}

\Abs \Satz  Der Beitragspflicht unterliegen alle Studentinnen, die Mitglied der Studentenschaft der TU Dresden sind.

\Abs \Satz  Fernstudentinnen, Studentinnen, die an Außenstellen oder Tochtereinrichtungen der TU Dresden außerhalb Sachsens immatrikuliert sind und dort studieren, Studentinnen, die der Fachschaft „Studierendenschaft IHI Zittau“ zugeordnet sind sowie Studentinnen, die vom Studium beurlaubt sind, sind, sofern sie den Antrag auf Beurlaubung bis zum Ende der Rückmeldefrist gemäß §~6~Abs.~1 Immatrikulationsordnung gestellt haben, während dieser Zeiten von der Zahlungspflicht für das Studentenjahresticket befreit. 


\section{Rückerstattung und Nachkauf}

\Abs \Satz  Der Studentenschaftsbeitrag kann in sozialen Härtefällen aus Mitteln des Studentenrates zurückerstattet werden\. Näheres regelt die Härtefallordnung. 

\Abs \Satz  In folgenden Fällen können Studentinnen auf schriftlichen Antrag an den Studentenrat den Beitragsanteil für das Studentenjahresticket zurück erhalten:
\begin{enumerate}
\item behinderte Studentinnen im Besitz eines Schwerbehindertenausweises mit einem der gültigen Merkzeichen (gem. SGB~IX)
\renewcommand{\labelitemi}{—}
\begin{itemize}
\item aG,
\item Bl,
\item H,
\item G mit gültiger Wertmarke,
\item Gl mit gültiger Wertmarke
\end{itemize}
oder mit anderweitig nachgewiesener Behinderung, die die Nutzung des Studentenjahrestickets verhindert,
\item Ableistung eines Praktikums oder einer sonstigen studienbedingten Anstellung außerhalb des räumlichen Gültigkeitsbereiches des Studentenjahrestickets,
\item Erstellung einer Diplomarbeit bzw. sonstiger Abschlussarbeit studienbedingt außerhalb des räumlichen Gültigkeitsbereiches des Studienjahrestickets,
\item Rücktritt vom Studienplatz,
\item nachträgliche Beurlaubung,
\item Promotion außerhalb des räumlichen Gültigkeitsbereiches des Studentenjahrestickets,
\item studienbedingter Auslandsaufenthalt ohne Beurlaubung, 
\item Im- oder Exmatrikulation.
\end{enumerate}

\Abs \Satz  Der Antrag auf Rückerstattung muss spätestens 14~Tage nach Eintreten des Rückerstattungsgrundes beim Studentenrat eingehen, andernfalls kann nur zeitanteilig erstattet werden\. Dabei ist eine nachträgliche Erstattung ausgeschlossen. 

\Abs \Satz  Als Eingangszeitpunkt eines Antrags auf Erstattung des anteiligen Beitrags für das Studentenjahresticket gilt der Zeitpunkt, zu dem dieser Antrag und der Studentenausweis dem Studentenrat vorliegen\. Die schriftlichen Unterlagen zum Nachweis der Voraussetzungen für eine Beitragserstattung gemäß §~4~Abs.~2 können binnen sechs Wochen nachgereicht werden. 

\Abs \Satz  Falls die Voraussetzungen für eine Erstattung des anteiligen Beitrags für das Studentenjahresticket nicht während eines gesamten Studienjahres vorliegen, wird der Beitragsanteil zeitanteilig erstattet\. Dabei wird für jeden angefangenen Monat Restgültigkeit ein Zwölftel des Beitragsanteils für das Studentenjahresticket abgezogen\. Außer im Fall der Ex- oder Immatrikulation erfolgt keine Rückerstattung von weniger als einem Zwölftel des Jahresbeitrags. 

\Abs \Satz  Anträge nach Abs.~2~Nr.~1~bis~7, die nach dem 28.2. für das laufende Studienjahr eintreffen, sind abzulehnen\. Bei Verlust des Studentenausweises erfolgt keine Rückerstattung. 

\Abs \Satz  Die Möglichkeit, das Studentenjahresticket nachträglich zu erwerben, haben alle Studentinnen mit Ausnahme der Fernstudentinnen, die nach §~3 von der Beitragspflicht des Semestertickets befreit sind\. Studentinnen, die nach §~3~Abs.~1 vom gesamten Studentenschaftbeitrag befreit sind, haben diesen beim Nachkauf des Studentenjahrestickets ebenfalls zeitanteilig nachzuentrichten\. Der Preis für das Studentenjahresticket im Nachkauf beträgt für jeden angefangenen Monat Restgültigkeit ein Zwölftel des Beitragsanteils für das Studentenjahresticket, mindestens jedoch ein Sechstel von diesem.

\Abs \Satz  Studentinnen, die, ohne die Voraussetzungen einer Beitragserstattung gemäß Absatz~2 zu erfüllen, im Laufe eines Studienjahres aus der Studentenschaft der TU Dresden austreten, bleiben bis zum Ende des betreffenden Studienjahres Inhaber des Studentenjahrestickets und verpflichtet, den Beitragsanteil für das Studentenjahresticket zu zahlen.


\section{Beitragserhebung und Fälligkeit}

\Abs \Satz  Der Semesterbeitrag ist in der vom Immatrikulationsamt bekannt gemachten Form einzuzahlen\. Er wird fällig mit der Einschreibung bzw. Rückmeldung.

\Abs \Satz  Der Beitragsanteil für das Studentenjahresticket ist wahlweise mit der Rückmeldung zum Wintersemester eines Studienjahres oder in zwei gleichen Raten zu je 166,20~Euro mit der Rückmeldung zum Wintersemester und zum darauffolgenden Sommersemester einzuzahlen. 

\section{Mittelverwaltung}

\Abs \Satz  Der StuRa zahlt aus der Summe der für ihn gemäß §~2~Abs.~1 bestimmten Mittel jeder Fachschaft einen Sockelbetrag in Höhe von 500,00~Euro.

\Abs \Satz  Der StuRa verwaltet die für ihn bestimmten Mittel entsprechend seiner Finanzordnung\. Die Fachschaften verwalten die ihnen übergebenen Mittel in eigener Verantwortung gemäß der Finanzordnung.

\Abs \Satz  Die Beiträge für das Studentenjahresticket des VVO werden durch das Immatrikulationsamt gemäß der mit diesen Unternehmen getroffenen Vereinbarung direkt überwiesen.

\Abs \Satz  Die Regelungen der §§~3~Abs.~4~und~8~Abs.~2~S.~2 der Finanzordnung bleiben unberührt.

%Ende für multicols
\end{multicols}

\nopagebreak
\vspace{1cm}
Inkraftgetreten am 01.~Juli~2013.
\\


\footnotesize
komplett neu gefasst am 27.~Juni~2013\\

\normalsize
~\\*[4cm]
\begin{center}
\hspace*{\fill}
\parbox{7cm}{Matthias Funke\\GF Finanzen}
\hfill\parbox{7cm}{Andreas Spranger\\GF Hochschulpolitik}
\hspace*{\fill}
\end{center}


%\include{Finanzordnung/finanzordnung-col}
%%\addchap[Richtlinie über die finanzielle Förderung studentischer Projekte der Studentenschaft der TU Dresden]{Richtlinie über die finanzielle Förderung studentischer Projekte\\der Studentenschaft der TU Dresden}
\markright{Richtlinie über die finanzielle Förderung studentischer Projekte}
\setcounter{section}{0} % ist nötig um den Paragrafenzähler zurücksetzen
\setcounter{sentence}{0} % ist nötig um den Satzzähler zurückzusetzen
\begin{multicols}{2}

\section{Förderausschuss}

\Abs \Satz Der Förderausschuss ist ein Ausschuss gemäß §~24 der Grundordnung\. Er besteht aus vier vom
StuRa gewählten StuRa-Mitgliedern und der Geschäftsführerin Finanzen.
\\ 

\Abs \Satz Der Förderausschuss entscheidet über die finanzielle Förderung studentischer Projekte laut §~33 der Finanzordnung und die Anerkennung von Hochschulgruppen gemäß Richtlinie zur Anerkennung von Hochschulgruppen.

\section{Haushaltsvorbehalt und Rechtsanspruch}
\Abs \Satz Eine Förderung erfolgt unter dem Vorbehalt verfügbarer Mittel im zugeordneten
Haushaltstitel.\\

\Abs \Satz Die Höhe der Förderung muss in Relation zur Gesamthöhe des Budgets liegen.\\

\Abs \Satz Auf eine Förderung besteht kein Rechtsanspruch.

\section{Grundsätzliches}
\Abs \Satz Projekte die gegen grundsätzliche Positionen des StuRa laufen werden nicht gefördert.\\

\Abs \Satz Der StuRa muss in Publikationen zum geförderten Projekt als Förderer genannt werden.\\

\Abs \Satz Kosten für Verpflegung werden nicht übernommen.\\

\Abs \Satz Materialien für den dauerhaften Gebrauch bleiben Eigentum der Studentenschaft
und werden nur als Dauerleihgaben vergeben.\\

\Abs \Satz Über dauerhafte Förderung über ein Wirtschaftsjahr hinaus entscheidet der StuRa gemäß
§~35 der Finanzordnung\. Der Förderausschuss gibt hierfür eine Empfehlung ab.\\

\Abs \Satz Genehmigte und nichtabgerufene Förderanträge verfallen 4 Monate nach Bewilligung.\\

\Abs \Satz Für die Abrechnung eines Förderantrages müssen alle tatsächlich angefallenen Einnahmen
und Ausgaben belegt werden.\\


\section{Öffentlichkeit}
\Abs \Satz Veranstaltungen und Exkursionen werden nur gefördert, wenn diese ausreichend beworben
werden und die Teilnahme grundsätzlich allen Studentinnen möglich ist.\\

\Abs \Satz Für Veranstaltungen und Exkursionen kann eine Eigenbeteiligung der Teilnehmerinnen
vorgesehen werden\. Die Höhe der Eigenbeteiligung darf nicht sozial Selektiv wirken.\\

\Abs \Satz Vom StuRa geförderte Veranstaltungen müssen barrierefrei sein\. Ist die Barrierefreiheit
nicht möglich, muss dies kurz und schriftlich erklärt werden.

\section{Sport}
\Abs \Satz Der StuRa fördert den freiwilligen Studierendensport finanziell\. Dazu gehören insbesondere
die Übernahme der Kosten von Sachpreisen und Mieten bei Turnieren, von Fahrtkosten zu Wettbewerben und von Werbungskosten für Veranstaltungen.\\

\Abs \Satz Der Wirtschaftsplan sieht einen eigenen Titel für Sportförderung vor.

\section{Lehrveranstaltungen und Exkursionen}
\Abs \Satz Kosten für Seminare, Ringvorlesungen und Exkursionen für die es Leistungsnachweise gibt
oder die zum Studienablauf gehören, werden nur übernommen wenn sie hauptsächlich der
Erfüllung der Aufgaben der Studentenschaft laut SächsHSG dienen.

\section{Partys}
\Abs \Satz Der StuRa fördert keine Partys großer Dimension.\\

\Abs \Satz Partys werden nur in Form von Ausfallbürgschaften gefördert\. Der vom StuRa gedeckte
Anteil beträgt höchstens die Hälfte des gesamten Fehlbetrags, maximal jedoch 500 Euro.\\

\Abs \Satz Stehen der Veranstalterin mehrere Bürgen zur Finanzierung des Fehlbetrages zur
Verfügung, übernimmt der StuRa nur einen der Anzahl der Bürgen entsprechenden Anteil am
Fehlbetrag.

\section{Förderung der Fachschaften}
\Abs \Satz Projekte einer Fachschaft werden nur gefördert wenn deren Rücklage (über 1500 Euro) das
Dreifache der Semestereinnahmen nicht übersteigt.\\

\Abs \Satz Der StuRa zahlt nicht mehr als der jeweilige FSR, sofern der FSR über weniger
als 100 Euro Guthaben verfügt.\\

\Abs \Satz Büroausstattung und Rechentechnik muss durch den FSR eigenständig finanziert werden.\\

\Abs \Satz Der Wirtschaftsplan sieht einen eigenen Titel für die Förderung der Fachschaften vor.\\

\Abs \Satz Bei Veranstaltungen von mehr als einem FSR gilt Abs. (1) nicht.
\end{multicols}

\nopagebreak
\vspace{1cm}
Inkraftgetreten am 16.~Februar~2009.
\\

\normalsize
~\\*[4cm]
\begin{center}
\hspace*{\fill}
\parbox{7cm}{Armin Grundig\\GF Soziales}
\hfill\parbox{7cm}{Enrico Lovasz\\GF Finanzen}
\hspace*{\fill}
\end{center}

%%Header Geschäftsordnung multicol
\markright{Geschäftsordnung}
\setcounter{section}{0} % ist nötig um den Paragrafenzähler zurücksetzen

\begin{multicols}{2}

\section{Konstituierung}

\Abs \Satz Die konstituierende Sitzung findet in der zweiten Woche nach Bekanntgabe der Ergebnisse der Wahlen der FSR statt.



\section{Zusammentreten}

\Abs \Satz Der Stura tagt donnerstags von 19.30~Uhr~bis~23.00 Uhr\. Einer gesonderten Einladung bedarf es nicht.

\Abs \Satz In der Woche nach der Wahl der FSR findet keine Sitzung statt.

\Abs \Satz Als Einladung für Sondersitzungen nach §~22 der Grundordnung gilt die fristgemäße Versendung einer E-Mail an das StuRa-Mitglied\. Auf Wunsch eines StuRa-Mitgliedes kann ihm die Einladung auch per Telefon, Fax oder auf dem Postweg (als fristwahrend gilt hier der Poststempel) zugestellt werden.



\section{Öffentlichkeit}

\Abs \Satz Die Sitzungen des StuRa sind grundsätzlich öffentlich\. Alle Anwesenden haben das Rederecht.

\Abs \Satz Angelegenheiten, die die Persönlichkeitsphäre oder die Angestellten des StuRa betreffen, sind in nicht-öffentlicher Sitzung zu behandeln.

\Abs \Satz Für den nicht-öffentlichen Teil sind die Anwesenden zur Verschwiegenheit verpflichtet.



\section{Beschlussfähigkeit}

\Abs \Satz Nach Eröffnung der Sitzung sind die Anwesenheit der Mitglieder und die Beschlussfähigkeit festzustellen.



\section{Sitzungsvorlagen und Fristen}

\Abs \Satz Die Sitzungsvorlagen an die StuRa-Mitglieder bestehen aus:
\begin{itemize}
\item zu behandelnden ordentlichen Anträgen nach §~10,
\item Kandidaturen,
\item dem Vorschlag zur Tagesordnung,
\item den Rechenschaftsberichten nach §~19,
\item den Beschlüssen der Geschäftsführung und der Ausschüsse,
\item dem Protokoll der Sitzungen der Geschäftsführung und der Ausschüsse,
\item aus unbestätigten Protokollen,
\item aus weiteren Vorlagen zu einzelnen Tagesordnungspunkten.
\end{itemize}

\Abs \Satz Die Sitzungsvorlagen müssen den Mitgliedern drei Tage vor Beginn der Sitzung zugänglich gemacht werden.

\Abs \Satz Initiativanträge müssen vor Sitzungsbeginn eingereicht werden.



\section{Tagesordnung}

\Abs \Satz Zu Beginn der Sitzung ist der Tagesordnungsvorschlag des Sitzungsvorstands vorzustellen und über Änderungsanträge zu beschließen\. Danach ist die Tagesordnung zu verabschieden.

\Abs \Satz Die Tagesordnung muss ein Verzeichnis aller vorliegenden Anträge, sowie deren Zuordnung zu Tagesordnungspunkten enthalten\. Sie muss folgende Punkte vorsehen:
\begin{enumerate}
\item die Genehmigung der vorliegenden Protokolle,
\item Bericht der Geschäftsführung und Debatte des Berichts,
\item Sonstiges.
\end{enumerate}

\Satz Die Punkte~1~und~2 dürfen nur auf ordentlichen Sitzungen behandelt werden.

\Abs \Satz In der Regel sind für Anträge eigene Tagesordnungspunkte einzurichten\. Tages\-ordnungspunkte, die unter Ausschluss der Öffentlichkeit behandelt werden, sind nach Möglichkeit an das Ende der Sitzung zu legen.

\Abs \Satz Abweichend von Abs.~1 ist auf außerordentlichen Sitzungen der TO-Vorschlag der Antragstellerinnen, so wie er im Beschluss der Sondersitzung enthalten ist, vorzustellen\. Änderungsanträge dürfen nur die Gliederung der außerordentlichen Sitzung betreffen.



\section{Versammlungsleiterin}

\Abs \Satz Die Versammlungsleiterin hat die Kompetenzen aus §~23 der Grundordnung.

\Abs \Satz Die Versammlungsleiterin strukturiert die Sitzung gemäß der Tagesordnung\. Sie kann Pausen nach eigenem Ermessen vorsehen.

\Abs \Satz Die Versammlungsleiterin stellt fest, wann die Behandlung eines Tagesordnungspunktes oder die Durchführung einer Wahl oder Beschlussfassung beginnt und endet.

\Abs \Satz Sie hat das Recht, einen Antrag nach ihrem Ermessen aufzugliedern und entsprechend diskutieren zu lassen.

\Abs \Satz Die Versammlungsleiterin erteilt das Wort\. Sie kann die Redezeit begrenzen, eine Rednerin zur Sache oder zur Form rufen\. Kommt eine Rednerin einer solchen Aufforderung nicht nach, kann die Versammlungsleiterin ihr das Wort entziehen.

\Abs \Satz Bei Diskussionen oder Beschlüssen, die die Versammlungsleiterin selbst betreffen, hat sie die Versammlungsleitung abzugeben.

\Abs \Satz Die Auslegung der Geschäftsordnung obliegt, mit Wirkung für die aktuelle Sitzung, der Versammlungsleiterin, gegebenenfalls nach Beratung des Sitzungsvorstands.



\section{Redeliste}

\Abs \Satz Vor Beginn einer Diskussion bittet die Versammlungsleiterin um Wortmeldungen und bildet eine Redeliste\. Nach dieser erteilt sie das Wort und ergänzt sie während der Debatte.

\Abs \Satz Vor der Debatte eines Antrags erteilt die Versammlungsleiterin der Antragstellerin das Wort\. Nach der Vorstellung des Antrags kann die Geschäftsführung zum Antrag Stellung nehmen.

\Abs \Satz Die Redeliste kann nach Ermessen der Versammlungsleiterin unterbrochen werden:
\begin{enumerate}
\item durch Wortmeldung der Antragstellerin bzw. Berichterstatterin zu diesem Tagesordnungspunkt und
\item durch Wortmeldungen der Geschäftsführung sofern Fragen an sie gerichtet sind.
\end{enumerate}

\Abs \Satz Es gilt das Erstrednerinnenrecht.

\Abs \Satz Eine Sitzungsteilnehmerin darf nur sprechen, wenn ihr die Versammlungsleiterin das Wort erteilt\. Möchte die Versammlungsleiterin selbst zur Sache sprechen, so setzt sie sich an das derzeitige Ende der Redeliste.



\section{Anträge zur Geschäftsordnung}

\Abs \Satz Anträge zur Geschäftsordnung gehen allen anderen Wortmeldungen vor\. Sie können nur von StuRa-Mitgliedern gestellt werden und sind durch das Erheben beider Hände zu kennzeichnen.

\Abs \Satz Ein Redebeitrag, eine Wahl oder Abstimmung darf durch einen Geschäftsordnungsantrag nicht unterbrochen werden.

\Abs \Satz Über Geschäftsordnungsanträge ist sofort zu beschließen.

\Abs \Satz Als Geschäftsordnungsanträge sind folgende Anträge anzusehen:
\begin{enumerate}
\item Änderung der beschlossenen Tagesordnung;
\item Schluss der Debatte, gegebenenfalls sofortige Beschlussfassung;
\item Ausschluss der Öffentlichkeit;
\item Abweichung von einzelnen Punkten der Geschäftsordnung;
\item Verlängerung der Sitzung um eine Stunde;
\item Auszählung, gegebenenfalls erneute Auszählung, der Stimmen;
\item erneute Feststellung der Beschlussfähigkeit;
\item fünfminütige Beratungspause;
\item Geheime Abstimmung;
\item einmaliger sofortige Richtigstellung,
\item Personaldebatte;
\item Schluss der Redeliste;
\item Zulassung Einzelner zur geschlossenen Sitzung;
\item Nichtbefassung eines Antrages;
\item Beschränkung der Redezeit;
\item schriftliche Abstimmung;
\item Vertagung eines Punktes der Tagesordnung.
\end{enumerate}

\Abs \Satz Anträge nach Abs.~4~Nr.~1~-~5 bedürfen einer \nicefrac{2}{3}~Mehrheit der anwesenden Mitglieder.

\Abs \Satz Bei einem Geschäftsordnungsantrag nach Abs.~4~Nr.~6~-~10 ist kein Widerspruch zulässig.

\Abs \Satz Der Geschäftsordnungsantrag nach Abs.~4~Nr.~6 muss unmittelbar nach erfolgter Abstimmung gestellt werden.

\Abs \Satz Geschäftsordnungsanträge Nr.~6~und~7 können auch kombiniert gestellt werden.

\Abs \Satz Beratungspausen können einmal pro Tagesordnungspunkt beantragt werden.

\Abs \Satz Personaldebatten finden unter Ausschluss der Öffentlichkeit und der Betroffenen statt.

\Abs \Satz Vor Schluss der Redeliste ist jedem Mitglied des StuRa Gelegenheit zu geben, sich noch auf diese setzen zu lassen.

\Abs \Satz Vertagungen nach § 9 (4) Satz 1 Nummer 17 können mit Terminen und Bedingungen versehen werden. Geschieht dies nicht, werden sie auf die nächste Sitzung vertagt.



\section{Anträge}

\Abs \Satz Neben den Anträgen nach §~9 sind folgende Anträge an den Studentenrat zulässig:
\begin{enumerate}
\item ordentliche Anträge,
\item Initiativanträge,
\item Änderungsanträge,
\item Antrag auf Neubefassung.
\end{enumerate}


\Abs \Satz Alle Anträge nach Abs.~1 sind schriftlich zu stellen\. Sie enthalten den Namen der Antragstellerin, den Antragstext und gegebenenfalls eine Begründung\. Anträge mit dem Ziel eine Finanzwirksamkeit für den StuRa zu entfalten, müssen zusätzlich eine Finanzaufstellung enthalten\. Anträge auf Einrichtung oder Änderung eines StuRa-Projektes müssen insbesondere die Namen der Projektsprecherin und der Mitarbeiterinnen enthalten.

%Absatznummerierung in 2a sollte besser gemacht werden
(2a) Die Rücknahme von Anträgen durch die Antragstellerin ist jederzeit zulässig.

\Abs \Satz Ordentliche Anträge, die vom StuRa behandelt werden, werden beim Sitzungsvorstand eingereicht\. Für Ordentliche Anträge nach Abs.~1~Nr.~1 gelten die Fristen aus §~5.

\Abs \Satz Der Initiativantrag ist der Form und dem Inhalt nach ein ordentlicher Antrag, der die Fristen für ordentliche Anträge gemäß §~5~Abs.~1~und~2 nicht erfüllt\. Für sie gilt §~5~Abs.~3. Er bedarf der Unterschrift sieben stimmberechtigter Mitglieder.

\Abs \Satz Änderungsanträge sind Anträge zu ordentlichen Anträgen, die diese in ihrer Sache oder Ausgestaltung ändern\.  Änderungsanträge werden beim Sitzungsvorstand eingereicht\. Über sie ist vor dem Hauptantrag zu beschließen\. Soweit der StuRa den Änderungsanträgen zustimmt oder sie von der Hauptantragsstellerin übernommen werden, wird der Hauptantrag in der geänderten Fassung zur Beschlussfassung gestellt\.  Die Antragstellerin des Hauptantrages hat bis zur endgültigen Beschlussfassung das Recht, auch eine geänderte Fassung ihres Antrages zurückzuziehen.

\Abs \Satz Anträge auf Neubefassung dürfen nur in Fällen nach § 20, Abs. 5 GrO und nur im Tagesordnungspunkt "`Bericht der Geschäftsführung und Debatte des Berichts"' gestellt werden\. Für sie gelten nicht die Fristen nach § 5.



\section{Lesungen}

\Abs \Satz Für Änderungen der Grundordnung und deren Ergänzungsordnungen sind drei Lesungen erforderlich\. Für die Aufstellung des Haushaltsplanes sind nur zweite und dritte Lesung erforderlich.

\Abs \Satz In der ersten Lesung wird der Antrag nur dem Grundsatz nach besprochen\.  Änderungsanträge dürfen entgegen §~10 nicht gestellt werden\. Am Ende der ersten Lesung beschließt der StuRa über die Überweisung in die zweite Lesung\. Diese findet im Anschluss statt.

\Abs \Satz In der zweiten Lesung wird der Antrag inhaltlich zur Diskussion gestellt\. Am Ende der zweiten Lesung beschließt der StuRa über die Überweisung in die dritte Lesung\. Diese erfolgt in der nächsten ordentlichen Sitzung.

\Abs \Satz In der dritten Lesung wird der Antrag erneut inhaltlich zur Diskussion gestellt\. Abschließend wird der Antrag verlesen und darüber beschlossen.



\section{Beschlussfassung}

\Abs \Satz Die Versammlungsleiterin eröffnet nach Abschluss der Beratung und Wiederholung der Anträge die Beschlussfassung.

\Abs \Satz Änderungsanträge sowie Redebeiträge sind von diesem Zeitpunkt an nicht mehr zulässig\. Das Recht auf Anträge zur Geschäftsordnung nach §~9~Abs.~4~Nr.~9~und~16 bleibt unberührt.

\Abs \Satz Soweit für einen Beschluss nicht eine einfache Mehrheit erforderlich ist, hat die Versammlungsleiterin vor der Beschlussfassung darauf hinzuweisen und die abgegeben Stimmen auszuzählen.

\Abs \Satz Ein Antrag gilt als beschlossen, wenn ihm nicht auf Nachfrage der Versammlungsleiterin widersprochen wird\. Der Widerspruch muss nicht begründet werden (formale Gegenrede).

\Abs \Satz Bei Widerspruch führt die Versammmlungsleiterin unverzüglich durch Abfrage von Zustimmung, Ablehnung und Stimmenthaltung die Abstimmung durch\. Die Abstimmung erfolgt in der Regel durch Handzeichen.

\Abs \Satz Die Abstimmung wird ohne erneute Aussprache einmal wiederholt, wenn die einfache Mehrheit der abgegebenen Stimmen Enthaltungen sind, außer wenn keine einzige Ja-Stimme abgegeben wurde.

\Abs \Satz Das Stimmrecht darf nur von anwesenden Mitgliedern des StuRa ausgeübt werden.

\Abs \Satz Liegen konkurrierende Anträge vor, so hat die Versammlungsleiterin die Beschlussfassung wie folgt durchzuführen:
\begin{enumerate}
\item Geht ein Antrag weiter als ein anderer, so ist über den weitergehenden zuerst zu beschließen. Wird dieser angenommen, so sind weniger weitgehende Anträge erledigt.
\item Lässt sich ein Weitergehen im Sinne von Nr.~1 nicht feststellen, so bestimmt sich die Reihenfolge, in der konkurrierende Anträge gestellt werden, nach der Reihenfolge der Antragstellung.
\end{enumerate}



\section{Schriftliche Abstimmungen}

\Abs \Satz Schriftliche Abstimmungen erfolgen mittels zugängiger Abstimmungsliste.

\Abs \Satz Die Abstimmungsliste enthält die zu Beginn der Abstimmung stimmberechtigten Mitglieder.

\Abs \Satz Schriftliche Abstimmungen können nur zu Gegenständen erfolgen, die mehr als eine einfache Mehrheit erfordern.

\Abs \Satz Die schriftliche Abstimmung ist mindestens bis zum Ablauf des auf die nächste Sitzung folgenden Tages zu ermöglichen, höchstens jedoch drei Wochen, außer in der vorlesungsfreien Zeit\. Die Abstimmungsdauer beschließt der StuRa unmittelbar nach dem Beschluss der schriftlichen Abstimmung.

\Abs \Satz Auf eine schriftliche Abstimmung und den Abstimmungsort ist auf der nächsten Sitzung sowie im Protokoll gesondert hinzuweisen.



\section{Geheime Abstimmungen}

\Abs \Satz Zur Durchführung von geheimen Abstimmungen bildet der StuRa eine Zählkommission\. Diese wird in der Regel für die Dauer einer Sitzung bestätigt.

\Abs \Satz Die Zählkommission hat aus mindestens drei Mitgliedern zu bestehen, die selbst nicht an der Abstimmung teilnehmen.

\Abs \Satz Die Zählkommission verteilt die Stimmzettel und sammelt sie ein\. Sie öffnet und schließt die erforderlichen Wahlgänge\. Sie zählt die Stimmen aus und verkündet dem StuRa das Abstimmungsergebnis\. Sie entscheidet bei Zweifeln über die Gültigkeit eines Stimmzettels.



\section{Schriftliche, geheime Abstimmungen}

\Abs \Satz Bei schriftlichen, geheimen Abstimmungen finden die Bestimmungen der §§~13~und~14 Anwendung\. Zusätzlich gilt:
\begin{enumerate}
\item Die Zugängigkeit zur Abstimmung gilt als gesichert, wenn der Abstimmungsort während der Arbeitszeiten der Kassenwärtin zugängig ist. In diesem Fall ist sicherzustellen, dass zu den Abstimmungszeiten mindestens ein Mitglied der Zählkommission im Abstimmungsraum anwesend ist.
\item Die Teilnahme an der Abstimmung wird durch Unterschrift bestätigt. Auf Verlangen eines Mitglieds der Zählkommission ist vor der Stimmabgabe ein Ausweisdokument vorzulegen.
\end{enumerate}



\section{Ausschreibungen}

\Abs \Satz Der StuRa schreibt zu Beginn einer neuen Legislatur alle Posten aus.

\Abs \Satz Die Posten gemäß §~16,~Abs.~2,~Nr.~4 der Grundordnung müssen ausgeschrieben werden.

\Abs \Satz Die Ausschreibungen erfolgen mit einer Dauer von mindestens zwei Wochen\. Nicht besetzte Posten bleiben bis auf weiteres ausgeschrieben.

\Abs \Satz Nach Rücktritt oder Abwahl ist sofort erneut auszuschreiben.



\section{Wahlen}

\Abs \Satz Kandidaturen auf ausgeschriebene Posten werden beim Sitzungsvorstand eingereicht.

\Abs \Satz Liegt für einen ausgeschriebenen Posten eine Kandidatur vor, findet auf der nächsten ordentlichen Sitzung eine Wahl statt\. Es gelten die Fristen nach §§~5~und~16.

\Abs \Satz Kandidatinnen können nur in Anwesenheit, einzeln und funktionsgebunden gewählt werden\. Als Geschäftsführerin kann nur gewählt werden, wer für die Wahlsitzung durch einen Fachschaftsrat in den Studentenrat entsendet ist\. Kandidaturen können jederzeit zurückgezogen werden.

\Abs \Satz Jedes Mitglied der Studentenschaft kann Fragen an die Kandidatinnen stellen\. Dies ist auch zwischen zwei Wahlgängen möglich.

\Abs \Satz Im ersten und zweiten Wahlgang ist die Mehrheit der Mitglieder erforderlich\. §~19~Abs.~2 der Grundordnung findet dabei keine Anwendung\. Soweit die erforderliche Mehrheit im ersten bzw. zweiten Wahlgang nicht erreicht wurde, erfolgt ein weiterer Wahlgang.

\Abs \Satz Wahlen finden durch geheime Abstimmung statt\. Eine Kandidatin ist gewählt, wenn sie die erforderliche Mehrheit erlangt und die Wahl angenommen hat.



\section{Protokollführung}

\Abs \Satz Die Protokolle der StuRa-Sitzungen werden durch den Sitzungsvorstand angefertigt und veröffentlicht.

\Abs \Satz Das Protokoll orientiert sich am Sitzungsverlauf.

\Abs \Satz Das Protokoll hat insbesondere zu enthalten:
\begin{enumerate}
\item Datum, Beginn und Ende der Sitzung,
\item die Anwesenheitsliste mit den entsprechenden Vermerken "`unentschuldigt"', "`entschuldigt"' bzw. "`ruht"' bei den fehlenden Mitgliedern,
\item den Wortlaut der Anträge und Beschlüsse gegebenenfalls nebst zugehöriger Abstimmungsergebnisse,
\item die wesentlichen Meinungen für und wieder den Antrag sowie
\item Wortmeldungen, die zuvor ausdrücklich zu Protokoll gegeben wurden.
\end{enumerate}

\Abs \Satz Personaldebatten werden nicht protokolliert.

\Abs \Satz Das Protokoll ist nach der Genehmigung durch den StuRa von der Protokollführerin und von der Versammlungsleiterin zu unterzeichnen und unverzüglich der Öffentlichkeit zugänglich zu machen.

\Abs \Satz Waren Teile der Sitzung nicht öffentlich, so sind die Protokollteile darüber nur den Mitgliedern des StuRa zugänglich.

\Abs \Satz Das Protokoll muss spätestens eine Woche nach der Sitzung den Mitgliedern zugestellt werden.



\section{Rechenschaftsberichte}

\Abs \Satz Die Rechenschaftsberichte im Sinne dieses Paragraphen sind vierteljährlich zu erstellen, dem StuRa vorzulegen und auf den nach § 21 (4) der Grundordnung festgelegten Sitzungen mündlich zu erläutern. Diese sind:
\begin{enumerate}
\item Übersicht über die Einnahmen und Ausgaben eines Monats sowie die Auslastung der Haushaltstitel,
\item kurzer Rechenschaftsbericht über die Arbeit jedes Referats,
\item kurzer politischer Bericht, der insbesondere Bezug nimmt auf die Umsetzung der Beschlüsse und des Arbeitsprogramms des StuRa.
\end{enumerate}

\Abs \Satz Die Berichte nach Abs.~1,~Nr.~1 sind von der Geschäftsführerin Finanzen, nach Abs.~1,~Nr.~2 von der jeweiligen Geschäftsführerin, nach Abs.~1,~Nr.~3 von der Geschäftsführung zu erstellen\. Die Berichterstattung nach Abs.~1,~Nr.~1 hat schriftlich zu erfolgen, sie müssen auch der Geschäftsführung vorgelegt werden\. Der Bericht nach Abs.~1,~Nr.~1 enthält insbesondere eine Übersicht aller gezahlten Aufwandsentschädigungen.



\section{Geschäftsführung}

\Abs \Satz Die Geschäftsführung tritt wöchentlich zusammen.

\Abs \Satz Die Geschäftsführung ist beschlussfähig, wenn die Mehrheit der Geschäftsführer anwesend ist\. Sie fasst Beschlüsse mit einfacher Mehrheit.

\Abs \Satz Die Sitzung der Geschäftsführung ist öffentlich\. Auf Beschluss der Geschäftsführung kann die Sitzung geschlossen werden\. Einzelne Gäste können zugelassen werden.

\Abs \Satz Es wird ein Protokoll geführt,dabei ist die GO § 18 (3) einzuhalten\.  Das Protokoll ist den StuRa-Mitgliedern zugänglich zu machen\. Es gelten die Fristen nach §~5\. Die Protokolle sind zu veröffentlichen.

\end{multicols}

\nopagebreak
\vspace{1cm}
Inkraftgetreten am 04.~Mai~2001.
\\


\footnotesize
Geändert am 04.~Juli~2003\\
§~16 Abs.~4: einfügen von: "` ;§~12 Abs.~2 Grundordnung findet dabei keine Anwendung."'

Geändert am 10.~August~2006\\
§~2 : in Abs. 1 einfügen von: "`bis 23.00 Uhr"'; NEU Abs. 3\\
§~5 : NEU\\
§~6,alt §~5 : in Abs. 2 streichen von "`Anträge"', in Abs. 3 NEU Satz 1; NEU Abs. 4\\
§~7,alt §§~17,19 und 20 : Zusammenfassen sämtlicher Kompetenzen des Versammlungsleiters, die nicht in der Satzung geregelt sind \\
§~8,alt §~18 : NEU Abs. 2; einfügen in Abs. 3 von "`nach Ermessen des Versammlungsleiters"'; NEU Abs. 4; alt Abs. 3 wird Abs. 5\\
§~9, alt §~7: Abs. 4: NEU Punkte 4,9 und 16; Änderung von 6, alt 5: "`erneute Auszählung"' statt "`erneute Beschlussfassung wegen objektiver Unklarheit"'; Änderung von 8, alt 7: "`Beratungs-"' statt "`Sitzungspause"'; NEU Abs. 7; Abs. 8, alt 7 Streichung von "`für jede im StuRa vertretene Fachschaft oder die Geschäftsführung von einem jeweiligen Vertreter"'\\
§~10, alt 8 und 9: NEU Abs. 1: Auflistung sämtlicher Anträge; NEU Abs. 2: Definition von Frist und Form der Anträge; NEU Abs. 3: spezielle Regelungen für ordentliche Anträge; NEU Abs. 4: spezielle Regelungen für Initiativanträge; NEU Abs. 5: spezielle Regelungen für AE-Anträge; NEU Abs. 6: spezielle Regelungen zu Änderungsanträgen; NEU Abs. 7: Regelungen zu Rücknahme von Anträgen\\
§~12, alt 10 und 11: Abs. 3: Pflicht zur Auszählung bei erhöhten Mehrheiten; Abs. 6: einfügen von "`,außer wenn keine einzige Ja-Stimme abgegeben wurde."'; Abs. 8 war vorher §~11\\
§~13: NEU Abs. 2\\
§~14: NEU\\
§~15: NEU\\
§~16: Abs. 3-5 (alt) gestrichen; alter Abs. 6 wird 3; alter Abs. 7 gestrichen\\
§~17: NEU Abs. 2; Aufteilung von alt Abs. 1 in neue Abs. 1 und 3\\
§~18, alt §~16: NEU Abs. 1; streichen von alt Abs. 5 weitere Kandidaturen zwischen Wahlgängen;\\
§~19, alt 21: Abs. 1 Änderung von "`durch den bestellten Protokollführer"' in "`durch die Sitzungsleitung"'; NEU Abs. 2, in Abs. 3 (alt 2) Streichung von "`die Schwerpunkte der Debatten"' und Ersetzung durch Punkt 4;\\
§~20, alt §~22: streichen in Satz 1 von "`nicht"'; NEU Abs. 4 und 5\\
Geändert am 23.\,11.\,06\\
§~5~(3): Einfügung von Satz 2.\\
§~5~(4): Streichen von "`bis zur 2. Sitzung"' und ersetzen durch "`bis zum 7. des Folgemonats"'

Geändert am 17.~Juli~2008\\
In der gesamten Ordnung "`Sitzungsleitung"' durch "`sitzungsvorstand"' ersetzt;\\
§ 5 Abs. 1 "`dem Bericht der Geschäftsführung"' in "`den Berichten nach § 19"' und "` dem Protokoll der Sitzung der Geschäftsführung"' in "`den Beschlüssen der Geschäftsführung und der Ausschüsse"' geändert;\\
§ 5 Abs. 2 "`72 Stunden"' in "`drei Tage"' geändert;\\
§ 5 Abs. 3 "`oder Ausschuss-Beschlusses"' eingefügt;\\
alt § 5 Abs. 4 gestrichen;\\
§ 6 Abs. 1 "`der Geschäftsführung"' in "`des Sitzungsvorstandes"' geändert;\\
§ 6 Abs. 2 Nr. 2 geändert in "`Bericht der Geschäftsführung und Debatee des Berichts"';\\
§ 7 Abs. 2 "`, dies erfolgt in der Regel nach circa eineinhalb Stunden"' gestrichen;\\
§ 8 Abs. 3 Nr. 1 NEU;\\
§ 9 Abs. 4 umnummeriert, Nr. 10 NEU, Nummerierung in nachfolgenden Absätzen entsprechend geändert;\\
§ 9 Abs. 8 NEU;\\
§ 10 Abs. 1 alt Nr. 3 gestrichen;\\
§ 10 Abs. 3 "`bei der Geschäftsführung"' durch "`beim Sitzungsvorstand"' ersetzt, "`die vom StuRa behandelt werden"' eingefügt;\\
§ 10 Abs. 4 alt S. 3 gestrichen;\\
§ 12 Abs. 2 S. 2 "`5"' durch "`9"' ersetzt;\\
§ 13 Abs. 14 S. 1 "`, außer in der vorlesungsfreien Zeit"' eingefügt;\\
alt § 16 gestrichen;\\
§ 14 Abs. 2 S. 2 NEU;\\
§ 16, alt § 17 Abs. 1 "`und Referate auf Grundlage der Struktur"' gestrichen;\\
§ 16, alt § 17 Abs. 2 Verweis korrigiert;\\
§ 16, alt § 17 Abs. 3 S. 2 NEU;\\
§ 17, alt § 18 Abs. 1 "`bei der Geschäftsführung"' durch "`beim Sitzungsvorstand"' ersetzt;\\
§ 17, alt § 18 Abs. 5 Verweis korrigiert;\\
§ 18, alt § 19 Abs. 7 vollständig neugefasst;\\
§ 19 NEU;\\
§ 20 Abs. 2 alt S. 3 gestrichen;\\
§ 20 Abs. 4 S. 4 NEU;\\
§ 20 alt Abs. 5 gestrichen;\\

Geändert am 16.~Juli~2010\\
§~18~Abs.~1 "`und veröffentlicht"' hinzugefasst;\\
§~18~Abs.~2~Satz~1 "`Das Protokoll wird ergebnisorientiert geführt\."' gestrichen;\\
§~18~Abs.~3 Punkt 4 eingefügt;\\
§~20~Abs.~4~Satz~1 "`dabei ist die GO § 18 (3) einzuhalten\."'eingefügt;\\

Geändert am 13.~August~2010\\
§~9~Abs.~12 hinzugefügt;\\
§~17~Abs.3 Satz 2 eingefügt;\\
§~19~Berichte in Rechenschaftsberichte geändert und von monatlich auf vierteljährlich geändert, "`und auf den nach § 21 (4) GrO festgelegten Sitzungen mündlich zu erläutern"' hinzugefügt;\\
§~5~Abs.1~Punkt 4 dementsprechen in Rechenschaftsberichte geändert;\\
§~21~entfernt, dafür in der Satzung §~4~a hinzugefügt;\\
§~10~Abs.~1~Punkt 4 hinzugefügt;\\
§~10~Abs.~2a hinzugefügt;\\
§~10~Abs.~5 Satz 4 hinzugefügt;\\
§~10~Abs.~6 neu;\\
§~5~Abs.~3~Satz 2 "`Initiativanträge zur Aufhebung eines Gf- oder Ausschuss-Beschlusses sind auf der Sitzung, auf der dieser Beschluss bekannt gegeben wird, davon ausgenommen."' gestrichen;\\
§~5~Abs.~1 "`und der Ausschüsse"' hinzugefügt;\\

Geändert am 24. Mai 2012\\
§ 10 Abs. 2 Satz 4 hinzugefügt\\

\normalsize
%~\\*[4cm]
%\begin{center}
%\hspace*{\fill}
%\parbox{7cm}{Christoph Lüdecke\\GF Soziales}
%\hfill\parbox{7cm}{Alexander Kasten\\GF Öffentliches}
%\hspace*{\fill}
%\end{center}
%\markright{Durchführungsbestimmungen Geschäftsordnung}
\setcounter{section}{0} % ist nötig um den Paragrafenzähler zurücksetzen
\begin{multicols}{2}


\section{Durchführungsbestimmung zum Tagesordnungspunkt "`Debatte des Berichts der Geschäftsführung"' gemäß §~6~(2) der Geschäftsordnung}

Für den Bericht der Geschäftsführung und die Debatte des Berichts auf den StuRa-Sitzungen gelten folgende Bestimmungen:

\setcounter{absatz}{0}

\Abs \Satz Der Bericht der Geschäftsführung (Gf) soll ein gemeinsamer Bericht der Gf über alle Geschäftsbereiche sein.

\Abs \Satz "Debatte des Berichts” ist großzügig auszulegen: \Satz Nicht nur Themen, die im Bericht erwähnt werden, sondern auch Nachfragen und spezifische Kritik an einzelnen Geschäftsführerinnen (Referentinnen, Arbeitsgemeinschaften, Referatsmitgliedern etc.) bzw. dem Verhalten der Geschäftsführung während des Berichtszeitraums können in diesem TOP diskutiert werden.

\Abs \Satz Anfragen, die während dieses TOPs an die Gf gestellt werden, sind zu protokollieren und von der Gf möglichst sofort, spätestens jedoch innerhalb der Frist aus §~21 der Geschäftsordnung zu beantworten.

\Abs \Satz Für grundsätzlichen Diskussionsbedarf über Abläufe, Regelungen o.~ä. im StuRa sind jedoch eigene TOPs einzurichten, die nach Möglichkeit mit einer Beschlussvorlage versehen sind.

\Abs \Satz Für eine Kritik an Geschäftsführerinnen, Referentinnen, Referatsmitgliedern, Arbeitsgemeinschaften oder Angestellten des StuRa, die sehr umfangreich oder sehr grundsätzlich ist oder deren öffentliche Diskussion die Persönlichkeitsrechte der Betroffenen verletzen könnte, ist eine Personaldebatte vorzusehen.

\end{multicols}

\nopagebreak
\vspace{1cm}
Inkraftgetreten am 12.~Oktober~2006.
\\


\footnotesize
Geändert am 17.~Juli~2008\\
alt~§~1~Abs.~5~S.~2 gestrichen.


\normalsize
~\\*[4cm]
\begin{center}
\hspace*{\fill}
\parbox{7cm}{Christoph Lüdecke\\GF Soziales}
\hfill\parbox{7cm}{Alexander Kasten\\GF Öffentliches}
\hspace*{\fill}
\end{center}
%%Header Grundordnung 
%\addchap[Grundordnung]{Grundordnung (GrO)\\der Studentenschaft der TU Dresden}
%\markright{Grundordnung}
%\setcounter{section}{0} % ist nötig um den Paragrafenzähler zurücksetzen


%Header Grundordnung multicols
%\addchap[Grundordnung]{Grundordnung (GrO)\\der Studentenschaft der TU Dresden}
\markright{Grundordnung}
\setcounter{section}{0} % ist nötig um den Paragrafenzähler zurücksetzen
\begin{multicols}{2}

\section*{Vorbemerkung}
\Satz Für den gesamten Text dieser Grundordnung und ihrer Ergänzungsordnungen schließen grammatikalisch feminine Formen zur Bezeichnung von Personen solche weiblichen und männlichen Geschlechts gleichermaßen ein\. Der 															Studentenrat der TU~Dresden wird im folgenden kurz StuRa, sowie die Fachschaftsräte kurz FSR genannt.



\begin{description}
\item[1. Abschnitt] Grundsätze der Studentenschaft
\item[2. Abschnitt] Fachschaften
\item[3. Abschnitt] Studentenrat
\item[4. Abschnitt] Legislative des StuRa
\item[5. Abschnitt] Exekutive des StuRa
\item[6. Abschnitt] Schlussbestimmungen
\end{description}



\section*{1. Grundsätze der Studentenschaft}



\section{Begriffsbestimmung und Rechtsstellung}

\Abs \Satz Alle eingeschriebenen Studentinnen der Technischen Universität Dresden bilden die Studentenschaft\. Jedes gewählte Mitglied der Studentenschaft hat das Recht, die weibliche oder die männliche Bezeichnung ihres Amtes zu führen\. Ausländische 	und staatenlose Studienbewerberinnen, denen befristet bis zum Bestehen bzw. endgültigen Nichtbestehen der Sprachprüfung oder 		der Feststellungsprüfung die Rechtsstellung von Studentinnen der TU~Dresden verliehen worden ist, werden im Rahmen dieser 			Grundordnung wie eingeschriebene Studentinnen behandelt.
  
\Abs \Satz Die Studentenschaft ist eine rechtsfähige Teilkörperschaft der Universität.
  
\Abs \Satz Sie ordnet im Rahmen der gesetzlichen Regelungen, der Grundordnung der Universität und dieser Grundordnung ihre Angelegenheiten selbstständig.
  
\Abs \Satz Sie hat das Recht, sich mit Studentenschaften anderer Hochschulen zu einem Verband zusammenzuschließen.



\section{Aufgaben der Studentenschaft}

\Abs \Satz Die Studentenschaft hat folgende Aufgaben:
\begin{enumerate}
\item Vertretung der Interessen ihrer Mitglieder als Angehörige der Universität,
\item Wahrnehmung der wirtschaftlichen und sozialen Belange einschließlich der sozialen Selbsthilfe ihrer Mitglieder und Stellungnahme zu diesbezüglichen Fragen,
\item Wahrnehmung der fachlichen Belange ihrer Mitglieder und Stellungnahme zu diesbezüglichen Fragen,
\item Unterstützung der kulturellen und sportlichen Interessen ihrer Mitglieder,
\item Pflege der überörtlichen und internationalen Studentinnenbeziehungen,
\item Förderung der politischen Bildung und des staatsbürgerlichen Verantwortungsbewusstsein der Studentinnen, fern jeglicher parteipolitischer Bindung.
\end{enumerate}

\Abs \Satz Zur Durchführung ihrer Aufgaben erhebt die Studentenschaft von ihren Mitgliedern Beiträge.



\section{Rechte und Pflichten der Mitglieder}
\Abs \Satz Jede Studentin hat das Recht, an der Studentischen Selbstverwaltung mitzuwirken.

\Abs \Satz Alle Mitglieder der Studentenschaft sind berechtigt, Anfragen an die Organe der Studentenschaft gemäß §~5 zu stellen\. Ferner hat jedes Mitglied das Recht Anträge an die beschlussfassenden Organe nach §~5 zu stellen.

\Abs \Satz Jedes Mitglied der Studentenschaft hat die Pflicht zur Beitragszahlung nach Maßgabe der jeweils gültigen Beitragsordnung.

\Abs \Satz Diese Grundordnung sowie alle ihre Ergänzungsordnungen sind für die Mitglieder der Studentenschaft verbindlich.



\section{Studentenbefragung}

\Abs \Satz Der StuRa kann in Angelegenheiten nach §~16,~Abs.~2,~Nr.~1~bis~3 mit \nicefrac{2}{3}~Mehrheit der Mitglieder eine Befragung der Studentenschaft beschließen.

\Abs \Satz Eine Befragung findet ebenfalls statt, wenn es in schriftlicher Form von fünf Prozent der Mitglieder der Studentenschaft beantragt wird\. Die Organisation der Befragung obliegt in diesem Fall den Antragstellerinnen\. Die Kosten trägt grundsätzlich der StuRa.

\Abs \Satz Die Befragung wird innerhalb von vier Vorlesungswochen nach Beschlussfassung des StuRa bzw. nach Antragstellung gemäß Abs. 2 an fünf aufeinander folgenden Vorlesungstagen von einer zu bildenden Kommission, in die der StuRa Vertreterinnen entsenden kann, durchgeführt.

\Abs \Satz Die Befragung erfolgt unmittelbar, allgemein, frei, gleich und geheim.

\Abs \Satz Das Ergebnis der Befragung dient dem StuRa bei seinem weiterem Handeln als Leitlinie, wenn sich mindestens 30~\% der Mitglieder der Studentenschaft an der Befragung beteiligten.


%Einfügen Paragraph 4a der nicht in die Nummerierung passt. Sollte unbedingt irgendwann korrigiert werden.
\setcounter{section}{3}
\section{a Anfragen}
\Abs \Satz Anfragen an die Organe der Studentenschaft sind von diesen binnen 14 Tagen zu beantworten\. Dies hat auf Wunsch schriftlich zu erfolgen\. Ist eine fristgerechte Beantwortung nicht möglich, so ist die der Anfragenden eine Begründung über den Grund der Verzögerung abzugeben.
\setcounter{section}{4}



\section{Die Organe}

\Abs \Satz Beschlussfassende Organe der Studentenschaft sind:
\begin{enumerate}
\item der Studentenrat,
\item die Geschäftsführung,
\item der Sitzungsvorstand und
\item ggf. die Ausschüsse.
\end{enumerate}

\Abs \Satz Die beschlussfassenden Organe der Fachschaft sind:
\begin{enumerate}
\item der Fachschaftsrat,
\item die Vertreterinnen der Fachschaft im Studentenrat und
\item ggf. die Fachschaftsvollversammlung.
\end{enumerate}

\Abs \Satz Neben diesen Organen werden als Grundordnungsorgane mit beratender Kompetenz eingerichtet:
\begin{enumerate}
\item die Referate und
\item die Arbeitsgemeinschaften.
\end{enumerate}

\Abs \Satz Im Rahmen der AE-Ordnung werden Ämter der Exekutive wie folgt definiert:
\begin{enumerate}
\item Referatsmitarbeiterinnen handeln im Auftrag der jeweiligen Referentinnen o der Geschäftsführerinnen,
\item Referentinnen stehen einem Referat vor, haben einen klar abgegrenzten Aufgabenbereich, handeln nach Tätigkeitsbeschreibung,
\item Geschäftsführerinnen leiten ihren Geschäftsbereich an, vertreten den StuRa nach außen und fällen Beschlüsse zwischen den StuRa Sitzungen.
\end{enumerate}

%Einfügen Paragraph 5a der nicht in die Nummerierung passt. Sollte unbedingt irgendwann korrigiert werden.
\setcounter{section}{4}
\section{a Beschlussfähigkeit}
\Abs \Satz Die Beschluss fassenden Organe der Studentenschaft nach § 5 (1) sind beschlussfähig, wenn die Sitzung ordnungsgemäß einberufen wurde und mehr als die Hälfte der Mitglieder mit aktivem Stimmrecht anwesend sind\.
\setcounter{section}{5}

\section*{2. Fachschaften}



\section{Gliederung}

\Abs \Satz Die Studentenschaft gliedert sich in die folgenden Fachschaften:
\begin{enumerate}
\item Mathematik
\item Physik
\item Psychologie
\item Chemie/Lebensmittelchemie
\item Biologie
\item der Philosophischen Fakultät
\item Sprach-, Literatur- und Kulturwissenschaften
\item Allgemeinbildende Schulen
\item Sozialpädagogik/Erziehungswissenschaften (M. A.)
\item Berufspädagogik
\item Jura
\item Wirtschaftswissenschaften
\item Informatik
\item Elektrotechnik
\item Maschinenwesen
\item Bauingenieurwesen
\item Architektur/Landschaftsarchitektur
\item Forstwissenschaften
\item Geowissenschaften
\item Hydrowissenschaften
\item Verkehrswissenschaften "'Studentenschaft Friedrich List"'
\item Medizin
\item IHI Zittau "'Studierendenschaft IHI"'
\end{enumerate}



\section{Grundsätzliches}

\Abs \Satz Die Fachschaft ist eine rechtsfähige Teilkörperschaft der TU Dresden und ihrer Studentenschaft.

\Abs \Satz Sie ordnet im Rahmen der gesetzlichen Regelungen, der Grundordnung der TU Dresden und der Grundordnung der Studentenschaft ihre Angelegenheiten selbst\. Neben den Aufgaben nach §~2 fördert die Fachschaft die fachlichen Interessen der Studentinnen und betreut deren Studienangelegenheiten.

\Abs \Satz Gehören einer Fakultät mehrere Fachschaften an, bilden diese einen Konvent\. Soweit nicht anders geregelt, entsenden die FSR dafür jeweils drei Delegierte.

\Abs \Satz Jedes Mitglied der Studentenschaft ist Mitglied in genau einer Fachschaft.



\section{Zusammensetzung des Fachschaftsrat}

\Abs \Satz Der Fachschaftsrat wird von den Mitgliedern der Fachschaft nach Maßgabe der Wahlordnung der TU Dresden auf ein Jahr gewählt\. Die Mitgliedschaft im FSR endet durch Rücktritt, Exmatrikulation oder Tod.

\Abs \Satz Die Anzahl der zu wählenden Mitglieder eines FSR wird durch Beschluss des FSR festgelegt\. Sie beträgt mindestens drei, jedoch höchstens fünfundzwanzig.

\Abs \Satz Wird in einer Fachschaft kein FSR gewählt, kann der StuRa diese Fachschaft vertreten.



\section{Aufgaben und Funktionen des FSR}

\Abs \Satz Der FSR vertritt die Studentinnen einer Fachschaft im Rahmen seiner Aufgaben nach §~7~Abs.~2.

\Abs \Satz Der FSR entsendet seine Vertreterinnen in den Studentenrat.

\Abs \Satz Rechtsgeschäftliche Erklärungen müssen von mindestens zwei Mitgliedern des Fachschaftsrates gemeinschaftlich abgegeben werden.



\section{Fachschaftsordnung}

\Abs \Satz Der FSR kann sich im Rahmen des SächsHG, der Wahlordnung der TU Dresden und der Grundordnung der Studentenschaft eine Fachschaftsordnung geben.

\Abs \Satz Die Fachschaftsordnung trifft insbesondere Regelungen über Zusammensetzung, Organe und Beschlussfassung des FSR.

\Abs \Satz Beschluss und Änderung der Fachschaftsordnung bedürfen einer \nicefrac{2}{3}~Mehrheit der Mitglieder des Fachschaftsrates.

\Abs \Satz Die Fachschaftsordnung kann eine Fachschaftsvollversammlung vorsehen.

\Abs \Satz Fachschaftsordnungen und deren Änderungen treten nach Kenntnisnahme durch die Geschäftsführung des StuRa in Kraft, wenn diese keine berechtigten Zweifel an der Rechtmäßigkeit vorbringt.

\Abs \Satz In Fachschaften ohne Fachschaftsordnung oder für nicht geregelte Angelegenheiten gilt die Geschäftsordnung des StuRa entsprechend.



\section{Finanzen}

\Abs \Satz Die Fachschaften verwalten die ihnen übertragenen und selbst erwirtschafteten Mittel selbständig nach Maßgabe der Finanzordnung der Studentenschaft und verwenden sie ausschließlich für ihre Grundordnungsgemäßen Aufgaben.

\Abs \Satz Der FSR ist dem StuRa über die Verwendung seiner Gelder rechenschaftspflichtig.


\section*{3. Studentenrat}



\section{Legislatur und Amtsperioden}

\Abs \Satz Die Legislatur des StuRa beginnt mit seiner Konstituierung.

\Abs \Satz Die Amtsperiode aller Wahlämter des StuRa dauert ein Jahr, von Beginn des Sommersemester bis Ende des darauf folgenden Wintersemesters\. Ausnahme hiervon sind die Vertreterinnen des StuRa im Verwaltungsrat des Studentenwerkes.

\Abs \Satz Als Amtsträgerinnen gelten die vom StuRa gewählten Personen\. Jede Amtsträgerin kann zurücktreten\. Der Rücktritt muss schriftlich erfolgen und auf einer Sitzung des StuRa bekannt gemacht werden, gleiches gilt für Mitglieder von Referaten.

\Abs \Satz Die Abwahl einer Amtsträgerin ist nur durch ein Misstrauensvotum der Mehrheit der Mitglieder des StuRa möglich.

\Abs \Satz Amtsträgerinnen müssen voll geschäftsfähig im Sinne des Bürgerlichen Gesetzbuches (BGB) sein.

\Abs \Satz Jede Amtsträgerin hat einen Anspruch auf Weiterbildung sofern sich diese auf deren Aufgabenbereich bezieht.

\Abs \Satz Amtsträgerinnen können nur an der TU Dresden immatrikulierte Studentinnen sein.



\section{Rechtsgeschäftliche Erklärungen}

\Abs \Satz Rechtsgeschäftliche Erklärungen bedürfen eines StuRa-Beschlusses und der Schriftform\. Sie sind von zwei Geschäftsführerinnen zu unterzeichnen.

\Abs \Satz Entsprechen rechtsgeschäftliche Erklärungen dem Aufgabenbereich einer Referentin die zugleich Mitglied des StuRa ist, kann diese anstelle der zweiten Geschäftsführerin unterzeichnen.



\section{Angestellte}

\Abs \Satz Der StuRa beschäftigt eine Angestellte als Kassenwärtin.

\Abs \Satz Einrichtung und Abschaffung von Stellen zur hauptberuflichen Beschäftigung müssen vom StuRa beschlossen werden.

\Abs \Satz Über Einstellung und Entlassung von hauptberuflich Beschäftigten entscheidet der StuRa\. Die Bedingungen des Beschäftigungsverhältnisses richten sich nach TV-L~(Tarifgebiet~Ost).

\Abs \Satz Die Angestellten haben das Recht, aus der Mitte des Studentenrates eine Vertrauensperson für die laufende Legislatur zu bestimmen, die Ansprechpartnerin für Probleme mit der Dienstvorgesetzten ist.


\section*{4. Legislative des StuRa}



\section{Zusammensetzung des StuRa}

\Abs \Satz Der StuRa setzt sich aus den von den einzelnen FSR entsandten Vertreterinnen zusammen\.

\Abs \Satz Der StuRa hat maximal 39~Sitze, die wie folgt besetzt werden:
\begin{enumerate}
\item Jeder FSR entsendet eine Vertreterin (Basisvertreterin).
\item Entsprechend der Größe der jeweiligen Fachschaft können zusätzlich bis zu drei Vertreterinnen (weitere Vertreterinnen) nach folgendem Verfahren entsandt werden. Es werden pro Fachschaft drei Kennzahlen durch Multiplikation der Anzahl der Fachschaftsmitglieder mit 30,~17,~7 und anschließender Division durch die Anzahl der Mitglieder der Studentenschaft gebildet. Anhand der Kennzahlen größer Eins werden nach dem Höchstzahlverfahren die weiteren Vertreterinnen bis zur maximalen Größe des Studentenrates von 33~Basis- und weiteren Vertreterinnen entsandt.
\item Geschäftsführerinnen werden zu Vertreterinnen mit besonderem Sitz (besondere Vertreterinnen), wenn der FSR die maximal mögliche Zahl an Basis- und weiteren Vertreterinnen entsandt hat. Ist die Geschäftsführerin Basis- oder weitere Vertreterin, kann der FSR eine Vertreterin neu entsenden.
\item Eine Fachschaft darf insgesamt nicht mehr als fünf Vertreterinnen haben.
\end{enumerate}

\Abs \Satz Entsendet ein FSR weniger weitere Vertreterinnen als ihm das nach Abs.~2~Nr.~2 möglich ist, geht die Möglichkeit der Entsendung dieser Vertreterinnen nach zwei aufeinanderfolgenden Sitzungen an die nach dem Höchstzahlverfahren gemäß Abs.~2~Nr.~2 nachfolgenden Fachschaften über.

\Abs \Satz Nimmt eine Vertreterin an zwei aufeinanderfolgenden Sitzungen unentschuldigt nicht teil, ruht ihr Mandat für die Zeit ihrer weiteren Abwesenheit\. Ruhende Mandate weiterer Vertreterinnen werden wie Nichtentsendungen nach Abs.~3 behandelt\. Mitglieder, deren Mandat ruht, besitzen kein aktives Stimmrecht.

\Abs \Satz Nach Rücktritt oder Abwahl einer Geschäftsführerin hat der entsprechende FSR alle Vertreterinnen neu zu entsenden.

\Abs \Satz Fachschaftsräte, die in der ablaufenden Amtsperiode mindestens eine Geschäftsführerin gestellt haben und/oder in der folgenden Amtperiode mindestens eine Geschäftsführerin stellen, müssen zur ersten Sitzung des Sommersemesters eine neue Entsendung vornehmen.

\Abs \Satz Die Mitgliedschaft einer Vertreterin im StuRa endet mit dem Ende der Legislatur des StuRa\. Fernen endet sie durch Rücktritt, Exmatrikulation, Tod oder Rücknahme der Entsendung durch den FSR.


%Einfügen Paragraph 15a der nicht in die Nummerierung passt. Sollte unbedingt irgendwann korrigiert werden.
\setcounter{section}{14}
\section{a Beratende Mitglieder}
\Abs \Satz Die Referentin Ausländische Studierende ist qua Amt Beratendes Mitglied des Studentenrats.
\setcounter{section}{15}


\section{Aufgaben und Funktionen des StuRa} % §16

\Abs \Satz Der StuRa ist das oberste beschlussfassende Organ der Studentenschaft\. Es bringt den Willen der Studentenschaft zum Ausdruck.

\Abs \Satz Der StuRa hat folgende Aufgaben:
\begin{enumerate}
\item Richtlinien für die Erfüllung der Aufgaben der Studentenschaft zu beschließen,
\item in fakultätsübergreifenden Angelegenheiten der Studentenschaft zu beschließen,
\item die Amtsträgerinnen des StuRa zu wählen und von ihnen Rechenschaft entgegenzunehmen,
\item die Entsendung von Mitgliedern in die Referate,
\item die Vertreterinnen der Studentenschaft in sonstige, die Gesamtinteressen der Studentenschaft berührende Einrichtungen und Organe zu entsenden bzw. zu nominieren, sofern dem nicht andere Bestimmungen entgegenstehen,
\item das Arbeitsprogramm und den Haushalt beschließen,
\item die Grundordnung der Studentenschaft und deren Ergänzungsordnungen zu beschließen,
\end{enumerate}

\Abs \Satz Die Mitglieder des StuRa haben das Recht zur Einsicht in Unterlagen der Geschäftsführung.

\Abs \Satz Die Mitglieder des StuRa sind verpflichtet, ihre Aufgaben ehrenamtlich nach bestem Wissen und Gewissen zu erfüllen.



\section{Öffentlichkeit}

\Abs \Satz Der StuRa verhandelt in öffentlichen Sitzungen.

\Abs \Satz Jedes Mitglied der Studentenschaft hat Rede- und Antragsrecht.

\Abs \Satz Die Protokolle der StuRa-Sitzungen sind zu veröffentlichen.

\Abs \Satz Ausnahmen hiervon bestehen nur im Rahmen der Geschäftsordnung.



\section{Stimmrechte}

\Abs \Satz Jedes StuRa-Mitglied kann jeweils nur eine Stimme wahrnehmen\. Eine Vertretung ist nicht statthaft.

\Abs \Satz Ausnahme von Abs. 1 ist die Fachschaft Forstwissenschaften\. Sie kann eine Stellvertreterin ihrer entsandten Vertreterin ernennen\. Dieser Absatz tritt außer Kraft, wenn die Fachschaft Forstwissenschaften mehr als eine Vertreterin entsenden darf oder ihr Sitz nicht mehr in Tharandt ist.

\Abs \Satz Ausnahme von Abs. 1 ist die Fachschaft IHI Zittau\. Sie kann eine Stellvertreterin ihrer entsandten Vertreterin ernennen\. Dieser Absatz tritt außer Kraft, wenn die Fachschaft IHI Zittau mehr als eine Vertreterin entsenden darf oder ihr Sitz nicht mehr in Zittau ist.


\section{Mehrheiten}

\Abs \Satz Im Rahmen dieser Grundordnung und ihrer Ergänzungsordnungen gelten folgende Mehrheiten:
\begin{enumerate}
\item Einfache Mehrheit (Mehrheit der anwesenden Mitglieder);
\item Mehrheit der Mitglieder (Mehrheit der aktiven Stimmrechte);
\item \nicefrac{2}{3}~Mehrheit der Mitglieder (\nicefrac{2}{3}~der aktiven Stimmrechte).
\end{enumerate}

\Abs \Satz Im Rahmen der Geschäftsordnung gilt anstatt der Mehrheit der Mitglieder die \nicefrac{2}{3}~Mehrheit der anwesenden Mitglieder.

\Abs \Satz Der StuRa entscheidet grundsätzlich mit einfacher Mehrheit sofern Grundordnung und Ergänzungsordnungen keine andere Mehrheit vorschreiben.



\section{Beschlussfähigkeit und Beschlussfassung}

\Abs \Satz Der StuRa ist beschlussfähig, wenn mehr als die Hälfte seiner Mitglieder mit aktivem Stimmrecht anwesend ist.

\Abs \Satz Beschlüsse des StuRa werden, wenn von diesem nichts anderes bestimmt wird, mit der Beschlussfassung wirksam.

\Abs \Satz Der StuRa kann in seiner Amtsperiode gefasste Beschlüsse nur mit einer höheren Mehrheit gemäß §~19~Abs.~1 ändern oder aufheben; bei früheren Beschlüssen mit Ausnahme von \S~29~Abs.~3 genügt eine einfache Mehrheit.

\Abs \Satz Beschlüsse, die den Studentenrat finanziell über das Haushaltsjahr hinaus binden, sowie Grundordnungsänderungen bedürfen eines Beschlusses auf einer ordentlichen Sitzung.

\Abs \Satz Beschlüsse eines beschlussfassenden Organs der Studentenschaft mit Ausnahme des StuRa werden wirksam, wenn auf der folgenden, ordentlichen, beschlussfähigen Sitzung des StuRa das Protokoll vorliegt und diesen nicht durch einen Antrag auf Neubefassung nach §10 (6) Geschäftsordnung widersprochen wird.



\section{Ordentliche Sitzungen}

\Abs \Satz Ordentliche Sitzungen des StuRa finden in der nicht vorlesungsfreien Zeit alle zwei Wochen gemäß der Geschäftsordnung statt.

\Abs \Satz In der vorlesungsfreien Zeit finden maximal drei ordentliche Sitzungen statt, zwischen denen jeweils maximal vier Wochen liegen.

\Abs \Satz Kann eine Sitzung aufgrund eines Feiertages oder eines sonstigen vorlesungsfreien Tages nicht regulär stattfinden, wird sie um eine Woche vorgezogen\. Alle nachfolgenden Sitzungstermine verschieben sich entsprechend.

\Abs \Satz Im Juni eines Jahres werden die Termine für die ordentlichen Sitzungen der folgende Amtsperiode des StuRa veröffentlicht\. Dabei sind die Termine für die Rechenschaftsberichte festzulegen.



\section{Außerordentliche Sitzungen}

\Abs \Satz Zusätzlich zu den ordentlichen StuRa-Sitzungen sind auf Beschluss des StuRa, des Sitzungsvorstands, der Geschäftsführung oder auf Initiative von mindestens \nicefrac{1}{3} der Mitglieder des StuRa Sondersitzungen möglich.

\Abs \Satz Auf außerordentlichen Sitzungen darf nur zu den auf der Einladung enthaltenen Themen diskutiert und beschlossen werden.

\Abs \Satz In der vorlesungsfreien Zeit beträgt die Ladungsfrist für außerordentlichen Sitzungen 10~Tage\. Sie reduziert sich in der nicht vorlesungsfreien Zeit auf 72~Stunden.



\section{Der Sitzungsvorstand}

\Abs \Satz Der Sitzungsvorstand besteht aus drei vom StuRa gewählten Mitgliedern\. Zusätzlich ist die Referentin Struktur Mitglied des Sitzungsvorstandes.

\Abs \Satz Der Sitzungsvorstand leitet und strukturiert die Sitzung des StuRa\. Er ist dafür verantwortlich, dass sämtliche Unterlagen für die Sitzung rechtzeitig bereitstehen\. Näheres regelt die Geschäftsordnung.

\Abs \Satz Der Sitzungsvorstand bestimmt die Versammlungsleiterin in der Regel aus seiner Mitte\. Die Versammlungsleiterin hat die Ordnungsgewalt auf der Sitzung des StuRa\. Ihr obliegt die Auslegung der Grundordnung und Ordnungen mit Wirkung für den Verlauf der aktuellen Sitzung\. Auf außerordentlichen Sitzungen hat die Versammlungsleiterin insbesondere das Recht, Initiativen abzulehnen, die §~22~Abs.~2~und~§~20~Abs.~3 zuwiderlaufen.

\Abs \Satz Der Sitzungsvorstand ist für die Erstellung, Veröffentlichung und Verwaltung des Protokolls zuständig.

\Abs \Satz Ruht das Mandat eines Mitgliedes des StuRa gemäß §~15~Abs.~4~S.~1, hat der Sitzungsvorstand unverzüglich den entsprechenden FSR zu informieren.


%Einfügen Paragraph 23a der nicht in die Nummerierung passt. Sollte unbedingt irgendwann korrigiert werden.
\setcounter{section}{22}
\section{a Referentin Struktur}

\Abs \Satz Die Referentin Struktur ist qua Amt Mitglied im Sitzungsvorstand.

\Abs \Satz Sie ist zuständig für:
\begin{enumerate}
\item Die Berechnung der Sitze der Fachschaften im StuRa nach Grundordnung,
\item Überprüfung der Entsendungen in den Studentenrat,
\item die Information der FSR über ruhende Mandate gemäß § 15, Abs. 4, Satz 1,
\item die Überwachung der Begründungen und Entscheidungen des StuRa auf Konformität mit Ordnungen der Studentenschaft,
\item die Überwachung der Ordnungen der Studentenschaft auf Änderungsbedarf,
\item die Archivierung der Protokolle sowie der Grundordnung und der weiteren Ordnungen des StuRa,
\item Erfassung und Verwaltung der Kontaktdaten der StuRa-Mitglieder und Mitarbeiter/innen,
\item die Verwaltung der Mailinglisten, E-Mail-Verteiler und Weiterleitungen sowie
\item die Ausschreibung der Posten und Aktualisierung der Struktur und Tätigkeitsbeschreibungen.
\end{enumerate}


\setcounter{section}{23}


\section{Ausschüsse} % §24

\Abs \Satz Ein Ausschuss besteht aus 4 bis 7 Mitgliedern des StuRa, welche zum Zeitpunkt ihrer Wahl über das aktive Stimmrecht im StuRa verfügen\. Sie werden vom Studentenrat für die laufende Legislatur der Legislative gewählt\.

\Abs \Satz Ausschüsse können mit der Mehrheit der Mitglieder zu Teilaufgaben des StuRa, die dieser mit einfacher Mehrheit beschließen kann, eingerichtet werden\. Dabei müssen Name, Laufzeit, Aufgaben, Sitzungsturnus und gegebenenfalls Sonderregelungen zur Besetzung festgelegt werden.

\Abs \Satz Die Abschaffung eines Ausschusses erfolgt mit der Mehrheit der Mitglieder ungeachtet § 20 Abs. 3 \. Dies gilt nicht für in der Grundordnung festgeschriebene Ausschüsse.

\Abs \Satz Es kann ständige und nichtständige Ausschüsse geben\. Ein ständiger Ausschuss ist ein vom StuRa unbefristet eingerichteter Ausschuss, ein nichtständiger Ausschuss wird für eine bestimmte Zeit eingerichtet.

\Abs \Satz Die Sitzungen sind zu protokollieren, dabei ist § 18, Abs. 3 GO einzuhalten\. Das Protokoll ist den StuRa-Mitgliedern zugänglich zu machen\. Es gelten die Fristen nach § 5 GO. Die Protokolle sind zu veröffentlichen.



%Einfügen Paragraph 24a und b der nicht in die Nummerierung passt. Sollte unbedingt irgendwann korrigiert werden.
\setcounter{section}{23}
\section{a Förderausschuss}

\Abs \Satz Der Förderausschuss ist ein ständiger Ausschuss\. Er tagt in der Vorlesungszeit wöchentlich, in der vorlesungsfreien Zeit in einem regelmäßigen, zuvor zu veröffentlichendem Rhythmus.

\Abs \Satz Der Förderausschuss setzt sich aus der Geschäftsführerin Finanzen, sowie vier bis sechs weiteren, gemäß §24 Abs.1 gewählten Mitgliedern zusammen.

\Abs \Satz Die Aufgaben des Förderausschusses ergeben sich aus der Richtlinie über die finanzielle Förderung studentischer Projekte.

\Abs \Satz Das Protokoll enthält zusätzlich zu den Bestimmungen nach § 18, Abs. 3 Geschäftsordnung die Finanzaufstellungen der Antragsteller.

\Abs \Satz Mitglieder des Förderausschusses dürfen monatlich gemäß den Bestimmungen der AE- Ordnung Aufwandsentschädigung in Höhe von bis zu 20 Euro beantragen. 

\Abs \Satz Sind Mitglieder des Förderausschusses auch in einem anderem Sinne gemäß der AE- Ordnung AE- berechtigt, bleiben die in der AE- Ordnung geltenden Bestimmungen von Abs. 5 unberührt.


\setcounter{section}{24}

\section*{5. Exekutive des StuRa}



\section{Referate} % §25

\Abs \Satz Ein Referat setzt sich aus einer oder mehreren Referentinnen sowie ihren Mitarbeiterinnen zusammen\. Referate werden durch Beschluss vom StuRa zu abgrenzbaren Aufgabenbereichen eingerichtet.

\Abs \Satz Die Referentinnen werden vom StuRa gewählt, die Referats-Mitglieder vom StuRa entsendet.

\Abs \Satz Die Referentin leitet ihr Referat an und trägt die Verantwortung für die Arbeit des Referats\. Sie ist die Ansprechpartnerin des Referats.

\Abs \Satz Die Referate setzen das Arbeitsprogramm und die Beschlüsse des StuRa um.

\Abs \Satz Die Referentinnen sollen auf den Sitzungen der Geschäftsführung anwesend sein.



\section{Geschäftsbereiche}

\Abs \Satz Ein Geschäftsbereich setzt sich aus einer Geschäftsführerin und ein oder mehreren Referaten zusammen\. Jedes Referat wird einem Geschäftsbereich zugeordnet\. Geschäftsbereiche werden durch Beschluss des StuRa eingerichtet.

\Abs \Satz Geschäftsführerinnen werden vom StuRa gewählt\. Sie müssen für die Dauer ihrer Amtsperiode in den StuRa entsendet sein, gegebenenfalls unberührt von §~15~Abs.~2~Nr.~2 auch zusätzlich.

\Abs \Satz Die Geschäftsführerin leitet ihren Geschäftsbereich an und trägt die Verantwortung für die Arbeit und die Erstellung des vierteljährlichen Rechenschaftsberichtes\. Sie ist die Ansprechpartnerin des Geschäftsbereichs.



\section{Geschäftsführung}

\Abs \Satz Die Geschäftsführung setzt sich aus mindestens drei Geschäftsführerinnen zusammen\. Sie kann innerhalb ihrer Aufgaben Beschlüsse fassen.

\Abs \Satz Sie führt die laufenden Geschäfte des StuRa und koordiniert die Arbeit der Geschäftsbereiche und Referate.

\Abs \Satz Die Geschäftsführung vertritt den StuRa und setzt seine Beschlüsse um\. Zwischen den Sitzungen des StuRa fasst Sie nicht aufschiebbare Beschlüsse.

\Abs \Satz Aus ihrer Mitte bestimmt die Geschäftsführung eine Dienstvorgesetzte der Angestellten.

\Abs \Satz Die Geschäftsführung ist dem StuRa zur Rechenschaft verpflichtet.



%Einfügen Paragraph 27a der nicht in die Nummerierung passt. Sollte unbedingt irgendwann korrigiert werden.
\setcounter{section}{26}
\section{a Dienstvorgesetze}

\Abs \Satz Dienstvorgesetzte der Angestellten ist eine Geschäftsführerin.

\Abs \Satz Die Dienstvorgesetzte ist unter anderem zuständig für:
\begin{enumerate}
\item Lohnanweisung,
\item Urlaubsgenehmigung,
\item Festlegung der Arbeitszeit,
\item Weiterbildungsmaßnahmen,
\item Dienstbesprechungen,
\item Arbeitsschutz,
\item Anpassung des Tätigkeitsprofils und des Arbeitsvertrages sowie
\item Erstellung und Aushändigung von schriftlichen Dienstanweisungen.
\end{enumerate}

\Abs \Satz Dienstbesprechungen zwischen den Angestellten und der Dienstvorgesetzten finden monatlich statt\. Diese sind zu protokollieren und in der Personalakte abzulegen.

\Abs \Satz Dienstanweisungen sind von der Geschäftsführung zu beschließen\. Die Dienstvorgesetzte händigt diese schriftlich den Angestellten aus und legt eine Kopie in der Personalakte ab.

\setcounter{section}{27}



\section{Arbeitsgemeinschaften} %28

\Abs \Satz Eine Arbeitsgemeinschaft (AG) ist ein durch den StuRa bestätigter und unterstützter Zusammenschluss von Mitgliedern der Studentenschaft, der innerhalb der Aufgaben gemäß §~74~Abs.~3 SächsHG arbeitet.

\Abs \Satz Eine AG ist inhaltlich nicht an Beschlüsse des StuRa gebunden.

\Abs \Satz Die Arbeitsgemeinschaft kann sich jederzeit selbst auflösen.

\Abs \Satz Der StuRa kann durch Beschluss den Status der Zugehörigkeit der Arbeitsgemeinschaft zum Studentenrat aufheben.

\Abs \Satz Die AG wählt aus ihrer Mitte eine Leiterin und zeigt sie dem StuRa an\. Die AG kann ihre Angelegenheiten durch eine Grundordnung regeln, welche nach Bestätigung durch den StuRa in Kraft tritt.

\Abs \Satz Innerhalb ihres Arbeitsbereiches darf sie sich als "`AG des Studentenrates"' selbstständig in der Öffentlichkeit äußern\. Dabei vertritt sie die Meinung der Mitglieder der AG.

\Abs \Satz Eine AG hat als solche Rede- und Antragsrecht auf einer StuRa-Sitzung.

\Abs \Satz Einer AG kann entgegen §~2~Abs.~1~Nr.~1 dieser Grundordnung gestattet werden, ihren Arbeitsbereich auch auf andere Hochschulen auszudehnen, wenn die Studentenschaft der entsprechenden Hochschule zustimmt.

\Abs \Satz Einzelne Mitglieder der AG können bevollmächtigt werden, eine Geschäftsführerin bei rechtsgeschäftlichen Erklärungen gemäß §~13~Abs.~1 zu vertreten\. Die Vollmacht ist inhaltlich und finanziell zu begrenzen.

\setcounter{section}{27}
\section{b Projekte des Studentenrates}


\Abs \Satz Ein Projekt des Studentenrates (StuRa-Projekt) ist ein vom Studentenratsplenum beschlossenes einmaliges Vorhaben. Ein StuRa Projekt übernimmt außerordentliche Aufgaben, die von der Struktur des StuRa nicht oder nur unzureichend abgebildet werden können\.

\Abs \Satz Bei der Einrichtung ist das Ziel des Projekts zu benennen\.

\Abs \Satz Ein StuRa-Projekt ist befristet, kann aber verlängert werden. Bei absehbarer Langfristigkeit soll die Integration der Aufgaben in die Struktur des StuRa geprüft werden\.

\Abs \Satz Ein StuRa-Projekt ist einer Geschäftsführerin zugeordnet\.

\Abs \Satz Es ist eine Projektsprecherin zu benennen, welche das Projekt gegenüber dem StuRa vertritt und Ansprechpartnerin ist. Weitere Projektmitarbeiterinnen sind ebenfalls zu benennen\.

\Abs \Satz Die Zahl der Mitarbeiterinnen eines StuRa-Projekts kann begrenzt werden\.

\Abs \Satz Insbesondere zum Abschluss des Projektes muss dem StuRa über die Arbeit der Projektgruppe berichtet werden. In dem Bericht sind ebenfalls die aufgewandten Mittel aufzuführen\.

\Abs \Satz Änderungen an Beschlüssen zu StuRa-Projekten werden abweichend von § 20, Absatz 3 stets mit einfacher Mehrheit beschlossen, wenn sie ausschließlich Antragsbestandteile nach den Punkten (3), (5) oder (6) betreffen\.


\setcounter{section}{28}


\section*{6. Schlussbestimmungen}



\section{Ergänzungsordnungen und Richtlinien}

\Abs \Satz Zur Ergänzung dieser Grundordnung beschließt der StuRa mit \nicefrac{2}{3}~Mehrheit seiner gewählten Mitglieder folgende Ergänzungsordnungen:
\begin{enumerate}
\item Finanzordnung der Studentenschaft
\item Beitragsordnung der Studentenschaft
\item Geschäftsordnung des StuRa
\item Härtefallordnung
\item Die AE-Ordnung der Studentenschaft
\item Die Mitgliedschaftsordnung der Studentenschaft
\end{enumerate}

\Abs \Satz Diese sind Bestandteile dieser Grundordnung.

\Abs \Satz Darüber hinaus kann der StuRa mit einfacher Mehrheit Beschlüsse zu Richtlinien und Durchführungsbestimmungen fassen.



\section{Grundordnungsänderung}

\Abs \Satz Als Grundordnungsänderung ist jede Änderung dieser Grundordnung und ihrer Ergänzungsordnungen anzusehen. Grundordnungsänderungen können vom StuRa nur mit \nicefrac{2}{3}~Mehrheit seiner Mitglieder beschlossen werden.



\section{Teilnichtigkeit}

\Abs \Satz Bei Nichtigkeit einzelner Bestimmungen dieser Grundordnung oder ihrer Ergänzungsordnungen gelten die übrigen Bestimmungen fort.



\section{Veröffentlichung}

\Abs \Satz Die Grundordnung der Studentenschaft und ihre Ergänzungsordnungen sowie Änderungen sind öffentlich innerhalb der Studentenschaft bekannt zu machen und jederzeit einsehbar.



\section{Übergangsbestimmungen}

\Abs \Satz Die zum Zeitpunkt der Eingliederung der Fachschaftsrahmenordnung in die Grundordnung gültigen Fachschaftsordnungen der jeweiligen Fachschaftsräte bleiben in Kraft.
\Abs \Satz Der von der Studentenschaft IHI Zittau gewählte Studentenrat nimmt kommissarisch bis zu den nächsten Wahlen der Studierendenvertretung am IHI Zittau die Rechte und Pflichten der Fachschaft IHI Zittau wahr.


\section{Inkrafttreten}

\Abs \Satz Die Grundordnung und ihre Ergänzungsordnungen treten unmittelbar nach ihrem Beschluss durch den StuRa in Kraft\. Dies gilt für Grundordnungsänderungen entsprechend.

\Abs \Satz Mit dem Inkrafttreten dieser Grundordnung treten alle früheren Satzungen der Studentenschaft der Technischen Universität Dresden außer Kraft.

\end{multicols}

\nopagebreak
\vspace{1cm}
Inkraftgetreten am 04.~Mai~2001.
\\ 
  

\footnotesize
Geändert am 04.~Juli~2003\\
§~18 Abs.~1 : einfügen in Satz zwei von "` , gegebenenfalls unberührt von §~7 Abs.~2 Nr.~2 auch zusätzlich,"\\
§~18 Abs.~4: Satz zwei wird Satz drei; NEU Satz zwei.\\

Geändert am 10.~August~2006\\
§~3 Abs.~2 : NEU Satz zwei\\
§~9 : gestrichen;  NEU Abs. zwei bis vier\\
§~10 : NEU\\
§~15, alt \S~14 : NEU Abs. drei\\
§~19 : NEU\\
§~20 : NEU\\
§~21, alt §~18: Anpassung an geänderten Sitzungsrhythmus

Geändert am 17.~Juli~2008\\
Darlehensordnung ersatzlos gestrichen;\\
Beratungsrichtlinie ersatzlos gestrichen;\\
AE-Ordnung in Finanzordnung integriert;\\
Fachschaftsrahmenordnungen in Grundordnung integriert;\\
In der Satzung, allen Ordnungen, Richtlinien und Durchführungsbestimmungen grammatikalisch maskuline in feminine Formulierungen geändert;\\
Umsortierung der Paragraphen;\\
§~3~Abs.~2~Korrektur der Verweise;\\
§~4~Abs.~1~Korrektur der Verweise;\\
§~5~Abs.~1~Nr.~4 NEU;\\
§~5~Abs.~2~Nr.~3 neu formuliert;\\
alt~§~4~Abs.~2~Nr.~1~und~4 gestrichen;\\
alt~§~4~Abs.~3 gestrichen;\\
§§~6~bis~11 NEU, ehemals Fachschaftsrahmenordnung;\\
§~12~NEU;\\
alt~§~17~Abs.~3 gestrichen;\\
§~14~NEU, ehemals §~18~und~§~39 der Finanzordnung;\\
§~15~Abs.~1, alt~§~7~Abs.~1 "`nach Maßgabe der Fachschaftsrahmenordnung"' gestrichen;\\
§~15~Abs.~6 NEU;\\
alt~§~6~Abs.~2~Nr.~7 gestrichen;\\
§~17~Abs.~3, alt~§~11~Abs.~3 geändert in "`Die Protokolle der StuRa-Sitzungen sind zu veröffentlichen."';\\
§~20~Abs.~3, alt~§~15~Abs.~2 Korrektur der Verweise;\\
§~22~Abs.~1, alt~§~10~Abs.~1 "`des Sitzungsvorstands"' eingefügt;\\
§~23~Abs.~1, alt~§~19~Abs.~1 geändert in "`Der Sitzungsvorstand besteht aus drei vom StuRa gewählten Mitgliedern."';\\
§~23~Abs.~2, alt~§~19~Abs.~2 gestrichen;\\
§~23~Abs.~3, alt~§~20~Abs.~2 Korrektur der Verweise;\\
§~23~Abs.~4, alt~§~20~Abs.~3 "`und Verwaltung"' eingefügt;\\
alt~§~22~Abs.~7 gestrichen;\\
§~23~Abs.~5 NEU;\\
alt~§~20~Abs.~4 gestrichen;\\
§~24,~25,~26,~27 NEU;\\
alt~§§~18,~19,~23,~25,~26,~27 gestrichen;\\
§~28~Abs.~4, alt~§~30~Abs.~2 vollständig neu gefasst;\\
alt~§~31~Abs.~1~Nr.~3~und~6 gestrichen;\\
alt~§~31~Abs.~3~S.~2 gestrichen;\\
§~33~NEU;\\
Geändert am 18.~Dezember~2008\\
In §~6Grundschulpädagogik in Allgemeinbildende Schulen/Grundschule umbenannt;

Geändert am 16.~Juli~2010\\
§~23~Abs.~4 "`Veröffentlichung"' hinzugefügt;\\
§~15~Abs.~1 Satz 2 "`Eine gesonderte Vertretung nach §~75~Abs.~1~S.~7 SächsHG existiert nicht\."' gestrichen;\\
§~15~a hinzugefügt;

Geändert am 13. August 2010\\
§~15~Abs.~4 Satz 3 hinzugefügt;\\
§~20~Abs.~1 "`mit aktivem Stimmrecht"' eingefügt;\\
§~5~a~hinzugefügt;\\
§~9~Abs.~2 "`Der FSR wählt die Vertreterinnen der Gruppe der Studenten in den jeweiligen Fakultätsrat\. Sie müssen Mitglied der Fakultät, nicht jedoch des FSR sein\. Bestehen in einer Fakultät mehrere FSR, so werden die Vertreterinnen in den Fakultätsrat durch den Konvent gewählt."' ersetzt;\\
§~26~Abs.~2 "`für die Dauer ihrer Amtsperiode"' eingefügt;\\
§~15~Abs.~6 eingefügt\\
§~21~Abs.~4 Satz 2 hinzugefügt;\\
§~26~Abs.~3 "`und die Erstellung des vierteljährlichen Rechenschaftsberichtes"' eingefügt;\\
§~4a~hinzugefügt, als Ersatz für § 21 Geschäftsordnung;\\
§~12~Abs.~3 "`gleiches gilt für Mitglieder von Referaten"' hinzugefügt;\\
§~16~Abs.~2~Punkt~4 eingefügt;\\
§~25~Abs.~2 dementsprechend gekürzt und angepasst;\\
§~27~a hinzugefügt;\\
§~14~Abs.~4 hinzugefügt;\\
§~23~a hinzugefügt;\\
§~23~Abs.~1 Satz 2 dementsprechend hinzugefügt;\\
§~24~a~neu;\\
§~4~Abs.~3 dementsprechend "`Ausschuss"' in "`Kommission"' geändert; um hier nicht die Bedingung für Ausschüsse erfüllen zu müssen;\\
§~24~b hinzugefügt;\\
§~20~Abs.~5 hinzugefügt;\\

Geändert am 7. Juli 2011\\
Umbennung der Satzung in Grundordnung: Der Begriff Satzung wird zur Übersichtlichkeit auch in den Übersichten und den Verlauf in Grundordnung geändert. Der Begriff Satzung ist, wenn er auf Ergänzungsordnungen und Dokumenten der Studentenschaft verwendet mit dem der Grundordnung gleichbedeutend.\\

Geändert am 24. Mai 2012\\
§ 6 Abs. 1 Nr. 20 "`Wasserwesen"' durch "`Hydrowissenschaften"' ersetzt; \\

Geändert am 24. Mai 2012\\
§ 28 b NEU; \\

Geändert am 30. August 2012\\
§ 5 Abs. 4 NEU; \\

Geändert am 08. November 2012\\
§ 6 Abs. 1 Nr. 8 "`Grundschulen"' gestrichen; \\

Geändert am 07. Februar 2013\\
§ 6 Abs. 1 Nr. 23 IHI Zittau '"Studierendenschaft IHI"' hinzugefügt; \\
§ 18 Abs. 3 hinzugefügt; \\
§ 18 Abs. 2, Abs. 3 geändert in Ausnahme von Abs. 1 ist [\ldots]; \\
§ 33 Abs. 2 hinzugefügt; \\

Geändert am 25. Oktober 2013 \\
§ 24 Abs. 1 geändert in: "`Ein Ausschuss besteht aus 4 bis 7 Mitgliedern des StuRa, welche zum Zeitpunkt ihrer Wahl über das aktive Stimmrecht im StuRa verfügen. Sie werden vom Studentenrat für die laufende Legislatur der Legislative gewählt."'; \\
§ 24 a Abs. 1 geändert in: "`Der Förderausschuss ist ein ständiger Ausschuss. Er tagt in der Vorlesungszeit wöchentlich, in der vorlesungsfreien Zeit in einem regelmäßigen, zuvor zu veröffentlichendem Rhythmus."'; \\
§ 24 a Abs. 2 geändert in: "`Der Förderausschuss setzt sich aus der Geschäftsführerin Finanzen, sowie vier bis sechs weiteren, gemäß §24 Abs.1 gewählten Mitgliedern zusammen."'; \\
§ 24 a Abs. 5 hinzugefügt; \\
§ 24 a Abs. 6 hinzugefügt; \\
§ 22 Abs. 3 "`14"' durch "`10"' ersetzt; \\

Geändert am 27. November 2014 \\
§ 29 Abs. 1 5. und 6. hinzugefügt; \\

\normalsize
~\\*[4cm]
\begin{center}
\hspace*{\fill}
\parbox{7cm}{Jan-Malte Jacobsen\\GF Hochschulpolitik}
\hfill\parbox{7cm}{Robert Georges\\GF Finanzen}
\hspace*{\fill}
\end{center}     

%%\addchap[Durchführungsbestimmungen zur Grundordnung]{Durchführungsbestimmungen\\zur Grundordnung\\der Studentenschaft der TU Dresden}
\markright{Durchführungsbestimmungen Geschäftsordnung}
\setcounter{section}{0} % ist nötig um den Paragrafenzähler zurücksetzen
\begin{multicols}{2}
 

\section{Durchführungsbestimmung zu Entsendungen}

\setcounter{sentence}{0}

\Satz Entsendungen müssen 8~Stunden vor der StuRa-Sitzung, für die sie Wirkung haben sollen, bei der Verantwortlichen für diese Frage eingereicht sein\. Wer konkret dafür verantwortlich ist, wird auf der StuRa-Homepage veröffentlicht.


\nopagebreak
\vspace{1cm}
Inkraftgetreten am 12.~Oktober~2006.
\\


\section{Durchführungsbestimmung zu Entschuldigungen für StuRa-Sitzungen}

\setcounter{sentence}{0}

\Satz Entschuldigungen müssen mindestens 2~Stunden vor der StuRa-Sitzung, für die sie Wirkung haben sollen, bei der Verantwortlichen für diese Frage eingereicht sein\. Wer konkret dafür verantwortlich ist, wird auf der StuRa-Homepage veröffentlicht.

\nopagebreak
\vspace{1cm}
Inkraftgetreten am 12.~Oktober~2006.
\\


\end{multicols}


\normalsize
~\\*[4cm]
\begin{center}
\hspace*{\fill}
\parbox{7cm}{Christoph Lüdecke\\GF Soziales}
\hfill\parbox{7cm}{Alexander Kasten\\GF Öffentliches}
\hspace*{\fill}
\end{center}
%%\addchap[Härtefallordnung zur Beitragsordnung \S~4 Abs.~1]{Härtefallordnung\\zur Beitragsordnung §~4 Abs.~1}
\markright{Härtefallordnung}
\setcounter{section}{0}
\begin{multicols}{2}



\section{Allgemeines}

\Abs \Satz In besonders schwerwiegenden sozialen Notlagen kann die Studentenschaft der TU Dresden einzelnen Mitgliedern der Studentenschaft den Studentenschaftsbeitrag sowie die Kosten des Semestertickets auf Antrag zurückerstatten. 


\section{Antragsberechtigte}

\Abs \Satz Antragsberechtigt sind alle Mitglieder der Studentenschaft der TU Dresden\. Die Antragstellerin hat in angemessenem Umfang zur Verbesserung ihrer finanziellen Situation beizutragen\. Der Bezug von Unterhaltsleistungen sowie anderen Sozialleistungen hat Vorrang vor der Anerkennung als Härtefall.

\Abs \Satz Befindet sich die Antragstellerin im Zweitstudium, ist eine Rückerstattung nur in begründeten Ausnahmefällen möglich.

\Abs \Satz Studentinnen, die wegen familiärer Verpflichtungen beurlaubt worden sind und das Semesterticket nachkaufen und somit freiwillig Studentenschafts- und Semesterticketbeitrag zahlen, können diese zurückerstattet bekommen, wenn für sie die Regelungen dieser Ordnung zutreffen.


\section{Einkommensbegriff}

\Abs \Satz Einkommen im Sinne dieser Ordnung sind alle Einkünfte nach §2 Abs. 1 und 2 EStG (insbesondere Einkommen aus selbständiger und nicht-selbständiger Arbeit), Stipendien, alle Unterhaltsansprüche sowie alle staatlichen Sozialleistungen, insbesondere Leistungen nach dem Bundesausbildungsförderungsgesetz (BAföG), Wohngeld und Kindergeld.

\Abs \Satz Nicht zum Einkommen zählen das Elterngeld bis zu einer Höhe von 300 Euro und Mutterschaftsgeld.

\Abs \Satz Zahlungen aus Studienkrediten sind zum Einkommen nicht hinzuzurechnen.

\Abs \Satz Die Einkommensgrenze für eine Bewilligung des Antrages liegt bei 370 Euro zuzüglich angemessener Mietkosten, Wohnnebenkosten (Wasser, Strom, Heizung) und der Krankenversicherung, wenn diese selbst gezahlt werden muss\.
Lebt die Antragstellerin mit einer oder mehreren anderen Person/en (insbesondere eigenen Kindern) in einer Haushalts- und Wirtschaftsgemeinschaft so ist deren Einkommen gemeinsam zu berücksichtigen\. Für jede weitere Person erhöht sich die Einkommensgrenze aus § 3 Abs. 4 Satz 1 dieser Ordnung um 350 Euro.

\Abs \Satz Zahlt die Antragstellerin Unterhalt für ein eigenes Kind, welches sich nicht im Haushalt befindet, erhöht sich die Einkommensgrenze um den Unterhalt für das Kind, maximal jedoch 350 Euro.

\Abs \Satz Leben zwei Antragssteller in einer eingetragenen Lebenspartnerschaft oder Ehe zusammen, sind Einkommen und Freibeträge gemeinsam zu berücksichtigen.

\section{Form und Fristen}

\Abs \Satz Der Antrag ist persönlich und schriftlich bei der Geschäftsführerin Soziales bzw. bei der von der Geschäftsführung bestimmten Verantwortlichen zu stellen.

\Abs \Satz Die Antragsfrist endet einen Monat nach Beginn des Semesters auf das sich der Antrag bezieht\. Als Tag des Antragseingangs gilt der Tag des Eingangs beim Studentenrat der TU Dresden.


\section{Verfahren}

\Abs \Satz Der Antrag ist fristgerecht einzureichen\. Zur Antragstellung soll das zur Verfügung gestellte Formblatt verwendet werden\. Ein verspätet eingegangener Antrag kann berücksichtigt werden, wenn für die Verspätung besondere, nicht durch den Antragsteller zu vertretende Gründe vorliegen\. Zur Wahrung der Frist kann der Antrag vorläufig auch formlos gestellt werden. Das ausgefüllte Formblatt ist in jedem Fall gemeinsam mit den restlichen Unterlagen nachzureichen.

\Abs \Satz Der Antrag muss folgende Unterlagen enthalten: \\
- Angaben zur Person (Antragsformular) \\ - eine Kopie des Personalausweises \\ - eine Immatrikulationsbescheinigung \\ - eine schriftliche Darlegung der aktuellen Situation und Notlage \\ - die Einkommensverhältnisse nach §3 dieser Ordnung unterbrechungsfrei für 3 Monate in Kopie \\ - eine Kopie des BaföG-Ablehnungsbescheides\. 
Ist offensichtlich, dass die Antragsstellerin nicht BaföG-berechtigt ist, kann auf den Ablehnungsbescheid verzichtet werden.

\Abs \Satz Fehlende Unterlagen sind nach Aufforderung nachzureichen\. Werden fehlende Unterlagen innerhalb einer festgesetzten Frist nicht nachgereicht, wird der Antrag abgelehnt.

\Abs \Satz Die Geschäftsführerin Soziales bzw. die von der Geschäftsführung bestimmte Verantwortliche erarbeitet eine Stellungnahme und legt diese sowie den vollständigen Antrag der Geschäftsführung des Studentenrates zur Beschlussfassung vor.


\section{Haushaltsvorbehalt und Rechtsanspruch}

\Abs \Satz Die Rückerstattung wird aus Mitteln der Studentenschaft der TU Dresden geleistet\. Für die Rückerstattung im Sinne dieser Ordnung ist ein eigenständiger Haushaltstitel im Haushalt der Studentenschaft zu führen.
\Abs \Satz Eine Rückerstattung erfolgt unter dem Vorbehalt verfügbarer Mittel im zugeordneten Haushaltstitel.
\Abs \Satz Auf die Rückerstattung des Beitrages besteht kein Rechtsanspruch.
\Abs \Satz Bei Widerspruch ist der Antrag durch die Geschäfstführerin Soziales, wenn von einer beauftragten Verantwortlichen bearbeitet, zu prüfen. Ist der Antrag durch die Geschäftsführerin Soziales bearbeitet worden, ist er von einer anderen Geschäftsführerin zu prüfen\.
Ist ein Antrag nach Widerspruch angenommen worden, kann eine Rückerstattung nur erfolgen, wenn entsprechende Mittel verfügbar sind.


\section{Inkrafttreten und Übergangsbestimmungen}

\Abs \Satz Die Härtefallordnung tritt zum 01.04.2014 in Kraft\. Gleichzeitig tritt die Härtefallordnung  vom 01.10.2010 außer Kraft.
\Abs \Satz Diese Härtefallordnung findet erstmals Anwendung für alle Anträge die für das Sommersemester 2014 gestellt werden.

\end{multicols}

\nopagebreak
\vspace{1cm}



\footnotesize

Vollständig neu beschlossen am 13.~November~2008\\

Geändert am 01. Oktober 2010 \\
§ 2 Abs. 1 Satz 1 geändert in "`350 Euro"'; \\
§ 2 Abs. 1 Satz 1 geändert in "`Mietkosten (inklusive aller Wohnnebenkosten)"'; \\
§ 3 Abs. 1 Satz1 geändert in "`Einkünfte"'; \\
§ 3 Abs. 3 neu formuliert; \\
§ 3 Abs. 4 Satz 1 geändert in "`Person/en (insbesondere eigenen Kindern)"'; \\
§ 3 Abs. 4 Satz 2 geändert in "`350 Euro"'; \\
§ 3 Abs. 4 NEU; \\
§ 7 neu formuliert; \\

Geändert am 25.10.2013 \\
§ 1 Abs. 1 Satz 1 "`Studentinnen"' geändert in "`Mitgliedern der Studentenschaft"'; \\
§ 2 Abs. 1 Satz 1 "`Studentinnen"' geändert in "`Mitglieder der Studentenschaft"' und Verschiebung der Einkommensgrenze in § 3 Abs. 4 Satz 1; \\
§ 3 Abs. 4 Satz 1 eingefügt aus § 2 Abs. 1 Satz 1 und Änderung der Grenze von 350 Euro auf 370 Euro, Spezifizierung der Nebenkosten, Aufnahme der Krankenversicherung; \\
§ 3 Abs. 6 NEU;
§ 4 Abs. 1 Satz 1 "`Verantwortlichen für Soziales"' geändert zu "`Verantwortlichen"';\\
§ 5 Abs. 1 Satz 4 NEU;\\
§ 5 Abs. 2 Satz 1 vervollständigt;\\
§ 5 Abs. 2 Satz 2 NEU;\\
§ 5 Abs. 4 Satz 1 "`Verantwortliche für Soziales"' geändert zu "`Verantwortliche"';\\
§ 6 Abs. 4 NEU;\\
§ 7 Abs. 1 Datum aktualisiert;\\

\normalsize
~\\*[4cm]
\begin{center}
\hspace*{\fill}
\parbox{7cm}{Matthias Funke\\GF Finanzen}
\hfill\parbox{7cm}{Jessica Rupf \\GF Soziales}
\hspace*{\fill}
\end{center}
%%\addchap[Richtlinie zur Anerkennung von Hochschulgruppen durch den Studentenrat der TU~Dresden]{Richtlinie zur Anerkennung von Hochschulgruppen\\ durch den Studentenrat der TU Dresden}
\markright{Richtlinie zur Anerkennung von Hochschulgruppen}
\setcounter{section}{0} % ist nötig um den Paragrafenzähler zurücksetzen
\begin{multicols}{2}




\section{Status Hochschulgruppe}

\Abs \Satz Auf Antrag kann eine Gruppe von Studierenden der TU Dresden als Hochschulgruppe im Sinne dieser Richtlinie (im folgenden "`Hochschulgruppe"') anerkannt werden.

\Abs \Satz Über die Anerkennung beschließt der Studentenrat, seine Geschäftsführung oder ein Ausschuss des Studentenrates.

\Abs \Satz Die Anerkennung als Hochschulgruppe wird bis zum Ende der Legislatur ausgesprochen\. Der Antrag muss eine kurze Beschreibung der Gruppe und ihrer Ziele, eine E-Mail-Adresse und nach Möglichkeit Telefonnummer enthalten\. Es müssen Vertreterinnen im Sinne dieser Richtlinie genannt werden\. Die Hochschulgruppe erklärt sich einverstanden, dass ihre E-Mail-Adresse in einen vom Studentenrat moderierten Verteiler aufgenommen wird.

\Abs \Satz Die Anerkennung der Hochschulgruppe kann verweigert werden\. Sie ist insbesondere zu verweigern, wenn 
\begin{enumerate}
\item die Gruppe aus weniger als fünf Mitgliedern besteht,
\item die Gruppe nicht ausschließlich oder zum ganz wesentlichen Teil aus Studierenden zusammengesetzt ist,
\item Zweifel bestehen, dass Studierende die Willensbildung der Gruppe maßgeblich prägen,
\item die Anerkennung der Erfüllung der Aufgaben der Studierendenschaft aus § 74 Abs. 3 SächsHG entgegensteht,
\item die Anerkennung der Erfüllung der Aufgaben der Hochschule aus § 4 SächsHG entgegensteht,
\item die Gruppe entgegen grundsätzlicher Positionen des Studentenrates handelt.
\end{enumerate}
\Satz Sofern Tatsachen später bekannt werden, die der Anerkennung einer Hochschulgruppe entgegenstehen, ist die Anerkennung der Hochschulgruppe gemäß §~49 Abs.~2 Satz~1 VwVfG durch das Plenum des Studentenrates zu widerrufen.

\Abs \Satz Änderungen der Daten sind unverzüglich dem StuRa bekannt zu geben.


\section{Rechte von Hochschulgruppen}

\Abs \Satz Hochschulgruppen können den Materialverleih des Studentenrates nutzen\. Näheres regelt die entsprechende Richtlinie.

\Abs \Satz Hochschulgruppen können auf Wunsch auf der Internetseite des Studentenrates verlinkt werden\. Sie können sich, ihre Projekte und ihre Termine auf der dafür vorgesehenen Internetseite des Studentenrates vorstellen.

\Abs \Satz Hochschulgruppen bekommen die Möglichkeit sich in der Broschüre "`spiritus rector"' des Studentenrates kurz vorzustellen\. Sie können ihre Projekte in der Zeitung des Studentenrates vorstellen\. Sie können sich auf der dafür vorgesehenen Pinnwand im Studentenrat vorstellen.

\Abs \Satz Hochschulgruppen können die Schneidemaschine und den Broschürentacker des Studentenrates nutzen, soweit diese nicht vom Studentenrat selber benötigt werden\. Der Studentenrat kann Flugblätter, Broschüren und Plakate für die Hochschulgruppen verteilen.

\Abs \Satz Die Geschäftsführung des Studentenrates kann Hochschulgruppen bei Anliegen an andere Institutionen unterstützen.

\Abs \Satz Hochschulgruppen können ein Postfach in den Räumlichkeiten des Studentenrates bekommen.



\section{Schlussbestimmungen}
\Abs \Satz Es ergibt sich mit der Anerkennung als Hochschulgruppe kein Rechtsanspruch auf unter §~2 genannte Rechte.



\end{multicols}

\pagebreak
\vspace{1cm}
Inkraftgetreten am 29.~Juni~2006.
\\


\footnotesize
Geändert am 17.~Juli~2008\\
§~1~Abs.~3~S.~4 "`die"' ersetzt durch "`ihre"';\\
§~2~Abs.~1 "`Durchführungsrichtlinie"' ersetzt durch "`Richtlinie"';\\
alt~§~2~Abs.~7 gestrichen;\\
alt~§~2~Abs.~8 "`Punkte"' durch "`Rechte"' ersetzt und als neuer §~3~Abs.~1 aufgeführt;\\
Geändert am 13.~November~2008\\
§~1~Abs.~2 Ausschuss ergänzt;\\
§~1~Abs.~4 NEU;\\
\\
Geändert am 15.~Juli~2010\\
§~1~Abs.~1~S.~4 Korrektur des VwVfG-Verweis und hinzufügen von "`durch das Plenum des Studentenrates"' \\

\normalsize
~\\*[4cm]
\begin{center}
\hspace*{\fill}
\parbox{7cm}{Steven Seiffert\\GF Hochschulpolitik}
\hfill\parbox{7cm}{Matthias Zagermann\\GF Inneres}\hspace*{\fill}
\end{center}
%%\addchap[Richtlinie für den Materialverleih des Studentenrates der TU~Dresden]{Richtlinie für den Materialverleih\\ des Studentenrates der TU Dresden}
\markright{Richtlinie für den Materialverleih}
\setcounter{section}{0} % ist nötig um den Paragrafenzähler zurücksetzen
\begin{multicols}{2}
 

\section{Ausleihberechtigte}

\Abs \Satz Material wird vorwiegend an den Studentenrat, Fachschaftsräte und anerkannte Hochschulgruppen verliehen\. Eine Vertreterin der jeweiligen Institution muss als Verantwortliche benannt werden\. Sie ist die Ausleihende.

\Abs \Satz Eine Reservierung des Materials ist für Hochschulgruppen und Fachschaftsräte maximal drei Wochen im Voraus möglich.



\section{Ausleihbedingungen}

\Abs \Satz Bei Abholung ist in einem Übergabeprotokoll festzuhalten, welche Gegenstände ausgeliehen werden und wie hoch die jeweilige Kaution und gegebenenfalls das Nutzungsentgelt ist\. Das Übergabeprotokoll enthält ferner den Zustand aller ausgeliehenen Gegenstände.

\Abs \Satz Das Material wird grundsätzlich über eine Nacht verliehen\. Es muss am folgenden Werktag um spätestens elf Uhr zurückgegeben werden.

\Abs \Satz Bei Verlust, Diebstahl oder Beschädigung haftet die Ausleihende\. Von letzterem ausgenommen sind nur Verschleißteile und im Übergabeprotokoll festgehaltene Beschädigungen.

\Abs \Satz Für ausgeliehenes Material wird eine Kaution erhoben\. Die Kaution ist gegen Quittung bei Abholung in bar zu hinterlegen und wird bei ordnungsgemäßer Rückgabe erstattet.

\Abs \Satz Neben Gründen nach Abs.~1~und~3 werden Teile der Kaution bei verspäteter Rückgabe oder Verschmutzung einbehalten.

\Abs \Satz Bei Material mit hohen laufenden Kosten oder hohen Anschaffungskosten wird ein Nutzungsentgelt erhoben\. Es ist bei Abholung in bar zu zahlen\. Die so eingenommenen Gelder werden für Wartung oder Neubeschaffung des Materials verwendet.



\section{Schlussbestimmungen}

\Abs \Satz Der Materialbestand des Studentenrates wird in einer öffentlich zugänglichen Liste aufgeführt\. Die Liste beinhaltet die genaue Bezeichnung des Materials, die Höhe der Kaution und gegebenenfalls das Nutzungsentgelt\. Sie enthält ferner eine Auflistung, in welchen Fällen Kaution einbehalten wird und wie hoch der entsprechende Teil ist.

\Abs \Satz Die Höhe der Kaution und gegebenenfalls das Nutzungsentgelt wird von der Geschäftsführung festgelegt\. Ob für einen Teil des Materialbestands ein Nutzungsentgelt erhoben wird, entscheidet die Geschäftsführung\. Von §~1~und~§~2~Abs.~2,5~und~6~Satz~1 kann nur im Einzelfall auf Beschluss der Geschäftsführung abgewichen werden\. Die Verwaltung des Materialverleihs wird über das Servicebüro geregelt.

\end{multicols}

\nopagebreak
\vspace{1cm}
Inkraftgetreten am 29.~Juni~2006.
\\


\footnotesize
Geändert am 17.~Juli~2008\\
alt §~2~Abs.~1 gestrichen;\\
§~2~Abs.~3~S.~2 "`Tag"' durch "`Werktag"' ersetzt;\\
§~3~Abs.~2~S.~3 "`3"' durch "`2"' ersetzt.


\normalsize
~\\*[4cm]
\begin{center}
\hspace*{\fill}
\parbox{7cm}{Christoph Lüdecke\\GF Soziales}
\hfill\parbox{7cm}{Alexander Kasten\\GF Öffentliches}
\hspace*{\fill}
\end{center}
%%\addchap[Mitgliedschaftsordnung der Studentenschaft der TU Dresden]{Mitgliedschaftsordnung\\der Studentenschaft der TU Dresden}
\markright{Mitgliedschaftsordnung}
\setcounter{section}{0}
\begin{multicols}{2}
 

\section{Mitgliedschaft}

\Abs \Satz  Mitglieder der Studentenschaft der TU Dresden sind alle immatrikulierten Studentinnen der TU Dresden, die
\begin{enumerate}
\item nicht nach §~2 dieser Ordnung ausgetreten sind oder
\item nach §~3 dieser Ordnung wieder in die Studentenschaft eingetreten sind.
\end{enumerate}

\Abs \Satz  Ausländische und staatenlose Studienbewerberinnen, denen befristet bis zum Bestehen bzw. endgültigen Nichtbestehen der Sprachprüfung oder der Feststellungsprüfung die Rechtsstellung von Studentinnen der TU Dresden verliehen worden ist, werden im Rahmen dieser Ordnung wie eingeschriebene Studentinnen behandelt.

\section{Austritt aus der Studentenschaft}

\Abs \Satz  Der Austritt aus der Studentenschaft kann frühestens nach einem Semester unterbrechungsfreier Mitgliedschaft erklärt werden.

\Abs \Satz  Die Erklärung des Austrittes muss innerhalb der durch die Immatrikulationsordnung der TU Dresden festgelegten Rückmeldefrist erfolgen.

\Abs \Satz  Der Austritt ist schriftlich durch das Austrittsformular gegenüber dem Studentenrat zu erklären.

\Abs \Satz  Als Eingangszeitpunkt für die Anzeige des Austrittswunsches gilt der Zeitpunkt, zu dem das Austrittsformular dem Studentenrat vorliegt.

\Abs \Satz  Es kann eine Bearbeitungsgebühr erhoben werden.

\Abs \Satz  Der Austritt aus der Studentenschaft führt zum Verlust von Rechten und Pflichten gemäß der Grundordnung der Studentenschaft der TU Dresden, insbesondere umfasst dies
\begin{enumerate}
\item den Verlust des Wahlrechts zu den Organen der studentischen Selbstverwaltung,
\item den Verlust des Anrechts auf Angebote und Leistungen der Organe der studentischen Selbstverwaltung, die durch Mitgliedsbeiträge ermöglicht werden. 
\end{enumerate}

\section{Eintritt in die Studentenschaft}

\Abs \Satz  Die Erklärung des Eintrittes muss innerhalb der durch die Immatrikulationsordnung der TU Dresden festgelegten Rückmeldefrist erfolgen.

\Abs \Satz Der Eintritt ist schriftlich durch das Eintrittsformular gegenüber dem Studentenrat zu erklären.

\Abs \Satz Als Eingangszeitpunkt für die Anzeige des Eintrittswunsches gilt der Zeitpunkt, zu dem das Austrittsformular dem Studentenrat vorliegt.
\end{multicols}

\nopagebreak
\vspace{1cm}
Inkraftgetreten am 01.~Juli~2013.\\

\footnotesize
Beschlossen am 27.~Juni~2013\\


\normalsize
~\\*[3cm]
\begin{center}
\hspace*{\fill}
\parbox{5cm}{Felix Walter\\GF Finanzen}
\hfill\parbox{7cm}{Jessica Rupf\\GF Soziales}
\hspace*{\fill}
\end{center}

%%\addchap[Rechnernutzungsrichtlinie des Studentenrates der TU~Dresden]{Rechnernutzungsrichtlinie\\des Studentenrates der TU Dresden}
\markright{Rechnernutzungsrichtlinie}
\setcounter{section}{0}
\begin{multicols}{2}


\section{Geltungsbereich}

\Abs \Satz Die Rechnernutzungsrichtlinie (IT-Richtlinie) gilt für für alle rechen- und kommunikationstechnischen Installationen des Studentenrates (StuRa)\. Weiterhin ist diese Richtlinie im Umgang mit Daten und Diensten des Studentenrates einzuhalten. 

\Abs \Satz Neben dieser Richtlinie sind die Nutzungsbedingungen übergeordneter Netzdienste und Dienstanbieter, insbesondere des Zentrums für Informationsdienste und Hochleistungsrechnen der TU Dresden (ZIH) und des Deutschen Forschungsnetzes (DFN), sowie geltende gesetzliche Bestimmungen einzuhalten.


\section{Begriffsbestimmung}

\Abs \Satz Unter rechen- und kommunikationstechnischen Installationen werden grundsätzlich alle Arbeitsrechner, Server und Peripheriegeräte aufgefasst, die im Besitz oder Eigentum des StuRas sind oder durch das Referat Technik betreut werden\. Weiterhin fallen unter dieser Definition alle Installationen, Geräte und Einrichtungsgegenstände, die eine Nutzung der IT-Infrastruktur ermöglichen oder unterstützen. 

\Abs \Satz Unter rechen- und kommunikationstechnischen Installationen werden grundsätzlich alle Arbeitsrechner, Server und Peripheriegeräte aufgefasst, die im Besitz oder Eigentum des StuRas sind oder durch das Referat Technik betreut werden\. Weiterhin fallen unter dieser Definition alle Installationen, Geräte und Einrichtungsgegenstände, die eine Nutzung der IT-Infrastruktur ermöglichen oder unterstützen. 

\Abs \Satz Daten sind alle Informationen, die im Rahmen der Aufgabenerfüllung des Studentenrates oder aufgrund der Nutzeraktivität mittels der IT-Infrastruktur verarbeitet oder gespeichert werden. 

\Abs \Satz Unter Dienste des Studentenrates werden alle vom StuRa zur Verfügung gestellten Server- und Rechnerfunktionen verstanden\. Der Nutzerkreis von Diensten des StuRas kann auf definierte Personengruppen eingeschränkt werden. 

\Abs \Satz Software sind Programme oder Programmteile, die für den StuRa lizenziert und mittels der IT-Infrastruktur des StuRas für die Benutzung zur Verfügung gestellt werden.

\Abs \Satz Als Zugang wird die persönliche Zugangskennung eines Nutzers definiert. 
 

\section{Nutzung, Zugang und Dauer}

\Abs \Satz Die Nutzung der IT-Infrastruktur erfolgt grundsätzlich durch personengebundene Zugänge\. Die Einrichtung eines Zuganges erfolgt durch die mit der Account-Verwaltung beauftragten Personen des Studentenrates oder durch das Referat Technik.

\Abs \Satz Vor Erteilung eines Zuganges muss die IT-Richtlinie anerkannt werden\. Die Anerkennung ist schriftlich aktenkundig zu machen. 

\Abs \Satz Folgenden Personenkreisen kann ein Zugang erteilt werden: 
\begin{enumerate}
\item Mitglieder der verfassten Studentenschaft der TU Dresden
\item vom StuRa beauftragte Personen, sofern zur Auftragserfüllung die Nutzung der IT-Infrastruktur erforderlich ist
\end{enumerate}

\Abs \Satz Die Erteilung eines Zuganges ist grundsätzlich auf die Dauer eines Semesters befristet\. Auf Beschluss des StuRa-Plenums oder der Geschäftsführung, insbesondere durch Wahrnehmung eines Wahlamtes oder einer Entsendung, kann eine weiterreichende Befristung erteilt werden. 


\section{Entzug des Zuganges}

\Abs \Satz Der Zugang zur IT-Infrastruktur ist zu entziehen, wenn 
\begin{enumerate}
\item der Nutzer es selbst wünscht
\item der Nutzer nicht mehr einer Zuteilung eines Zuganges nach §~3~(3) berechtigt ist
\item der Nutzer die Anerkennung der IT-Richtlinie zurückzieht.
\end{enumerate}
 
\Abs \Satz Bei Verstößen gegen die IT-Richtlinie oder den übergeordneten Bestimmungen durch einen Nutzer kann auf Beschluss der Geschäftsführung oder des StuRa-Plenums ihm der Zugang entzogen werden\. Ist durch den Verstoß die Integrität der IT-Infrastruktur gefährdet, erfolgt eine sofortige Sperre durch das Referat Technik\. Der Vorfall ist der Geschäftsführung zu melden. 


\section{Verarbeitung personenbezogener Daten}

\Abs \Satz Gemäß §~14~(4) des Sächsischen Hochschulfreiheitsgesetzes (SächsHSFG) in Verbindung mit §~4~(1)~Punkt~2 des Sächsischen Datenschutzgesetzes (SächsDSG) werden für die Nutzung der IT-Infrastruktur personenbezogene Daten vom Nutzer erhoben. 

\Abs \Satz Für die Erteilung eines Zuganges werden vom Nutzer folgende personenbezogene Daten erhoben, verarbeitet und gespeichert: 
\begin{enumerate}
\item Vorname und Name
\item E-Mail-Adresse
\item Angaben über den Tätigkeitsbereich im Studentenrat
\item Mitgliedsstatus in der verfassten Studentenschaft
\item Beginn und Ende Studentenstatus an der TU Dresden 
\item Durch die Benutzung entstehende dienst-spezifische Metadaten, insbesondere Zeitpunkt des letzten Logins und Anzahl der versuchten Passworteingaben.
\end{enumerate}

\section{Rechte und Pflichten des Nutzers}

\Abs \Satz Der persönliche Zugang darf nur vom Nutzer selbst benutzt werden, eine Weitergabe an Dritte ist ein Verstoß gegen die IT-Richtlinie\. Beim Verlassen des Rechnerarbeitsplatzes ist der dieser so zu hinterlassen, dass eine Nutzung des Zuganges durch Dritte nicht möglich ist.

\Abs \Satz Private Tätigkeiten sind gegenüber inhaltlichen Arbeiten zurückzustellen. 

\Abs \Satz Neben den Geschäftsführern sind auch nachrangig die Referenten und die Referatsmitglieder berechtigt, die Nutzung der Rechner jederzeit zu verlangen, insofern sie den StuRa betreffende Aufgaben zu erledigen haben\. Daraufhin sind die betreffenden Rechner freizugeben.

\Abs \Satz Die Rechner sind bei Systemwartungsarbeiten des Referates Technik sofort freizugeben. 

\Abs \Satz Dem Nutzer ist nicht gestattet, auf den installierten Speichermedien nicht für den StuRa lizenzierte Programme (auch keine Spiele und Schriften) abzulegen. 

\Abs \Satz Dem Nutzer ist nicht gestattet, Änderungen an der installierten Software, insbesondere Betriebssystem, Anwendungen, Schriftarten, und den Systemeinstellungen selbst vorzunehmen. Änderungswünsche sind dem Referat Technik mitzuteilen und von diesem nach Prüfung gegebenenfalls umzusetzen. 

\Abs \Satz Zum Speichern von Daten sind ausschließlich die vom Referat Technik dafür vorgesehene Verzeichnisse zu nutzen\. Auf den allgemein zugänglichen Netzlaufwerken dürfen ausnahmslos nur Daten gespeichert werden, die dem StuRa direkt zuzuordnen sind. 

\Abs \Satz Die Speicherung von Daten bei externen Dienstleistern, die nicht die Voraussetzung gemäß § 7 SächsDSG erfüllen, ist zu unterlassen\. Hierzu zählen insbesondere Webseiten und Cloud-Dienste Dritter. 

\Abs \Satz Hinweise auf Fehler in der installierten Software, unsachgemäße Nutzung von Laufwerken, sonstige Störungen und der Verdacht auf Viren müssen umgehend den Mitgliedern des Referates Technik oder den Angestellten mitgeteilt werden. 


\section{Haftung}

\Abs \Satz Die Nutzung der IT-Infrastruktur erfolgt eigenverantwortlich. 

\Abs \Satz Ansprüche Dritter, die sich auf Handlungen des Nutzers begründen, sind von im selbst zu regulieren\. Hierzu zählen insbesondere Verstöße des Nutzer gegen das Urheber- und Markenrecht. 

\Abs \Satz Der Nutzer haftet gegenüber dem StuRa in Höhe des entstandenen Sachschadens bei vorsätzlicher oder grob fahrlässiger Beschädigung der IT-Infrastruktur. 



\end{multicols}

\nopagebreak
\vspace{1cm}
Inkraftgetreten am 24.~April~2014.
\\



\normalsize
~\\*[4cm]
\begin{center}
\hspace*{\fill}
\parbox{7cm}{Matthias Funke\\GF Finanzen}
\hfill\parbox{7cm}{Andreas Spranger\\GF Hochschulpolitik}
\hspace*{\fill}
\end{center}
%Header Wahlordnung multicols
\markright{Wahlordnung}
\setcounter{section}{0} % ist nötig um den Paragrafenzähler zurücksetzen
\begin{multicols}{2}

\section*{Vorbemerkung}
\Satz Aufgrund von § 26 Abs. 1 des Gesetzes über die Hochschulen im Freistaat Sachsen (Sächsisches Hochschulgesetz – SächsHSG) erlässt der Studentenrat der Studierendenschaft der Technischen Universität Dresden folgende Wahlordnung. Der in dieser Ordnung verwendete Begriff „Studierendenschaft“ entspricht der Studentenschaft im Sinne des § 25 SächsHSG.


%\begin{description}
%\item[1. Abschnitt] Grundsätze der Studentenschaft
%\item[2. Abschnitt] Fachschaften
%\item[3. Abschnitt] Studentenrat
%\item[4. Abschnitt] Legislative des StuRa
%\item[5. Abschnitt] Exekutive des StuRa
%\item[6. Abschnitt] Schlussbestimmungen
%\end{description}

\section*{Erster Abschnitt}

\section{Geltungsbereich und Mandatsdauer}

\Abs \Satz Diese Wahlordnung gilt für:
\begin{enumerate}
\item die Wahlen zu den Fachschaftsräten
\item die Wahlen zum Studentenrat
\end{enumerate}

\Abs \Satz Die Mitglieder des Studentenrates und der Fachschaftsräte werden für ein Jahr gewählt und bleiben bis zur Konstituierung des neuen Fachschafts- beziehungsweise Studentenrats im Amt.

\section*{Zweiter Abschnitt - Die Fachschaftsräte}
\section  {Wahlgrundsätze}
\Abs \Satz Die Wahlen sind nach den Grundsätzen des § 26 Abs. 1 SächsHSG (frei, gleich, geheim) durchzuführen.
\Abs \Satz Die Wahl muss barrierefrei gestaltet werden.

\section {Wahlorgane, Zusammensetzung und Aufgaben}
\Abs \Satz Wahlorgane sind der Wahlausschuss, die Wahlleiterin und die Abstimmungsausschüsse (§ 11 Absatz 2)\. Die Wahlbewerber dürfen weder Mitglied im Wahlausschuss noch im Abstimmungsausschuss der eigenen Fachschaft sein\. Eine gleichzeitige Mitgliedschaft in mehreren Wahlorganen ist unzulässig\. Dies betrifft nicht die gleichzeitige Mitgliedschaft des Wahlleiters im Wahlausschuss.

\Abs \Satz Der Wahlausschuss besteht auf fünf bis sieben Mitgliedern\. Die Mitglieder des Wahlausschusses werden vom Studentenrat bestellt\. Sie müssen wahlberechtigt im Sinne von § 4 Abs. 1 sein\. Diese Bestellung erfolgt so rechtzeitig, dass der Wahlausschuss und die Wahlleiterin ihre Aufgaben innerhalb der vorgeschriebenen Fristen erfüllen können\. Die Zusammensetzung des Wahlausschusses wird mit dem Protokoll des Studentenrates veröffentlicht\. Die Amtszeit des Wahlausschusses dauert bis zur erneuten Bestellung eines Wahlausschusses an\. Sie beträgt in der Regel ein Jahr.

\Abs \Satz Der Wahlausschuss nimmt die ihm durch diese Wahlordnung übertragenen Aufgaben wahr\. Er beschließt über die Regelungen von Einzelheiten der Wahlvorbereitungen und der Wahldurchführung, insbesondere über den Wahltermin.

\Abs \Satz Der Wahlausschuss wählt aus seiner Mitte die Wahlleiterin und ihre Stellvertreterin\. Bei Stimmengleichheit entscheidet das Los\. Die erste Sitzung des Wahlausschusses wird vom Geschäftsführer Finanzen des Studentenrates einberufen und von diesem bis zur Wahl der Wahlleiterin geleitet.

\Abs \Satz Die Wahlleiterin ist für die ordnungsgemäße Vorbereitung und Durchführung der Wahl verantwortlich\. Sie sorgt insbesondere für: 
\begin{enumerate}
\item die Bekanntgabe der Wahlausschreibung
\item die Erstellung des Wählerverzeichnisses
\item den Druck der Stimmzettel sowie die Bereitstellung der Wahleinrichtungen
\end{enumerate}
\Satz Sie führt die Beschlüsse des Wahlausschusses aus.

\Abs \Satz Die Sitzungen des Wahlausschusses sollen vom Wahlleiter geleitet werden und können von jedem Mitglied einberufen werden\. Der Wahlausschuss ist beschlussfähig, wenn mehr als die Hälfte der Mitglieder anwesend sind und die Sitzung ordnungsgemäß einberufen wurde\. Der Wahlausschuss entscheidet mit der Mehrheit der Stimmen der Anwesenden\. Kann in einer Angelegenheit eine Entscheidung des Wahlausschusses nicht rechtzeitig herbeigeführt werden, so entscheidet der Wahlleiter\. Von dieser Entscheidung ist der Wahlausschuss unverzüglich zu unterrichten.

\Abs \Satz Die Wahlorgane haben bei ihren Entscheidungen zu berücksichtigen, dass durch die Regelung des Wahlverfahrens und die Bestimmung des Zeitpunktes der Wahl die Voraussetzungen für eine möglichst hohe Wahlbeteiligung geschaffen werden.

\Abs \Satz Die Wahlorgane können zur Erfüllung ihrer Aufgaben Wahlhelferinnen heranziehen.

\Abs \Satz Die Wahlorgane und die Wahlhelferinnen sind zur unparteiischen und gewissenhaften Erfüllung ihrer Aufgaben verpflichtet\. Sie üben ihre Tätigkeit ehrenamtlich aus.

\section{Wahlberechtigung und Wählbarkeit}
\Abs \Satz Wahlberechtigt (aktives Wahlrecht) und wählbar (passives Wahlrecht) ist jedes Mitglied der Studierendenschaft nach § 24 Abs. 1 SächsHSG\. Gasthörerinnen besitzen kein Wahlrecht.
\newpage %Formatkorrektur, mal gucken ob wir da was besseres finden
\Abs \Satz Mitglieder der Studierendenschaft, die mehr als einer Fachschaft angehören, geben bis zur Schließung des Wählerverzeichnisses ab, in welcher Fachschaft sie ihr Wahlrecht ausüben\. Wird diese Erklärung nicht abgegeben, bestimmt sich die Wahlberechtigung nach jener Fachschaft, die für den ersten Eintrag auf dem Studentenausweis zugeordnet ist.

\Abs \Satz Mit dem Verlust des aktiven Wahlrechts entfällt auch das entsprechende passive Wahlrecht\. Die Betroffene scheidet als Mitglied aus dem Fachschaftsrat aus.

\section{Ausübung des Wahlrechts, Wählerverzeichnis}
\Abs \Satz Das aktive und passive Wahlrecht für die Wahlen nach § 1 Abs. 1 Nr. 1 können nur Wahlberechtigte ausüben, die in das Wählerverzeichnis eingetragen sind.

\Abs \Satz Das Wählerverzeichnis wird von der zentralen Universitätsverwaltung erstellt\. Die Wahlleiterin nach dieser Ordnung setzt den Kanzler der TU Dresden mit einer Vorlaufzeit von mindestens 14 Tagen über die beabsichtigte Abforderung des Wählerverzeichnisses in Kenntnis\. Das Wählerverzeichnis gliedert sich nach Fachschaften\. Im Übrigen ist es in alphabetischer Reihenfolge zu führen oder in anderer Weise übersichtlich zu gestalten\. Es muss den Namen, den Vornamen, das Geburtsdatum und das Geschlecht der Wahlberechtigten sowie ein Feld für Bemerkungen enthalten\. Rechtzeitig vor der Auslegung nach § 3 Satz 2 ist ein den Anforderungen dieser Wahlordnung entsprechender Ausdruck zu erstellen.

\Abs \Satz Am 14. Tag vor dem ersten Wahltag wird das Wählerverzeichnis geschlossen\. Es wird während der letzten sieben Arbeitstage vor der Schließung zur Einsicht ausgelegt\. Arbeitstage im Sinne dieser Ordnung sind Wochentage Montag bis Freitag mit Ausnahme der gesetzlichen Feiertage.

\Abs \Satz Gegen die Nichteintragung oder eine falsche Eintragung in ein Wählerverzeichnis kann jede Wahlberechtigte schriftlich während der Dauer der Auslegung Erinnerung bei der Wahlleiterin einlegen\. Die Wahlleiterin trifft unverzüglich, spätestens innerhalb von 3 Kalendertagen nach Schließung des Wählerverzeichnisses eine Entscheidung\. Die betroffene Person soll vorher gehört werden\. Ist die Erinnerung begründet, so berichtigt die Wahlleiterin das Wählerverzeichnis.

\Abs \Satz Eine Berichtigung hinsichtlich der in Abs. 2 Satz 4 bis 6 genannten Angaben ist von der Wahlleiterin auch nach Schließung des Wählerverzeichnisses von Amts wegen vorzunehmen. \\
\Satz Die Wahlleiterin hat auch dann eine Berichtigung des Wählerverzeichnisses vorzunehmen, wenn ihr bis zum Wahltag Tatsachen bekannt werden, die zu einem Verlust der Wahlberechtigung bzw. Wählbarkeit am Wahltag führen (z.B. Ausscheiden aus der Studierendenschaft)\. Eine Berichtigung des Wählerverzeichnisses nach dessen Schließung ist durch die Wahlleiterin in einer Anlage zum Wählerverzeichnis zu vermerken.

\section{Wahlausschreibung}
\Abs \Satz Spätestens am 28. Kalendertag vor dem ersten Wahltag erlässt die Wahlleiterin die Wahlausschreibung\. Sie wird auf den Internetseiten des Studentenrats und durch Aushang bekannt gemacht.

\Abs \Satz Die Wahlausschreibung muss folgende Punkte enthalten:
\begin{enumerate}
\item den Ort und Tag ihres Erlasses,
\item die Erklärung, dass die Vertreter der Fachschaften gewählt werden sollen,
\item den Hinweis, wer wahlberechtigt ist,
\item die Zahl der zu stellenden Vertreter,
\item die Angabe, wann und wo das Wählerverzeichnis zur Einsicht ausliegt,
\item den Hinweis, dass die Ausübung des Wahlrechtes von der Eintragung in das Wählerverzeichnis abhängt, sowie den Hinweis auf die Fristen nach § 5 Abs. 4 und 5,
\item die Aufforderung, Wahlvorschläge einzureichen, den Zeitraum für die Abgabe der Wahlvorschläge und den letzten Tag der Einreichungsfrist,
\item den Hinweis, dass nur fristgerecht eingereichte Wahlvorschläge berücksichtigt werden und dass nur gewählt werden kann, wer zur Wahl vorgeschlagen wurde,
\item den Ort, an dem die Wahlvorschläge bekannt gemacht werden,
\item den Wahltermin, den Ort und die Zeit der jeweiligen Stimmabgabe,
\item den Hinweis, dass die Möglichkeit der Briefwahl besteht;\\ zur Erläuterung ist ein Hinweis auf § 12 dieser Wahlordnung ausreichend,
\item den Hinweis darauf, dass die Wahlberechtigten keine Wahlbenachrichtigung erhalten.
\end{enumerate}

\section{Wahltermine, Zeit und Ort der Stimmabgabe}
\Abs \Satz Die Wahlen finden in der Vorlesungszeit so rechtzeitig statt, dass die konstituierenden Sitzungen der Fachschaftsräte und des Studentenrates vor dem Ende der Vorlesungszeit desselben Semesters durchgeführt werden können\. Sie sollen in der Regel im Wintersemester stattfinden.

\Abs \Satz Die Stimmabgabe ist an drei aufeinander folgenden nicht vorlesungsfreien Tagen durchzuführen\. Die Zeiten der Stimmabgabe werden vom Wahlausschuss bestimmt.

\Abs \Satz Die Wahlen finden für alle Fachschaften an den gleichen Tagen statt, die Uhrzeiten für die Stimmabgabe müssen nicht für alle Fachschaften gleich sein\. Ein Wechsel des Abstimmungsraumes innerhalb eines Abstimmungstages ist möglich\. Der Wahlausschuss stellt sicher, dass bei Wechsel des Abstimmungsraumes ein Zeitintervall von einer Stunde eingehalten wird\. Die vom Wahlausschuss beschlossenen und veröffentlichten Orte sind zwingend einzuhalten\. Die Abstimmungsräume müssen barrierefrei zugänglich sein

\section{Wahlvorschläge}
\Abs \Satz Wahlvorschläge sind nur als Einzelwahlvorschläge zulässig.

\Abs \Satz Wahlvorschläge bedürfen der Schriftform, zulässig sind auch mehrere Einzelwahlvorschläge auf einem Dokument in Tabellenform\. Aus den Wahlvorschlägen muss ersichtlich sein, dass sie die Wahl gemäß § 1 Abs. 1 Nr. 1 (Fachschaftsräte) betreffen\. Es muss weiterhin ersichtlich sein, welche Fachschaft sie betreffen\. Ein Wahlvorschlag muss den Namen, den Vornamen, den Studiengang und das Fachsemester, das Geburtsdatum, das Geschlecht sowie eine E-Mailadresse der Bewerberin enthalten.

\Abs \Satz Die Bewerberin hat auf dem Wahlvorschlag ihr Einverständnis schriftlich zu erklären oder eine Entsprechende schriftliche Erklärung gesondert abzugeben\. Mit diesem Einverständnis soll auch das Einverständnis darüber verbunden werden, dass Mitteilungen und Erklärungen der Wahlorgane gegenüber der Bewerberin in Textform (E-Mail) erfolgen können.

\Abs \Satz Eine Bewerberin darf nur für eine Fachschaft kandidieren.

\Abs \Satz Vorgeschlagene Bewerberinnen können durch schriftliche Erklärung gegenüber dem Wahlleiter ihre Bewerbung zurücknehmen, solange nicht über die Zulassung des Wahlvorschlags entschieden ist.

\Abs \Satz Wahlvorschläge können nur innerhalb der vom Wahlleiter festgesetzten Frist eingereicht werden\. Diese Frist beträgt zwei Wochen und endet regelmäßig am 14. Kalendertag vor dem ersten Wahltag.

\Abs \Satz Werbung für einen Wahlvorschlag (Wahlkampf) ist ab dem Tage der Einreichung des Wahlvorschlages zulässig.

\section{Prüfung der Wahlvorschläge}
\Abs \Satz Der Wahlausschuss prüft die Wahlvorschläge unverzüglich nach ihrem Eingang und entscheidet über ihre Gültigkeit und Zulassung\. Stellt er Mängel fest, gibt er den Wahlvorschlag an die Bewerberin mit der Aufforderung zurück, die Mängel innerhalb einer Frist von drei Kalendertagen zu beseitigen\. Werden die Mängel nicht fristgerecht beseitigt, ist der Wahlvorschlag ungültig.

\Abs \Satz Aufgrund der zugelassenen Wahlvorschläge werden vom Wahlleiter Stimmzettel erstellt\. Die Reihenfolge der Wahlvorschläge auf dem Stimmzettel wird durch den Wahlausschuss per Los bestimmt.

\Abs \Satz Spätestens am 11 Kalendertag vor dem ersten Wahltag gibt der Wahlleiter die zugelassenen Wahlvorschläge bekannt\. Mit der Bekanntgabe ist die weitere Werbung für nicht zugelassene Wahlvorschläge unzulässig.

\section{Vorbereitung der Wahl und Gestaltung der Wahlunterlagen}
\Abs \Satz Für die Wahl jedes Fachschaftsrates werden gesonderte Stimmzettel hergestellt\. Auf den Stimmzetteln sind die Wahlvorschläge jeweils in Reihenfolge der Losnummern mit den in § 8 Abs. 2 genannten Angaben aufzuführen, jedoch ohne die Angabe zu Geburtsdatum, Geschlecht und E-Mailadresse\. Auf den Stimmzetteln ist auf die Möglichkeit der Stimmabgabe nach § 11 Abs. 4 hinzuweisen\. Die Stimmzettel sind nach den Grundsätzen der Barrierefreiheit anzufertigen.

\Abs \Satz Im Übrigen entscheidet der Wahlausschuss über die äußere Gestaltung der Wahlunterlagen.

\section {Stimmabgabe}
\Abs \Satz Für jeden Abstimmungsraum wird von der Wahlleiterin ein Abstimmungsausschuss bestellt, der so groß sein soll, dass die Einhaltung von §7(4) gewährleistet ist\. Er muss mindestens aus drei Personen bestehen\. Zur Vorbereitung der Bestellung schlägt der amtierende Fachschaftsrat bis zum 21. Tag vor dem ersten Abstimmungstag eine Vorsitzende vor\. Sobald diese durch die Wahlleiterin ernannt wird, schlägt sie der Wahlleiterin mindestens zwei weitere Mitglieder vor\. Mindestens zwei Mitglieder des Abstimmungsausschusses müssen ständig im Abstimmungsraum anwesend sein, solange dieser für die Stimmabgabe geöffnet ist\. Jegliche Beeinflussung der Wahlberechtigten im Abstimmungsraum ist unzulässig\. Jedes Mitglied des Abstimmungsausschusses kann im näheren Umkreis von Wahllokalen sichtliche Beeinflussung von Wahlbeteiligten sowie den Aufenthalt von Personen untersagen die dort nicht aus dienstlichen Gründen oder zur Wahlhandlung anwesend sein müssen\. Dieser Umkreis ist zu kennzeichnen.

\Abs \Satz Die Wahlleiterin trifft Vorkehrungen, dass der Wähler den Stimmzettel in dem ihm gemäß § 7 zugewiesenen Abstimmungsraum unbeobachtet kennzeichnen kann\. Für die Aufnahme der Stimmzettel sind Wahlurnen zu verwenden\. Vor der ersten Stimmabgabe hat der Abstimmungsausschuss sicherzustellen, dass die Urne leer ist.

\Abs \Satz Die Stimmberechtigten erhalten vom Wahlvorstand beim Betreten des Abstimmungsraumes die erforderlichen Stimmzettel, sofern sie im jeweiligen Abstimmungsraum wahlberechtigt sind und noch nicht gewählt haben\. Eine Vertretung bei der Stimmabgabe ist unzulässig.

\Abs \Satz Der Wähler gibt seine Stimme ab, indem er eindeutig kenntlich macht, welche Kandidaten er wählt\. Bei jeder Wahl kann der Wahlberechtigte bis zu drei Stimmen abgeben\. Die Wählerin kann einem Bewerberin bis zu drei Stimmen geben (kumulieren) oder auch ihre drei Stimmen auf mehrere Bewerberin verteilen (panaschieren).

\Abs \Satz Vor Einwurf des gefalteten Stimmzettels in die Urne ist die Wahlberechtigung anhand des Wählerverzeichnisses zu überprüfen\. Die Wählerin hat sich auf Verlangen über seine Person auszuweisen\. Unmittelbar danach wirft sie ihren Stimmzettel in die Wahlurne\. Die Stimmabgabe ist im Wählerverzeichnis zu vermerken.

\Abs \Satz Wird die Wahlhandlung unterbrochen oder wird das Wahlergebnis nicht unmittelbar nach Abschluss der Stimmabgabe festgestellt, hat der Abstimmungsausschuss für die Zwischenzeit die Wahlurne zu verschließen und aufzubewahren\. Er hat sicherzustellen, das der Einwurf oder die Entnahme von Stimmzetteln ohne Beschädigung des Verschlusses unmöglich sind\. Bei erneuter Öffnung der Wahlurne oder bei Entnahme der Stimmzettel hat sich der Abstimmungsausschuss davon zu überzeugen, dass der Verschluss unversehrt geblieben ist.

\Abs \Satz Nach Ablauf der für die Stimmabgabe festgesetzten Zeit dürfen nur noch die Wahlberechtigten ihre Stimme abgeben, die sich zu diesem Zeitpunkt im Wahlraum aufhalten\. Nachdem diese ihre Stimmzettel in die Wahlurne eingeworfen haben und im Wählerverzeichnis vermerkt worden sind, erklärt der Abstimmungsausschuss am letzten Tag die Stimmabgabe für beendet.

\section{Briefwahl}
\Abs \Satz Die Stimmabgabe ist auch in der Form der Briefwahl zulässig.

\Abs \Satz Eine Wahlberechtigte, die eine Stimmabgabe in der Form der Briefwahl beabsichtigt, beantragt bei der Wahlleiterin schriftlich die Übersendung oder Aushändigung der Wahlunterlagen\. Der eigenhändig unterzeichnete Antrag muss:
\begin{itemize}
\item [a.] beim Antrag auf Übersendung spätestens am 14 Kalendertag
\item [b.] beim Antrag auf Aushändigung spätestens am 5. Kalendertag
\end {itemize}
vor dem ersten Wahltag bei der Wahlleiterin eingehen\. Die Wahlleiterin prüft die Wahlberechtigung. Sie sendet der Wahlberechtigten unverzüglich nach Bekanntgabe der zugelassenen Wahlvorschläge die Wahlunterlagen zu oder händigt sie aus\. Sie vermerkt die Übersendung oder Aushändigung im Wählerverzeichnis\. Ein Wahlberechtigter, bei dem im Wählerverzeichnis die Übersendung oder Aushändigung der Briefwahlunterlagen vermerkt ist, kann seine Stimme nur durch Briefwahl abgeben.

\Abs \Satz Die Wahlunterlagen bestehen aus einem Stimmzettel, einem amtlich gekennzeichneten Wahlumschlag, einem Wahlschein und einem für das Inland freigemachten Briefwahlumschlag, der die Anschrift des Wahlleiters und als Absender den Namen und die Anschrift der wahlberechtigten Person sowie den Vermerk \glqq schriftliche Stimmabgabe\grqq\, trägt\. Der Wahlschein enthält mindestens den Namen, Vornamen, die Anschrift sowie die vorgedruckte Erklärung, den beigefügten Stimmzettel persönlich gekennzeichnet zu haben.

\Abs \Satz Beim Antrag auf Aushändigung erfolgt diese im Servicebüro des Studentenrat.

\Abs \Satz Die Stimmabgabe erfolgt dadurch dass: 1. die Briefwählerin den Stimmzettel persönlich gemäß § 11 Absatz 4 kennzeichnet, in den Wahlumschlag legt, und diesen verschließt, 2. sie den Wahlschein mit der vorgedruckten Erklärung persönlich unterzeichnet, 3. sie den Wahlschein und den Wahlumschlag in den zugegangenen Briefwahlumschlag legt und diesen verschließt (Wahlbrief) und 4. der Wahlbrief rechtzeitig vor Ablauf der für die Stimmabgabe festgesetzten Frist dem Wahlleiter zugeht. % in einer Änderung könnte, man das mal untereinander schreiben

\Abs \Satz Auf dem Wahlbrief sind von der Wahlleiterin oder einer von ihr benannten Wahlhelferin Tag und Uhrzeit des Eingangs zu vermerken\. Die eingegangenen Wahlbriefe werden gezählt und ihre Anzahl in die Wahlniederschrift nach § 15 eingetragen.

\Abs \Satz Spätestens Nach Ablauf der für die Stimmabgabe festgesetzten Zeit werden zur Überprüfung die rechtzeitig eingegangenen Wahlbriefe geöffnet; die nicht rechtzeitig im Sinne von Absatz 5 eingegangenen Wahlbriefe bleiben ungeöffnet\. Die Wahlscheine werden mit den Eintragungen im Wählerverzeichnis verglichen.\\
\Satz Ein Wahlbrief wird zurückgewiesen, wenn
\begin{enumerate}
\item er nicht bis zum Ablauf der für die Stimmabgabe festgesetzten Zeit eingegangen ist,
\item er unverschlossen eingegangen ist,
\item der Wahlumschlag nicht amtlich gekennzeichnet oder mit einem Kennzeichen
versehen ist,
\item der Wahlumschlag kein mit der unterschriebenen vorgedruckten Erklärung
versehener Wahlschein beigefügt ist,
\item sich Stimmzettel außerhalb des Wahlumschlags befinden oder
\item die Angaben auf dem Wahlschein mit den Eintragungen im Wählerverzeichnis
nicht übereinstimmen und keine Berichtigung nach § 5 Abs. 6 erfolgt.
\end{enumerate}

\Abs \Satz In den Fällen des Absatz 7 Satz 3 liegt eine Stimmabgabe nicht vor\. Die zurückgewiesenen Wahlbriefe sind einschließlich ihres Inhaltes auszusondern und im Fall des Absatz 7 Satz 3 Nr. 1 ungeöffnet, im Übrigen ohne Öffnung des Wahlumschlags, der Wahlniederschrift nach § 15 als Anlage beizufügen.

\Abs \Satz Die Wahlumschläge aus nicht zurückgewiesenen Wahlbriefen werden nach der im Wählerverzeichnis vermerkten Stimmabgabe ungeöffnet in die Wahlurne gelegt.

\section{Auszählung}
\Abs \Satz Unverzüglich nach Beendigung der Stimmabgabe (§ 11 Abs. 7) sind von den Abstimmungsausschüssen die Abstimmungsergebnisse vorläufig zu ermitteln und dem Wahlausschuss zusammen mit den Wahlunterlagen zu übergeben\. Die Bildung von Zählgruppen, die mindestens aus einem Mitglied des Abstimmungsausschusses und einer Hilfskraft bestehen müssen ist zulässig\. Spätestens 6 Tage nach Beendigung der Stimmabgabe zählt der Wahlausschuss in Zweifelsfällen nach\. Die Auszählung ist hochschulöffentlich.

\Abs \Satz Sofort nach der Öffnung der Wahlurnen werden die ungeöffneten Wahlbriefe geöffnet und unter Wahrung des Wahlgeheimnisses deren Inhalt unter die übrigen Stimmzettel gemischt\. Dann werden die Stimmzettel auf ihre Gültigkeit überprüft\. Ein abgegebener Stimmzettel ist ungültig,
\begin{enumerate}
\item wenn keine Bewerberin gekennzeichnet wurde,
\item wenn er nicht als amtlich erkennbar ist,
\item wenn der Stimmzettel einen Zusatz, der nicht der Kennzeichnung der gewählten
Bewerberin dient oder einen Vorbehalt enthält,
\item wenn mehr als drei Stimmen abgegeben wurden,
\item wenn aus dem Stimmzettel der Wille der Wählerin nicht zweifelsfrei erkennbar ist
\end {enumerate}

\Abs \Satz Bei Zweifeln über die Gültigkeit oder Ungültigkeit der Stimmabgabe entscheidet der Wahlausschuss.

\Abs \Satz Der Wahlausschuss stellt für jede Wahl die Zahl der abgegebenen Stimmzettel, die Zahl der ungültigen Stimmzettel sowie die Zahlen der gültigen Stimmen fest, die auf die einzelnen Wahlvorschläge und Bewerberinnen entfallen sind\. Die Zahl der abgegebenen Stimmzettel muss mit der Zahl der Abstimmungsvermerke im Wählerverzeichnis übereinstimmen\. Ergibt sich auch nach wiederholter Zählung keine Übereinstimmung, so ist dies in der Niederschrift anzugeben und, soweit möglich, zu erläutern.

\section {Feststellung des Wahlergebnisses}
\Abs \Satz Der Wahlausschuss hat die von den Abstimmungsausschüssen getroffenen Entscheidungen über die Gültigkeit von Stimmzetteln und Stimmen zu überprüfen und gegebenenfalls das Ergebnis der Zählung zu berichtigen. Er stellt die Ergebnisse fest. Er stellt weiter die Gesamtzahl der abgegebenen Stimmen, die Zahl der ungültigen Stimmen und die Anzahl der gültigen Stimmen je Bewerberin und die damit gewählten Bewerberinnen und die Reihenfolge der Ersatzvertreter nach Maßgabe der Absätze 3 bis 5 fest.

\Abs \Satz Die Wahlleiterin gibt das festgestellte Wahlergebnis spätestens sieben Arbeitstage nach Abschluss der Wahl auf den Internetseiten des Studentenrats bekannt. Sie hat es von Amts wegen zu berichtigen, wenn innerhalb von vier Monaten nach Feststellung Schreibfehler, Rechenfehler oder ähnliche Unrichtigkeiten bekannt werden.

\Abs \Satz Zunächst werden die dem Geschlecht in der Minderheit zustehenden Mindestsitze verteilt\. Dazu werden die dem Geschlecht in der Minderheit zustehenden Mindestsitze mit Angehörigen dieses Geschlechts in der Reihenfolge der jeweils höchsten auf sie entfallenden Stimmenzahlen besetzt, sofern diese mindestens eine Stimme erhalten haben.

\setcounter{sentence}{0} %Tut mir Leid, ich hatte keine Ahnung, wie ich 3b sinnvoll einfügen sollte
(3b)\Satz Ist kein Geschlecht in einer Fachschaft mit weniger als 40\% vertreten, so findet Abs. 3 Satz 1 keine Anwendung\. Stattdessen werden dann zunächst jeweils je Geschlecht abgerundete 40\% der Sitze in der Reihenfolge der jeweils höchsten auf die Bewerberinnen entfallenden Stimmen besetzt, sofern sie Mindestens eine Stimme erhalten haben.

\Abs \Satz Maßgeblich für die Bestimmung des Geschlechtes in der Minderheit und die Anzahl der Mindestsitze einer Fachschaft ist das Wählerverzeichnis\. Die Anzahl der Mindestsitze ergibt sich aus dem aufgerundeten Anteil des Minderheitengeschlechts im Verhältnis zu der Zahl der Sitze im jeweiligen Fachschaftsrat\. Sollte es für die nach Satz 2 vorgesehenen Sitze nicht genügend Bewerberinnen des Minderheitengeschlechts geben, entfallen die restlichen Sitze jeweils auf das andere Geschlecht.

\Abs \Satz Nach der Verteilung der Mindestsitze des Geschlechts in der Minderheit nach Absatz 3 bzw. nach der Verteilung der Sitze je Geschlecht nach Absatz 3b erfolgt die Verteilung der weiteren Sitze\. Die weiteren Sitze werden mit Bewerberinnen und Bewerbern, unabhängig von ihrem Geschlecht, in der Reihenfolge der jeweils höchsten auf sie entfallenden Stimmenzahlen besetzt.

\Abs \Satz Entfällt auf mehrere Bewerberinnen die gleiche Stimmenanzahl, so entscheidet der Wahlausschuss in einem zu protokollierenden Verfahren durch das Los über die Reihung der Kandidaten. Zuvor sind die strittigen Stimmen erneut auszuzählen\. Auf das Verfahren nach Satz 1 und 2 kann verzichtet werden, wenn alle betreffenden Kandidaten einen Sitz im Fachschaftsrat erhalten\. Die Entscheidung des Loses ist nicht anfechtbar.

\Abs \Satz Gibt es mehrere Bewerber mit mindestens einer Stimme als Sitze vorhanden sind, so sind die nicht gewählten Bewerber in absteigender Reihenfolge ihrer Stimmanzahl Ersatzvertreter in der nach Absatz (4) vorgesehenen Aufteilung.

\section {Wahlniederschrift, Aufbewahrung von Wahlunterlagen}
\Abs \Satz Über die Verhandlung des Wahlausschusses und seine Beschlüsse sowie über die Wahlhandlungen und die Tätigkeit der Wahlorgane sind Niederschriften zu fertigen\. Die Niederschriften über die Tätigkeit der Wahlorgane werden von den Mitgliedern des jeweiligen Wahlorgans, die übrigen von der Wahlleiterin unterzeichnet.

\Abs \Satz Die Wahlniederschriften sollen insbesondere den Gang der Wahlhandlung aufzeichnen, das Wahlergebnis festhalten und besondere Vorkommnisse vermerken.

\Abs \Satz Die Wählerverzeichnisse, Stimmzettel und Wahlniederschriften sind bis zum Ablauf der Amtszeit der gewählten VertreterInnen aufzubewahren.

\section{Annahme der Wahl}
\Abs \Satz Die Wahlleiterin hat die Gewählten unverzüglich von ihrer Wahl schriftlich zu verständigen. Die Wahl gilt als angenommen, wenn nicht spätestens am fünften Tag nach Zugang der Benachrichtigung der Wahlleiterin eine Ablehnung der Wahl in schriftlicher Form aus wichtigem Grund vorliegt. Ob ein wichtiger Grund vorliegt entscheidet der Wahlausschuss.

\Abs \Satz Nach Annahme der Wahl können die Gewählten von ihrem Amt nur zurücktreten, wenn der Ausübung des Amtes wichtige Gründe entgegenstehen\. Über die Annahme des Rücktritts entscheidet die Wahlleiterin.

\section {Nachrücken von Ersatzvertretern}
\Abs \Satz Wird die Wahl von einer Person rechtswirksam nicht angenommen, rückt der Ersatzvertreter nach, der gemäß § 14 in der Reihenfolge der Ersatzvertreter der Nächste ist\. Sind Ersatzvertreter nicht vorhanden, verringert sich die Zahl der Sitze des jeweiligen Fachschaftsrates entsprechend.

\Abs \Satz Scheidet eine gewählte Vertreterin aus, gilt Absatz 1 und § 16 entsprechend.

\section{Wahlprüfung}
\Abs \Satz Jede Wahlberechtigte kann nach der Bekanntgabe des Wahlergebnisses die Wahl innerhalb von 6 Kalendertagen unter Angabe von Gründen anfechten. Die Anfechtung erfolgt durch schriftliche Erklärung gegenüber der Wahlleiterin.

\Abs \Satz Die Anfechtung ist begründet, wenn wesentliche Vorschriften über das Wahlrecht, die Wählbarkeit oder das Wahlverfahren verletzt worden sind und diese Verletzung zu einer fehlerhaften Sitzverteilung geführt hat oder hätte führen können. Eine Anfechtung der Wahl mit der Begründung, das eine Wahlberechtigte an der Ausübung ihres Wahlrechtes gehindert gewesen sei, weil sie nicht oder nicht richtig in das Wählerverzeichnis eingetragen worden sei oder das eine Person an der Wahl teilgenommen habe, die zwar in das Wählerverzeichnis eingetragen, aber nicht wahlberechtigt gewesen sei, ist nicht zulässig.

\Abs \Satz Über die Anfechtung entscheidet der Wahlausschuss. Der Beschluss ist schriftlich zu begründen, mit einer Rechtsbehelfsbelehrung zu versehen und der Antragstellenden sowie der unmittelbar betroffenen Person zuzustellen. Ist die Anfechtung begründet, hat der Wahlausschuss entweder das Wahlergebnis bei fehlerhafter Auszählung zu berichtigen oder die Wahl in dem erforderlichen Umfang für ungültig zu erklären und insoweit eine Wiederholungswahl anzuordnen. Vorbehaltlich einer anderweitigen Entscheidung im Wahlprüfungsverfahren wird bei der Wiederholungswahl nach den gleichen Vorschlägen und aufgrund des gleichen Wählerverzeichnisses gewählt wie bei der für ungültig erklärten Wahl; Wirkt sich ein Verstoß über die Sitzverteilung nur in einer Fachschaft aus, ist nur diese Wahl für ungültig zu erklären und zu wiederholen. Eine Wiederholung der Wahl ist unverzüglich durchzuführen. Die Wahlleiterin legt den Wahltermin und die Zeit der Stimmabgabe fest.

\section{Fristen}
\Abs \Satz Soweit für die Stellung von Anträgen oder die Einreichung von Vorschlägen die Wahrung einer Frist vorgeschrieben ist, läuft die Frist am letzten Tag um 16 Uhr ab. § 12 Abs. 5 Nr. 4 bleibt unberührt.

\section{Konstituierung der Fachschaftsräte}
\Abs \Satz Die Fachschaftsräte konstituieren sich frühestens 7 und spätestens 21 Kalendertage nach der Bekanntgabe der Wahlergebnisse.

\section*{Dritter Abschnitt - Der Studentenrat}
\section{Wahl des Studentenrats}
\Abs \Satz Der Studentenrat setzt sich aus den von den einzelnen Fachschaftsräten entsandten Vertretern zusammen.

\Abs \Satz Der Studentenrat hat maximal 39 Sitze, die wie folgt besetzt werden:
\begin{enumerate}
\item Jeder Fachschaftsrat entsendet durch Wahl einen Vertreter (Basisvertreter)
\item \Satz Entsprechend der Größe der jeweiligen Fachschaft können zusätzlich bis zu drei Vertreter nach folgendem Verfahren entsandt werden\. Es werden pro Fachschaft drei Kennzahlen durch Multiplikation der Anzahl der Fachaftsmitglieder mit 30, 17, 7 und anschließender Division durch die Anzahl der Mitglieder der Studierendenschaft gebildet\. Anhand der Kennzahlen größer eins werden nach dem Höchstzahlverfahren die weiteren Vertreter bis zur maximalen Größe des Studentenrates von 33 Basis- und weiteren Vertretern entsandt.
\item Für Fachschaften die mehr als einen Vertreter nach Punkt 1 und 2 entsenden muss jedes Geschlecht mindestens zur abgerundeten Hälfte vertreten sein.
\end{enumerate}

\Abs \Satz Geschäftsführer werden zu Vertretern mit besonderem Sitz (besondere Vertreter), wenn der Fachschaftsrat die maximal mögliche Zahl an Basis- und weiteren Vertretern entsandt hat\. Ist der Geschäftsführer Basis- oder weiterer Vertreter, kann der Fachschaftsrat einen Vertreter neu entsenden.

\Abs \Satz Eine Fachschaft darf insgesamt nicht mehr als fünf Vertreter haben.

\Abs \Satz Entsendet ein Fachschaftsrat weniger weitere Vertreter als ihm das nach Abs. 2 Nr. 2 möglich ist, geht die Möglichkeit der Entsendung dieser Vertreter nach zwei aufeinander folgenden Sitzungen an die nach dem Höchstzahlverfahren gemäß Absatz 2 Nr. 2 nachfolgenden Fachschaften über.

\Abs \Satz Nimmt ein Vertreter an zwei aufeinander folgenden Sitzungen unentschuldigt nicht teil, ruht sein Mandat für die Zeit seiner weiteren Abwesenheit\. Ruhende Mandate weiterer Vertreter werden wie Nichtentsendungen nach Abs. 3 behandelt.

\Abs \Satz Nach Rücktritt oder Abwahl eines Geschäftsführers hat der entsprechende Fachschaftsrat alle Vertreter neu zu entsenden.

\Abs \Satz Die Mitgliedschaft eines Vertreters im Studentenrat endet mit dem Ende der Legislatur des Studentenrates. Ferner endet sie durch Rücktritt, Exmatrikulation, Tod oder Rücknahme der Entsendung durch den Fachschaftsrat.

\section{Konstituierung des Studentenrats}
\Abs \Satz Der Studentenrat konstituiert sich spätestens 28 Tag nach der Bekanntgabe der Wahlergebnisse gemäß § 14 Abs. 2.

\section*{Vierter Abschnitt}
\section{Übergangsbestimmungen}
\Abs \Satz Die Regelung des § 7 Absatz 3 Satz 2 tritt erst im Jahr 2010 in Kraft.

\Abs \Satz Alle Fristen der Wahl der Fachschaftsräten richten sich 2009 nach den Fristen der Universitätswahlen.

\Abs \Satz Mit Inkrafttreten dieser Wahlordnung werden sämtliche anders lautenden Regelungen zur Wahl und der darauf folgenden Zusammensetzung der Fachschaftsräte und des Studentenrates der Technischen Universität ungültig.
\end{multicols}
\nopagebreak
\vspace{1cm}
Inkraftgetreten am 10.~Dezember~2008.
\\ 
  

\footnotesize
Geändert am 13.~August~2009\\
§~3 Abs.~2 : NEU Satz zwei\\
§~9 : gestrichen;  NEU Abs. zwei bis vier\\

Geändert am 06.~Januar~2014\\
§~3 Abs.~2 : NEU Satz zwei\\
§~9 : gestrichen;  NEU Abs. zwei bis vier\\

\normalsize
~\\*[4cm]
\begin{center}
\hspace*{\fill}
\parbox{7cm}{Jessica Rupf\\GF Soziales}
\hfill\parbox{7cm}{Matthias Funke\\GF Finanzen}
\hspace*{\fill}
\end{center}     

%\include{heft-satzung-db-col}

\end{document}
