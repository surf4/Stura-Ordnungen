% Normaldruck
\documentclass[10pt, a4paper, smallheadings, halfparskip, cleardoubleplain, pagesize, notitlepage, oneside, tocleft]{scrartcl}
%
% Entwurfsdruck (ohne Bilder usw.)
%\documentclass[10pt, a4paper, smallheadings, halfparskip, cleardoubleplain, pagesize, notitlepage, twoside, tocleft, draft]{scrartcl}

\usepackage[ngerman]{babel}
\usepackage[utf8]{inputenc}
\usepackage[T1]{fontenc}
\usepackage{graphicx}
\usepackage[bindingoffset=0.5cm,inner=1.5cm,outer=1.5cm,top=1cm,bottom=1cm,includeheadfoot]{geometry}
\usepackage{units}     %nötig für die netten Brüche
\usepackage{multicol}  %nötig für schönen zweispaltigen Satz

\addtokomafont{section}{\centering}  %  ist format für Paragrafen
\addtokomafont{subsection}{\centering}  %  ist format für Absätze
\renewcommand*{\raggedsection}{\centering} % sorgt für mittige Kapitelüberschriften

\newcounter{absatz}[section] %Zähler für Absätze
\newcounter{sentence}[absatz] %Zähler für Sätze

\renewcommand*{\thesection}{\S\ensuremath{\,}\arabic{section}} %ausgabeformat fr paragraphen
\renewcommand*{\theabsatz}{(\arabic{absatz})}   % ausgabeformat fr absatzzhler

%\setcounter{secnumdepth}{5} % damit bis zur Ebene paragraph die Zhler ausgegeben werden
%\setcounter{tocdepth}{1} %damit nur die kapitelberschriften im inhaltsverzeichnis sind

%Absatzzhler
\newcommand*{\Abs}{\stepcounter{absatz}\theabsatz}

%Satzzhler
\renewcommand{\.}{. \Satz} %Definiert \. um auf \Satz, damit dieses Zeichen ebenfalls fr eine Satzzhlung sorgt.
\newcommand{\Satz}[1][\thesentence]{\stepcounter{sentence}\setcounter{sentence}{#1}\textsuperscript{\thesentence}}
\newcommand{\s}{\Satz}  %Michas altes Kommando fr Satzzhler; ich war zu faul, alles zu ersetzen; betrifft: S, GO, AEO

\setlength{\parindent}{0pt}
\setlength{\columnsep}{1cm}
\setlength{\columnseprule}{0.5pt}
%\pagestyle{myheadings}
%\markboth{Studentenrat der TU Dresden}{}
\hyphenation{Ta-ges-ord-nungs-punkte}

\begin{document}
%Kopfzeile Titelseite
 \titlehead{
    \begin{tabular}[b]{c}
    \includegraphics[height=2.1cm,keepaspectratio=true]{bilder/Logo_Graustufen.jpg}\\ %fr kleine Dateien
%     \includegraphics[height=1.5cm,keepaspectratio=true]{StuRa-logo.jpg}\\ %fr größere Dateien
%    \large{\textbf{STUDENTENRAT}}
    \end{tabular}
    \hfill
    \includegraphics[height=1.7cm,keepaspectratio=true]{bilder/TU_Logo_SW.pdf}
    \vskip5pt \hrule
    }
    
    
    
%berschriftenzeile auf Titelseite --> in \title muss der jeweilige titel eingefgt werden
 \vspace{3cm}
 \title{Finanzordnung\\des Studierendenrates der TU Dresden}
 \author{\normalsize{Erstellt am \today.}}
 \date{\ }
%Titel ausgeben
 \maketitle
 
%TOC (Liste aller Paragraphen zweispaltig auf Titelseite ausgeben)
%\begin{multicols}{2}
\tableofcontents
%\end{multicols}

%jeweiliges Hauptdokument einbinden
\normalsize

%		 \include{heft-grundordnung-col}
%    \include{heft-mitgliedschaftsordnung-col}
%    \include{heft-satzung-col}
%    \include{heft-satzung-db-col}
%   \include{heft-geschaeftsordnung-col}
%    \include{heft-geschaeftsordnung-db-col}
%    \include{heft-finanzordnung}
%    \include{heft-foerderrichtlinie-col}
%    \include{heft-beitragsordnung-col}
%    \include{heft-haertefallordnung}
%    \include{heft-rechnernutzungsrichtlinie-col}
%    \include{heft-hochschulgruppenrichtlinie-col}
%    \include{heft-materialverleihrichtlinie-col}

\newpage
ä ö ü ß
\end{document}
