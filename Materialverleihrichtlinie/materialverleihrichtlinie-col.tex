%\addchap[Richtlinie für den Materialverleih des Studentenrates der TU~Dresden]{Richtlinie für den Materialverleih\\ des Studentenrates der TU Dresden}
\markright{Richtlinie für den Materialverleih}
\setcounter{section}{0} % ist nötig um den Paragrafenzähler zurücksetzen
\begin{multicols}{2}
 

\section{Ausleihberechtigte}

\Abs \Satz Material wird vorwiegend an den Studentenrat, Fachschaftsräte und anerkannte Hochschulgruppen verliehen\. Eine Vertreterin der jeweiligen Institution muss als Verantwortliche benannt werden\. Sie ist die Ausleihende.

\Abs \Satz Eine Reservierung des Materials ist für Hochschulgruppen und Fachschaftsräte maximal drei Wochen im Voraus möglich.



\section{Ausleihbedingungen}

\Abs \Satz Bei Abholung ist in einem Übergabeprotokoll festzuhalten, welche Gegenstände ausgeliehen werden und wie hoch die jeweilige Kaution und gegebenenfalls das Nutzungsentgelt ist\. Das Übergabeprotokoll enthält ferner den Zustand aller ausgeliehenen Gegenstände.

\Abs \Satz Das Material wird grundsätzlich über eine Nacht verliehen\. Es muss am folgenden Werktag um spätestens elf Uhr zurückgegeben werden.

\Abs \Satz Bei Verlust, Diebstahl oder Beschädigung haftet die Ausleihende\. Von letzterem ausgenommen sind nur Verschleißteile und im Übergabeprotokoll festgehaltene Beschädigungen.

\Abs \Satz Für ausgeliehenes Material wird eine Kaution erhoben\. Die Kaution ist gegen Quittung bei Abholung in bar zu hinterlegen und wird bei ordnungsgemäßer Rückgabe erstattet.

\Abs \Satz Neben Gründen nach Abs.~1~und~3 werden Teile der Kaution bei verspäteter Rückgabe oder Verschmutzung einbehalten.

\Abs \Satz Bei Material mit hohen laufenden Kosten oder hohen Anschaffungskosten wird ein Nutzungsentgelt erhoben\. Es ist bei Abholung in bar zu zahlen\. Die so eingenommenen Gelder werden für Wartung oder Neubeschaffung des Materials verwendet.



\section{Schlussbestimmungen}

\Abs \Satz Der Materialbestand des Studentenrates wird in einer öffentlich zugänglichen Liste aufgeführt\. Die Liste beinhaltet die genaue Bezeichnung des Materials, die Höhe der Kaution und gegebenenfalls das Nutzungsentgelt\. Sie enthält ferner eine Auflistung, in welchen Fällen Kaution einbehalten wird und wie hoch der entsprechende Teil ist.

\Abs \Satz Die Höhe der Kaution und gegebenenfalls das Nutzungsentgelt wird von der Geschäftsführung festgelegt\. Ob für einen Teil des Materialbestands ein Nutzungsentgelt erhoben wird, entscheidet die Geschäftsführung\. Von §~1~und~§~2~Abs.~2,5~und~6~Satz~1 kann nur im Einzelfall auf Beschluss der Geschäftsführung abgewichen werden\. Die Verwaltung des Materialverleihs wird über das Servicebüro geregelt.

\end{multicols}

\nopagebreak
\vspace{1cm}
Inkraftgetreten am 29.~Juni~2006.
\\


\footnotesize
Geändert am 17.~Juli~2008\\
alt §~2~Abs.~1 gestrichen;\\
§~2~Abs.~3~S.~2 "`Tag"' durch "`Werktag"' ersetzt;\\
§~3~Abs.~2~S.~3 "`3"' durch "`2"' ersetzt.


\normalsize
~\\*[4cm]
\begin{center}
\hspace*{\fill}
\parbox{7cm}{Christoph Lüdecke\\GF Soziales}
\hfill\parbox{7cm}{Alexander Kasten\\GF Öffentliches}
\hspace*{\fill}
\end{center}